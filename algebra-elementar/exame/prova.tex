\documentclass{prova}

\usepackage{amssymb}

\renewcommand{\sin}{\mbox{sen}}
\newcommand{\ra}{\rightarrow}
\newcommand{\lra}{\leftrightarrow}
\newcommand{\Ra}{\Rightarrow}
\newcommand{\LRa}{\Leftrightarrow}
\renewcommand{\lnot}{\sim}
\newcommand{\larg}{\vdash}

\professor{Prof. Adriano Barbosa}
\disciplina{\'Algebra Elementar}
\avaliacao{Final}
\curso{Matem\'atica}
\data{14/12/2018}

\begin{document}
	\cabecalho{5}  % o numero 5 indica a qnt de quadros na tabela de nota

	\textbf{Todas as respostas devem ser justificadas.}
	\begin{questionario}
        \q{Determine o valor l\'ogico de $p$ e $q$ sabendo que:}
            \begin{questionario}
                \qq{$V(p \land q)=F$}
                \qq{$V(p \lra q)=F$ e $V(\lnot p \lor q)=F$}
                \qq{$V(p \ra q)=V$ e $V(p \land q)=F$}
            \end{questionario}
        \q{Mostre que se $n$ \'e um inteiro \'{\i}mpar, ent\~ao $5n-3$ \'e um inteiro par.}
            \begin{questionario}
                \qq{De forma direta.}
                \qq{Por absurdo.}
            \end{questionario}
        \q{}
            \begin{questionario}
                \qq{D\^e um exemplo de conjuntos $A$ e $B$ tais que $A\subset B$
                e verifique se $B^C\subset A^C$.}
                \qq{Mostre que $A\subset B \Ra B^C\subset A^C$, quaiquer que
                sejam $A$ e $B$.}
            \end{questionario}
        \q{Prove, usando indu\c{c}\~ao, que $\displaystyle\frac{1}{1\cdot2} +
        \frac{1}{2\cdot3} + \frac{1}{3\cdot4} + \cdots + \frac{1}{n(n+1)} =
        \frac{n}{n+1},\ \forall n\in\mathbb{N}$.}
        \q{Sabendo que $A \times A$ tem nove elementos, $(a,b)\in A\times A$ e
        $(b,c)\in A\times A$, determine os sete elementos restantes de $A\times
        A$.}
	\end{questionario}
\end{document}
