\documentclass{prova}

\renewcommand{\sin}{\mbox{sen}}
\newcommand{\ra}{\rightarrow}
\newcommand{\lra}{\leftrightarrow}
\newcommand{\Ra}{\Rightarrow}
\newcommand{\LRa}{\Leftrightarrow}
\renewcommand{\lnot}{\sim}
\newcommand{\larg}{\vdash}

\professor{Prof. Adriano Barbosa}
\disciplina{\'Algebra Elementar}
\avaliacao{P1}
\curso{Matem\'atica}
\data{05/10/2018}

\begin{document}
	\cabecalho{5}  % o numero 5 indica a qnt de quadros na tabela de nota

	\textbf{Todas as respostas devem ser justificadas.}
	\begin{questionario}
        \q{Determine $V(p)$ sabendo que:}
            \begin{questionario}
                \qq{$V(q)=F$ e $V(p\lor q)=V$}
                \qq{$V(q)=F$ e $V(p\ra q)=F$}
                \qq{$V(q)=V$ e $V(p\lra q)=F$}
            \end{questionario}
        \q{Dada a proposi\c{c}\~ao $P: (p\ra q)\land\lnot q\ra\lnot p$.}
            \begin{questionario}
                \qq{Construa a tabela verdade da proposi\c{c}\~ao $P$.}
                \qq{A proposi\c{c}\~ao $P$ \'e uma tautologia?}
            \end{questionario}
        \q{Dada a proposi\c{c}\~ao $P:$ se $n$ \'e \'{\i}mpar, ent\~ao $5n-3$ \'e par.}
            \begin{questionario}
                \qq{Escreva a forma rec\'{\i}proca de $P$.}
                \qq{Escreva a forma contrapositiva de $P$.}
            \end{questionario}
        \q{}
            \begin{questionario}
                \qq{Se n\~ao chover ($C$) e Jo\~ao acordar cedo ($A$), ent\~ao ele ir\'a a
                praia ($P$). Jo\~ao n\~ao foi a praia. O que podemos concluir?}
                \qq{Se Ana vai a festa ($A$), ent\~ao Byanca vai a festa ($B$). Se
                Byanca vai a festa, ent\~ao Christian n\~ao vai a festa ($\lnot C$). Se
                Christian vai a festa, ent\~ao Dieine n\~ao vai a festa ($\lnot D$).
                Se Byanca vai a festa, Eduardo n\~ao vai a festa ($\lnot E$). Se
                Dieine n\~ao vai a festa, ent\~ao Felipe vai a festa (F). Eduardo
                vai a festa e Felipe n\~ao vai a festa. Quem vai a festa?}
            \end{questionario}
        \q{Mostre que se $m$ e $n$ s\~ao inteiros pares, ent\~ao $m+n$ \'e um inteiro
        par.}
	\end{questionario}
\end{document}
