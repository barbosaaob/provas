\documentclass{prova}

\usepackage{amssymb}

\renewcommand{\sin}{\mbox{sen}}
\newcommand{\ra}{\rightarrow}
\newcommand{\lra}{\leftrightarrow}
\newcommand{\Ra}{\Rightarrow}
\newcommand{\LRa}{\Leftrightarrow}
\renewcommand{\lnot}{\sim}
\newcommand{\larg}{\vdash}

\professor{Prof. Adriano Barbosa}
\disciplina{\'Algebra Elementar}
\avaliacao{P2}
\curso{Matem\'atica}
\data{30/11/2018}

\begin{document}
	\cabecalho{5}  % o numero 5 indica a qnt de quadros na tabela de nota

	\textbf{Todas as respostas devem ser justificadas.}
	\begin{questionario}
        \q{D\^e um exemplo de conjuntos $A$, $B$ e $C$ tais que:}
            \begin{questionario}
                \qq{$A\subset B$, $B\not\subset C$ e $A\subset C$}
                \qq{$A\not\subset B$, $B\not\subset C$ e $A\subset C$}
                \qq{$A\in B$, $B\not\in C$ e $A\not\in C$}
            \end{questionario}
        \q{Mostre que $(A\cup B)^C = A^C \cap B^C$.}
        \q{Use o princ\'{\i}pio de indu\c{c}\~ao para mostrar que:}
            \[1^2+2^2+3^2+\cdots+n^2 = \displaystyle\frac{n(n+1)(2n+1)}{6}\]
        \q{Sejam $A=\{x\in\mathbb{R}\ |\ 0\le x\le 5\}$, $B=(2,5]$ e $C=(3,6)$.
        Determine os conjuntos:}
            \begin{questionario}
                \qq{$A-B$}
                \qq{$A\cap B$}
                \qq{$A-C$}
                \qq{$A\cup C$}
            \end{questionario}
        \q{Dados os conjuntos $A=\{2,6,12\}$ e $B=\{2,3,4\}$:}
            \begin{enumerate}
                \qq{Determine os elementos do conjunto $A\times B$.}
                \qq{Determine os elementos da rela\c{c}\~ao $R=\{(x,y)\in A\times B\
                |\ y \mbox{ divide } x\}$.}
            \end{enumerate}
	\end{questionario}
\end{document}
