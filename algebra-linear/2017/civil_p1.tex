\documentclass[a4paper,5pt]{amsbook}
%%%%%%%%%%%%%%%%%%%%%%%%%%%%%%%%%%%%%%%%%%%%%%%%%%%%%%%%%%%%%%%%%%%%%

\usepackage{booktabs}
\usepackage{graphicx}
% \usepackage[]{float}
\usepackage{amssymb}
% \usepackage{amsfonts}
% \usepackage[]{amsmath}
% \usepackage[]{epsfig}
% \usepackage[brazil]{babel}
% \usepackage[utf8]{inputenc}
% \usepackage{verbatim}
%\usepackage[]{pstricks}
%\usepackage[notcite,notref]{showkeys}
\usepackage{subcaption}
\usepackage{systeme}

%%%%%%%%%%%%%%%%%%%%%%%%%%%%%%%%%%%%%%%%%%%%%%%%%%%%%%%%%%%%%%

\newcommand{\sen}{\text{sen}}
\newcommand{\ds}{\displaystyle}
\newcommand{\tr}{\,\mbox{tr}\,}

%%%%%%%%%%%%%%%%%%%%%%%%%%%%%%%%%%%%%%%%%%%%%%%%%%%%%%%%%%%%%%%%%%%%%%%%

\setlength{\textwidth}{16cm} %\setlength{\topmargin}{-0.1cm}
\setlength{\leftmargin}{1.2cm} \setlength{\rightmargin}{1.2cm}
\setlength{\oddsidemargin}{0cm}\setlength{\evensidemargin}{0cm}

%%%%%%%%%%%%%%%%%%%%%%%%%%%%%%%%%%%%%%%%%%%%%%%%%%%%%%%%%%%%%%%%%%%%%%%%

% \renewcommand{\baselinestretch}{1.6}
% \renewcommand{\thefootnote}{\fnsymbol{footnote}}
% \renewcommand{\theequation}{\thesection.\arabic{equation}}
% \setlength{\voffset}{-50pt}
% \numberwithin{equation}{chapter}

%%%%%%%%%%%%%%%%%%%%%%%%%%%%%%%%%%%%%%%%%%%%%%%%%%%%%%%%%%%%%%%%%%%%%%%

\begin{document}
\thispagestyle{empty}
\hspace{-0.6cm}
\begin{minipage}[p]{0.14\linewidth}
	\includegraphics[scale=0.24]{../../ufgd.png}
\end{minipage}
\begin{minipage}[p]{0.7\linewidth}
\begin{tabular}{c}
\toprule{}
{{\bf UNIVERSIDADE FEDERAL DA GRANDE DOURADOS}}\\
{{\bf Prof.\ Adriano Barbosa}}\\

{{\bf \'{A}lgebra Linear e Geometria Anal\'{\i}tica --- Avalia\c{c}\~ao P1}}\\

\midrule{}
Eng. Civil\hspace{5cm}05 de Julho de 2017 \\
\bottomrule{}
\end{tabular}
\vspace{-0.45cm}
%
\end{minipage}
\begin{minipage}[p]{0.15\linewidth}
\begin{flushright}
\def\arraystretch{1.2}
\begin{tabular}{|c|c|}  % chktex 44
\hline\hline  % chktex 44
1 & \hspace{1.2cm} \\
\hline  % chktex 44
2& \\
\hline  % chktex 44
3& \\
\hline  % chktex 44
4&  \\
\hline  % chktex 44
5&  \\
\hline  % chktex 44
{\small Total}&  \\
\hline\hline  % chktex 44
\end{tabular}
\end{flushright}
\end{minipage}

%------------------------
\vspace{0.5cm}
{\bf Aluno(a):}\dotfill{}  % chktex 36
%----------------------------

\vspace{0.2cm}
%%%%%%%%%%%%%%%%%%%%%%%%%%%%%%%%   formulario  inicio  %%%%%%%%%%%%%%%%%%%%%%%%%%%%%%%%
\begin{enumerate}
	\vspace{0.5cm}
	\item Mostre que se o sistema
		\[\systeme{
				x+y+2z=a,
				2x+y+3z=b,
				x+z=c}\]
	tem solu\c{c}\~ao, ent\~ao as constantes $a$, $b$ e $c$ devem satisfazer $c = b-a$.

	\vspace{0.5cm}
	\item Determine $a$, $b$, $c$ e $d$ tais que
		\[\begin{bmatrix}
			a-b & b+c \\
			3d+c & 2a-4d
		\end{bmatrix}
		=
		\begin{bmatrix}
			8 & 1 \\
			7 & 6
		\end{bmatrix}\]
	
	\vspace{0.5cm}
	\item Encontre os valores de $\lambda$ para os quais $\det(A)=0$, com
		$A=\begin{bmatrix}
			\lambda-2 & 1 \\
			-5 & \lambda+4
		\end{bmatrix}$.

	\vspace{0.5cm}
	\item Determine o valor de $n$ para que o \^angulo entre as retas seja
		$\frac{\pi}{6}$:

	$r_1:\begin{array}{l}
		\ds\frac{x-6}{4} = \frac{y-5}{5} = \frac{z-3}{3}
	\end{array}$
	\ \ \ \ \ e\ \ \ \ \ 
	$r_2:\left\{\begin{array}{l}
		y = nx + 5 \\
		\frac{z}{2} = x - 1
	\end{array}\right.$

	\vspace{0.5cm}
	\item Dados os planos $\pi_1: y=3-x$ e
		$\pi_2: \left\{\begin{array}{l}
			x = 2+2t-2s \\
			y = 1-t-s \\
			z = 2-2t+2s
		\end{array}\right.$
	\begin{enumerate}
		\item Escreva a equa\c{c}\~ao impl\'{\i}cita de $\pi_2$.
		\item Calcule a interse\c{c}\~ao entre $\pi_1$ e $\pi_2$.
	\end{enumerate}

\end{enumerate}

\begin{flushright}
	\vspace{1cm}
	\textit{Boa Prova!}
\end{flushright}

\end{document}
