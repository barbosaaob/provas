\documentclass{prova}

\usepackage{amssymb}
\usepackage{amsmath}
\usepackage{gensymb}
\usepackage[utf8]{inputenc}

\professor{Prof.\@ Adriano Barbosa}
\disciplina{\'{A}lgebra Linear e Geometria Anal\'{\i}tica}
\avaliacao{P2}
\curso{Eng. Civil}
\data{16/08/2017}

\begin{document}
	\cabecalho{5}  % o numero 5 indica a qnt de quadros na tabela de nota
	\begin{questionario}
		\q{Dado o conjunto $\{(1,2,3),(3,2,1),(1,1,1)\} \subset \mathbb{R}^3$,
			verifique:}
			\begin{questionario}
				\qq{Os vetores s\~ao LI ou LD?}
				\qq{Podemos escrever qualquer vetor de $\mathbb{R}^3$ como
					combina\c{c}\~ao linear dos vetores dados?}
				\qq{Os vetores formam uma base de $\mathbb{R}^3$?}
			\end{questionario}
			\q{Dada $T: \mathbb{R}^3 \rightarrow \mathbb{R}^3$, $T(x,y,z) = (-x+z,-2x+2z,-3x+2z)$}
			\begin{questionario}
				\qq{Calcule a matriz can\^onica de $T$.}
				\qq{Calcule o n\'ucleo e a imagem de $T$.}
				\qq{$T$ \'e injetiva? E sobrejetiva?}
			\end{questionario}
		\q{Calcule a matriz can\^onica da proje\c{c}\~ao ortogonal de vetores de $\mathbb{R}^2$ sobre a
			reta $y=x$.}
		\q{Combine as matrizes de rota\c{c}\~ao de $45\degree$ e $60\degree$ para obter a
			matriz de rota\c{c}\~ao de $105\degree$.}
		\q{Dada
			$A=\begin{bmatrix}
				4 & 0 & 1 \\
				2 & 3 & 2 \\
				1 & 0 & 4
			\end{bmatrix}$, calcule:}
			\begin{questionario}
				\qq{Seus autovalores.}
				\qq{Seus autovetores.}
				\qq{Uma matriz $P$ que diagonaliza $A$.}
				\qq{$P^{-1}AP$.}
			\end{questionario}
	\end{questionario}
\end{document}
