\documentclass{prova}

\usepackage{amssymb}
\usepackage{mathtools}
\usepackage{gensymb}
\usepackage[utf8]{inputenc}

\professor{Prof.\@ Adriano Barbosa}
\disciplina{\'{A}lgebra Linear e Geometria Anal\'{\i}tica}
\avaliacao{Exame}
\curso{Matemática e Eng. Civil}
\data{06/09/2017}

\begin{document}
	\cabecalho{5}  % o numero 5 indica a qnt de quadros na tabela de nota
	\begin{questionario}
		\q{Quantas soluções possui um sistema linear cuja matriz aumentada é
			$A=\begin{bmatrix}
				2 & -1 & 3 \\
				1 & 4 & 2 \\
				1 & -5 & 1 \\
				4 & 16 & 8
			\end{bmatrix}?$}
		\q{Encontre a equação implícita do plano que contém as retas
			$r_1:\left\{
			\begin{matrix*}[l]
				x = t \\
				y = 0 \\
				z = 4+2t
			\end{matrix*}
			\right.$
			e
			$r_2:\left\{
			\begin{matrix*}[l]
				x = -2s \\
				y = -s \\
				z = 4-2s
			\end{matrix*}
			\right.$.
		}
		\q{Encontre a transformação linear $T:\mathbb{R}^2 \rightarrow
			\mathbb{R}^2$ tal qua $T(1,1) = (3,2)$ e $T(0,-2)=(0,1)$.}
		\q{Descreva em palavras o efeito geométrico de multiplicar um vetor $(x,y)$ pela matriz $A$:}
			\begin{questionario}
				\qq{$A=\begin{bmatrix}
					2 & 0 \\
					0 & 2
				\end{bmatrix}$}
				\qq{$A=\begin{bmatrix}
					1 & 0 \\
					0 & -1
				\end{bmatrix}$}
				\qq{$A=\begin{bmatrix}
					2 & 0 \\
					0 & 2
				\end{bmatrix}
				\cdot{}
				\begin{bmatrix}
					1 & 0 \\
					0 & -1
				\end{bmatrix}$}
			\end{questionario}
		\q{Seja
		$A = \begin{bmatrix}
			0 & 1 & 0 \\
			0 & 0 & 1 \\
			-1 & 0 & 0
		\end{bmatrix}$:
		}
			\begin{questionario}
				\qq{Quantos autovalores distintos a matriz $A$ possui?
					Quais são eles?}
				\qq{Calcule os autovetores de $A$.}
				\qq{É possível obter uma base de $\mathbb{R}^3$ formada
					por autovetores de $A$? Exiba uma base, se possível.}
			\end{questionario}
	\end{questionario}
\end{document}
