\documentclass[a4paper,5pt]{amsbook}
%%%%%%%%%%%%%%%%%%%%%%%%%%%%%%%%%%%%%%%%%%%%%%%%%%%%%%%%%%%%%%%%%%%%%

\usepackage{booktabs}
\usepackage{graphicx}
% \usepackage[]{float}
\usepackage{amssymb}
% \usepackage{amsfonts}
% \usepackage[]{amsmath}
% \usepackage[]{epsfig}
% \usepackage[brazil]{babel}
% \usepackage[utf8]{inputenc}
% \usepackage{verbatim}
%\usepackage[]{pstricks}
%\usepackage[notcite,notref]{showkeys}
\usepackage{subcaption}
\usepackage{systeme}

%%%%%%%%%%%%%%%%%%%%%%%%%%%%%%%%%%%%%%%%%%%%%%%%%%%%%%%%%%%%%%

\newcommand{\sen}{\text{sen}}
\newcommand{\ds}{\displaystyle}
\newcommand{\tr}{\,\mbox{tr}\,}

%%%%%%%%%%%%%%%%%%%%%%%%%%%%%%%%%%%%%%%%%%%%%%%%%%%%%%%%%%%%%%%%%%%%%%%%

\setlength{\textwidth}{16cm} %\setlength{\topmargin}{-0.1cm}
\setlength{\leftmargin}{1.2cm} \setlength{\rightmargin}{1.2cm}
\setlength{\oddsidemargin}{0cm}\setlength{\evensidemargin}{0cm}

%%%%%%%%%%%%%%%%%%%%%%%%%%%%%%%%%%%%%%%%%%%%%%%%%%%%%%%%%%%%%%%%%%%%%%%%

% \renewcommand{\baselinestretch}{1.6}
% \renewcommand{\thefootnote}{\fnsymbol{footnote}}
% \renewcommand{\theequation}{\thesection.\arabic{equation}}
% \setlength{\voffset}{-50pt}
% \numberwithin{equation}{chapter}

%%%%%%%%%%%%%%%%%%%%%%%%%%%%%%%%%%%%%%%%%%%%%%%%%%%%%%%%%%%%%%%%%%%%%%%

\begin{document}
\thispagestyle{empty}
\hspace{-0.6cm}
\begin{minipage}[p]{0.14\linewidth}
	\includegraphics[scale=0.24]{../../ufgd.png}
\end{minipage}
\begin{minipage}[p]{0.7\linewidth}
\begin{tabular}{c}
\toprule{}
{{\bf UNIVERSIDADE FEDERAL DA GRANDE DOURADOS}}\\
{{\bf Prof.\ Adriano Barbosa}}\\

{{\bf \'{A}lgebra Linear e Geometria Anal\'{\i}tica --- Avalia\c{c}\~ao P1}}\\

\midrule{}
F\'{\i}sica\hspace{5cm}30 de Junho de 2017 \\
\bottomrule{}
\end{tabular}
\vspace{-0.45cm}
%
\end{minipage}
\begin{minipage}[p]{0.15\linewidth}
\begin{flushright}
\def\arraystretch{1.2}
\begin{tabular}{|c|c|}  % chktex 44
\hline\hline  % chktex 44
1 & \hspace{1.2cm} \\
\hline  % chktex 44
2& \\
\hline  % chktex 44
3& \\
\hline  % chktex 44
4&  \\
\hline  % chktex 44
5&  \\
\hline  % chktex 44
{\small Total}&  \\
\hline\hline  % chktex 44
\end{tabular}
\end{flushright}
\end{minipage}

%------------------------
\vspace{0.5cm}
{\bf Aluno(a):}\dotfill{}  % chktex 36
%----------------------------

\vspace{0.2cm}
%%%%%%%%%%%%%%%%%%%%%%%%%%%%%%%%   formulario  inicio  %%%%%%%%%%%%%%%%%%%%%%%%%%%%%%%%
\begin{enumerate}
	\vspace{0.5cm}
	\item Dado o sistema linear:
		\[\systeme[xyz]{
				x + 2y - 3z = 4,
				3x - y + 5z = 2,
				4x + y + {(a^2-14)}z = a+2
			}\]
		\vspace{0.3cm}
		\begin{enumerate}
			\item Para quais valores de $a$ o sistema n\~ao admite solu\c{c}\~ao? Justifique.
			\item Para quais valores de $a$ o sistema admite solu\c{c}\~ao \'unica? Justifique.
			\item Para quais valores de $a$ o sistema admite infinitas solu\c{c}\~oes? Justifique.
		\end{enumerate}

	\vspace{0.5cm}
	\item Sendo
		\[A = \begin{bmatrix}
				3 & 0 \\
				-1 & 2 \\
			\end{bmatrix},
		B = \begin{bmatrix}
				4 & -1 \\
				0 & 2
			\end{bmatrix},
		C = \begin{bmatrix}
				6 & 1 & 3 \\
				-1 & 1 & 2 \\
				4 & 1 & 3
			\end{bmatrix}\]
		\begin{enumerate}
			\item Calcule $\tr(C^T+C)$.
			\item Calcule $\tr(A^{-1}B)$.
			\item Calcule $\tr(C^T+C + A^{-1}B)$, se poss\'{\i}vel. Justifique.
		\end{enumerate}

	\vspace{0.5cm}
	% \item Mostre que $x=0$ e $x=2$ s\~ao solu\c{c}\~ao da equa\c{c}\~ao
	% 	\[\begin{vmatrix}
	% 		x^2 & x & 2 & x \\
	% 		2 & 1 & 1 & 1 \\
	% 		0 & 0 & -5 & 0 \\
	% 		0 & 0 & 3 & 1
	% 	\end{vmatrix}=0\]
	\item Mostre que $x=0$ e $x=2$ s\~ao solu\c{c}\~ao da equa\c{c}\~ao
		\[\begin{vmatrix}
			x^2 & x & 2 \\
			2 & 1 & 1 \\
			0 & 0 & -5
		\end{vmatrix}=0\]

	\vspace{0.5cm}
	\item Dados $A = (3, 4, -2)$ e 
	$r:\left\{\begin{array}{l}
		x = 1 + t \\
		y = 2 - t \\
		z = 4 + 2t
	\end{array}\right.$. Determine a equa\c{c}\~ao param\'etrica da reta que passa por
$A$ e \'e perpendicular a $r$.

	\vspace{0.5cm}
	\item Dado o plano $\pi: 3x + y - z = 1$, calcule:
	\begin{enumerate}
		\item O valor de $k$ para que o ponto $P = (k, 2, k-1)$ perten\c{c}a a
			$\pi$.
		\item O valor de $k$ para que o plano $\pi_1:kx-4y+4z = 7$ seja
			paralelo a $\pi$.
	\end{enumerate}
\end{enumerate}

\begin{flushright}
	\vspace{1cm}
	\textit{Boa Prova!}
\end{flushright}

\end{document}
