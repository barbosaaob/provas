\documentclass[a4paper,5pt]{amsbook}
%%%%%%%%%%%%%%%%%%%%%%%%%%%%%%%%%%%%%%%%%%%%%%%%%%%%%%%%%%%%%%%%%%%%%

\usepackage{booktabs}
\usepackage{graphicx}
% \usepackage[]{float}
\usepackage{amssymb}
% \usepackage{amsfonts}
% \usepackage[]{amsmath}
% \usepackage[]{epsfig}
% \usepackage[brazil]{babel}
% \usepackage[utf8]{inputenc}
% \usepackage{verbatim}
%\usepackage[]{pstricks}
%\usepackage[notcite,notref]{showkeys}
\usepackage{subcaption}
\usepackage{systeme}

%%%%%%%%%%%%%%%%%%%%%%%%%%%%%%%%%%%%%%%%%%%%%%%%%%%%%%%%%%%%%%

\newcommand{\sen}{\text{sen}}
\newcommand{\ds}{\displaystyle}
\newcommand{\tr}{\,\mbox{tr}\,}

%%%%%%%%%%%%%%%%%%%%%%%%%%%%%%%%%%%%%%%%%%%%%%%%%%%%%%%%%%%%%%%%%%%%%%%%

\setlength{\textwidth}{16cm} %\setlength{\topmargin}{-0.1cm}
\setlength{\leftmargin}{1.2cm} \setlength{\rightmargin}{1.2cm}
\setlength{\oddsidemargin}{0cm}\setlength{\evensidemargin}{0cm}

%%%%%%%%%%%%%%%%%%%%%%%%%%%%%%%%%%%%%%%%%%%%%%%%%%%%%%%%%%%%%%%%%%%%%%%%

% \renewcommand{\baselinestretch}{1.6}
% \renewcommand{\thefootnote}{\fnsymbol{footnote}}
% \renewcommand{\theequation}{\thesection.\arabic{equation}}
% \setlength{\voffset}{-50pt}
% \numberwithin{equation}{chapter}

%%%%%%%%%%%%%%%%%%%%%%%%%%%%%%%%%%%%%%%%%%%%%%%%%%%%%%%%%%%%%%%%%%%%%%%

\begin{document}
\thispagestyle{empty}
\hspace{-0.6cm}
\begin{minipage}[p]{0.14\linewidth}
	\includegraphics[scale=0.24]{../../ufgd.png}
\end{minipage}
\begin{minipage}[p]{0.7\linewidth}
\begin{tabular}{c}
\toprule{}
{{\bf UNIVERSIDADE FEDERAL DA GRANDE DOURADOS}}\\
{{\bf Prof.\ Adriano Barbosa}}\\

{{\bf T\'opicos de \'{A}lgebra Linear --- Avalia\c{c}\~ao P1}}\\

\midrule{}
Matem\'atica\hspace{5cm}07 de Julho de 2017 \\
\bottomrule{}
\end{tabular}
\vspace{-0.45cm}
%
\end{minipage}
\begin{minipage}[p]{0.15\linewidth}
\begin{flushright}
\def\arraystretch{1.2}
\begin{tabular}{|c|c|}  % chktex 44
\hline\hline  % chktex 44
1 & \hspace{1.2cm} \\
\hline  % chktex 44
2& \\
\hline  % chktex 44
3& \\
\hline  % chktex 44
4&  \\
\hline  % chktex 44
5&  \\
\hline  % chktex 44
{\small Total}&  \\
\hline\hline  % chktex 44
\end{tabular}
\end{flushright}
\end{minipage}

%------------------------
\vspace{0.5cm}
{\bf Aluno(a):}\dotfill{}  % chktex 36
%----------------------------

\vspace{0.2cm}
%%%%%%%%%%%%%%%%%%%%%%%%%%%%%%%%   formulario  inicio  %%%%%%%%%%%%%%%%%%%%%%%%%%%%%%%%
\begin{enumerate}
	\vspace{0.5cm}
	\item Em $\mathbb{R}^2$, defina as seguintes opera\c{c}\~oes de soma e produto
		por escalar:
		\[(x_1,y_1) + (x_2,y_2) = (3y_1+3y_2, -x_1-x_2)\]
		\[\alpha(x,y) = (3\alpha y, -\alpha x).\]
	Verifique se $\mathbb{R}^2$, com estas opera\c{c}\~oes, \'e um espa\c{c}o vetorial.

	\vspace{0.5cm}
	\item Sejam $F_1$ e $F_2$ subespa\c{c}os de um espa\c{c}o vetorial $E$. Verifique
		quais das seguintes afirma\c{c}\~oes s\~ao verdadeiras justificando sua
		resposta:
		\begin{enumerate}
			\item $F = F_1 \cup F_2$ \'e um subespa\c{c}o de $E$.
			% \item $F = F_1 \times F_2$ \'e um subespa\c{c}o de $E$.
			\item $F = \{v_1+v_2; v_1 \in F_1 \text{ e } v_2 \in F_2\}$ \'e um subespa\c{c}o
				de $E$.
		\end{enumerate}

	\vspace{0.5cm}
	\item Seja $X$ um conjunto infinito. Para cada $a \in X$, seja $f_a: X
		\rightarrow \mathbb{R}$ a fun\c{c}\~ao tal que $f_a(a) = 1$ e $f_a(x) = 0$ se
		$x \neq a$. Prove que o conjunto $Y \subset \mathcal{F}(X;\mathbb{R})$
		formado por estas fun\c{c}\~oes \'e linearmente independente. Prove ainda que
		$Y$ n\~ao gera $\mathcal{F}(X;\mathbb{R})$.

	\vspace{0.5cm}
	\item Uma matriz quadrada $A = [a_{ij}]$ chama-se sim\'etrica
		(respectivamente anti-sim\'etrica) quando $a_{ij} = a_{ji}$
		(respectivamente $a_{ij} = -a_{ji}$) para todo $i$ e todo $j$. Prove
		que o conjunto $S$ das matrizes sim\'etricas e o conjunto $A$ das
		matrizes anti-sim\'etricas $n \times n$ s\~ao subespa\c{c}os vetoriais de
		$\mathbb{M}(n \times n)$ e que se tem $\mathbb{M}(n \times n) = S
		\oplus A$.

	\vspace{0.5cm}
	\item Seja $\mathcal{M}(n \times n)$ o espa\c{c}o vetorial das matrizes $n
		\times n$ e seja $B$ uma matriz $n \times n$ fixa. Se
		\[T(A) = AB - BA\]
		verifique se $T$ \'e uma transforma\c{c}\~ao linear de $\mathcal{M}(n \times
		n)$ em $\mathcal{M}(n \times n)$.

	\vspace{0.5cm}
	\item Seja $A: E \rightarrow F$ uma transforma\c{c}\~ao linear
		\begin{enumerate}
			\item Se os vetores $Av_1, Av_2, \ldots, Av_m \in F$ s\~ao L.I.,
				prove que $v_1, v_2, \ldots, v_m \in E$ tamb\'em s\~ao L.I..
			\item Se $F = E$ e os vetores $Av_1, Av_2, \ldots, Av_m$ geram $E$,
				prove que $v_1, v_2, \ldots, v_m$ geram $E$. [Dica: use o item (a)]
			\item Valem as rec\'{\i}procas de (a) e (b)? Seria (b) verdadeira com $F
				\neq E$?
		\end{enumerate}
\end{enumerate}

\begin{flushright}
	\vspace{1cm}
	\textit{Boa Prova!}
\end{flushright}

\end{document}
