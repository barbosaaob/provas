\documentclass{prova}

\usepackage{amssymb}
\usepackage{amsmath}
\usepackage{gensymb}
\usepackage[utf8]{inputenc}

\professor{Prof.\@ Adriano Barbosa}
\disciplina{T\'opicos de \'{A}lgebra Linear}
\avaliacao{P2}
\curso{Matem\'atica}
\data{25/08/2017}

\begin{document}
	\cabecalho{5}  % o numero 5 indica a qnt de quadros na tabela de nota
	\begin{questionario}
		\q{Sejam $E$ e $F$ espa\c{c}os vetoriais e $B:E\rightarrow F$ um
			isomorfismo. Mostre que $T:\mathcal{L}(E;E) \rightarrow
			\mathcal{L}(F;F)$, $T(A) = BAB^{-1}$ \'e um isomorfismo.}
		\q{Seja $A$ um operador linear em $\mathbb{R}^2$ definido por $A(x,y) = (-y, x)$.}
			\begin{questionario}
				\qq{Calcule a matriz de $A$ relativa a base can\^onica de $\mathbb{R}^2$.}
				\qq{Calcule a matriz de $A$ relativa a base $\beta = \{(1,2), (1,-1)\}\subset\mathbb{R}^2$.}
				\qq{Mostre que para todo n\'umero real $c$ o operador $A-cI$ \'e invert\'{\i}vel.}
				\qq{Mostre que se $\beta$ \'e uma base qualquer de ${\mathbb{R}}^2$
					e ${[A]}_{\beta}^{\beta} = [a_{ij}]$, ent\~ao $a_{12}a_{21} \ne 0$.\\
					(${[A]}_{\beta}^{\beta}$ denota a matriz de $A$ relativa a base $\beta$)
				}
			\end{questionario}
		\q{Mostre que se $F_1$ e $F_2$ s\~ao subespa\c{c}os invariantes por um
			operador linear $A:E\rightarrow E$, ent\~ao os subespa\c{c}os $F_1+F_2$ e
			$F_1\cap F_2$ tamb\'em s\~ao invariantes por $A$.}
		\q{Dado o operador linear $A:\mathbb{R}^2 \rightarrow \mathbb{R}^2$,
			$A(x,y) = (x+2y, 2x+y)$, encontre uma base de
			$\mathbb{R}^2$ tal que a matriz de $A$ relativa a essa base seja
			diagonal e escreva essa matriz diagonal.}
		\q{Suponha que o operador $A:\mathbb{R}^3 \rightarrow \mathbb{R}^3$ \'e
			n\~ao nulo e que $A^2 = 0$. Mostre que dim$Im(A) = 1$.}
	\end{questionario}
\end{document}
