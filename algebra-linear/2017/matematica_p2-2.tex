\documentclass{prova}

\usepackage{amssymb}
\usepackage{amsmath}
\usepackage{gensymb}
\usepackage[utf8]{inputenc}

\professor{Prof.\@ Adriano Barbosa}
\disciplina{\'{A}lgebra Linear e Geometria Anal\'{\i}tica}
\avaliacao{P2}
\curso{Matemática -- 2ª chamada}
\data{24/08/2017}

\begin{document}
	\cabecalho{5}  % o numero 5 indica a qnt de quadros na tabela de nota
	\begin{questionario}
		\q{Dado o conjunto $\{(3,3,3,3),(0,2,2,2),(0,0,1,1)\} \subset \mathbb{R}^4$,
			verifique:}
			\begin{questionario}
				\qq{Os vetores s\~ao LI ou LD?}
				\qq{Podemos escrever qualquer vetor de $\mathbb{R}^4$ como
					combina\c{c}\~ao linear dos vetores dados?}
				\qq{Os vetores formam uma base de $\mathbb{R}^4$?}
			\end{questionario}
			\q{Dada $T: \mathbb{R}^2 \rightarrow \mathbb{R}^2$, $T(x,y) = (2x+y, 3x+4y)$}
			\begin{questionario}
				\qq{Calcule a matriz can\^onica de $T$.}
				\qq{Calcule o n\'ucleo e a imagem de $T$.}
				\qq{$T$ \'e invertível? Calcule sua inversa, se possível.}
			\end{questionario}
		\q{Verifique se $T_1 \circ T_2 = T_2 \circ T_1$, onde
			$T_1:\mathbb{R}^2 \rightarrow \mathbb{R}^2$ é a rotação por um
			ângulo $\theta$ em torno da origem no sentido anti-horário e $T_2:\mathbb{R}^2 \rightarrow \mathbb{R}^2$
			é a projeção ortogonal sobre o eixo $x$.}
		\q{Determine a transformação linear resultante de uma escala de
			fator $2$ seguida de uma rotação de $45\degree$ em torno da
			origem no sentido anti-horário.}
		\q{Dada
			$A=\begin{bmatrix}
				1 & 0 & 3 \\
				1 & 2 & 1 \\
				0 & 0 & -2
			\end{bmatrix}$, calcule:}
			\begin{questionario}
				\qq{Seus autovalores.}
				\qq{Seus autovetores.}
				\qq{$A$ é diagonalizável? Justifique.}
				\qq{Encontre uma matriz $P$ que diagonaliza
					$A$, se possível.}
			\end{questionario}
	\end{questionario}
\end{document}
