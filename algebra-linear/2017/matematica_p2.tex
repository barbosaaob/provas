\documentclass{prova}

\usepackage{amssymb}
\usepackage{amsmath}
\usepackage{gensymb}
\usepackage[utf8]{inputenc}

\professor{Prof.\@ Adriano Barbosa}
\disciplina{\'{A}lgebra Linear e Geometria Anal\'{\i}tica}
\avaliacao{P2}
\curso{Matemática}
\data{23/08/2017}

\begin{document}
	\cabecalho{5}  % o numero 5 indica a qnt de quadros na tabela de nota
	\begin{questionario}
		\q{Dado o conjunto $\{(3,3,3),(0,2,2),(0,0,1)\} \subset \mathbb{R}^3$,
			verifique:}
			\begin{questionario}
				\qq{Os vetores s\~ao LI ou LD?}
				\qq{Podemos escrever qualquer vetor de $\mathbb{R}^3$ como
					combina\c{c}\~ao linear dos vetores dados?}
				\qq{Os vetores formam uma base de $\mathbb{R}^3$?}
			\end{questionario}
			\q{Dada $T: \mathbb{R}^2 \rightarrow \mathbb{R}^3$, $T(x,y) = (x,y,x+y)$}
			\begin{questionario}
				\qq{Calcule a matriz can\^onica de $T$.}
				\qq{Calcule o n\'ucleo e a imagem de $T$.}
				\qq{$T$ \'e injetiva? E sobrejetiva?}
			\end{questionario}
		\q{Mostre que os vetores $T(v)$ e $v - T(v)$ s\~ao ortogonais,
			onde $T:\mathbb{R}^2 \rightarrow \mathbb{R}^2$ \'e uma
			proje\c{c}\~ao ortogonal sobre os eixos coordenados}.
		\q{Determine a transformação linear resultante de uma rotação
			de $45\degree$ em torno da origem no sentido anti-horário
			seguida de uma reflexão em torno do eixo $y$.}
		\q{Dada
			$A=\begin{bmatrix}
				0 & 0 & -2 \\
				1 & 2 & 1 \\
				1 & 0 & 3
			\end{bmatrix}$, calcule:}
			\begin{questionario}
				\qq{Seus autovalores.}
				\qq{Seus autovetores.}
				\qq{Uma matriz $P$ que diagonaliza $A$.}
				\qq{$P^{-1}AP$.}
			\end{questionario}
	\end{questionario}
\end{document}
