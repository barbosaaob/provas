\documentclass{prova}

\usepackage{amssymb}
\usepackage{amsmath}
\usepackage{gensymb}
\usepackage[utf8]{inputenc}

\professor{Prof.\@ Adriano Barbosa}
\disciplina{\'{A}lgebra Linear e Geometria Anal\'{\i}tica}
\avaliacao{PS}
\curso{Matemática}
\data{30/08/2017}

\begin{document}
	\cabecalho{5}  % o numero 5 indica a qnt de quadros na tabela de nota

	\textbf{(\ \ \ ) Avaliação P1:}
	\begin{questionario}
		\qq{É possível obter valores de $a$, $b$ e $c$ tais que
			$a(1,1,1) + b(1,1,0) +c(1,0,1) = (6,3,4)$?}
		\qq{Dadas as matrizes
			$A=\begin{bmatrix}
				0 & 0 & 2 \\
				0 & 1 & 0 \\
				1 & 0 & 1
			\end{bmatrix}$
			e
			$B=\begin{bmatrix}
				0 & -2 & 1 \\
				1 & 0 & 0 \\
				0 & 1 & 1
			\end{bmatrix}$.
			Calcule a matriz $X$ que satisfaz a equação $AB = BX$.
		}
		\qq{Resolva a equação
			$\begin{vmatrix}
				x & 0 & 2 \\
				0 & x-1 & 0 \\
				1 & 0 & x-1
			\end{vmatrix}=0$.
		}
		\qq{Determine $k$ de modo que que a reta $r$ seja paralela a reta $s$:

			$r:\left\{
			\begin{matrix}
				x = 1 + (k+2)t \\
				y = 2 + (2k)t \\
				z = 3 + 2t
			\end{matrix}
			\right.$, \hspace{2cm}
			$s:\left\{
			\begin{matrix}
				x = 1 + t \\
				y = -2 + 2t \\
				z = 2 - 2s
			\end{matrix}
			\right.$
		}
		\qq{Encontre a equação implícita do plano que passa pela origem
			e é ortogonal ao plano $x-y+z=0$.}
	\end{questionario}

	\noindent\line(1,0){450}

	\textbf{(\ \ \ ) Avaliação P2}
	\begin{questionario}
		\qq{Verifique se os vetores $(1,1,1)$, $(1,1,0)$ e $(1,0,1)$ formam uma base de $R^3$.}
		\qq{Dada a transformação linear $T:\mathbb{R}^2 \rightarrow \mathbb{R}^2$, $T(x,y) = (x+2y, 2x+y)$:}
			\begin{questionario}
				\qq{Verifique se $T$ é injetiva e se é sobrejetiva.}
				\qq{Calcule a inversa de $T$, se possível.}
			\end{questionario}
		\qq{Calcule a transformação linear resultante da aplicação de
			uma projeção ortogonal sobre o eixo $x$, seguida de uma escala
			de fator $2$, seguida de uma rotação de $30\degree$ no sentido
			anti-horário em torno da origem.}
		\qq{Calcule a matriz canônica da projeção ortogonal sobre a reta $y = -x$.}
		\qq{Dada a matriz
			$A=\begin{bmatrix}
				0 & 0 & 2 \\
				0 & 1 & 0 \\
				1 & 0 & 1
			\end{bmatrix}$,
			calcule seus autovalores e seus autovetores.}
	\end{questionario}
\end{document}
