\documentclass{prova}

\usepackage{amsmath}
\usepackage{amsfonts}
\usepackage[inline]{enumitem}

\setlength{\textheight}{25cm}

\DeclareMathOperator{\sen}{sen}
\DeclareMathOperator{\tg}{tg}
\DeclareMathOperator{\tr}{tr}
\newcommand{\ds}{\displaystyle}

\professor{Prof.\@ Adriano Barbosa}
\disciplina{\'Algebra Linear e Geometria Anal\'{\i}tica}
\avaliacao{Final}
\curso{Eng. de Alimentos}
\data{07/11/2022}

\begin{document}
	\cabecalho{5}  % o numero 5 indica a qnt de quadros na tabela de nota

    \textbf{Todas as respostas devem ser justificadas.}

    \begin{questionario}
        \q{Determine o valor de $k$ para que o sistema admita uma \'unica solu\c{c}\~ao
        e apresente essa solu\c{c}\~ao. }
            \[\left\{\begin{array}{rcl}
                -4x+3y & = & 2 \\
                5x-4y & = & 0 \\
                2x-y & = & k
            \end{array}\right.\]
        \q{Seja $A = \left[\begin{array}{cc}
					2 & x^2 \\
					2x-1 & 0
           \end{array}\right]$. Se $A^T = A$, qual o valor de $x$?}
        \q{Encontre a equa\c{c}\~ao da reta que passa pelo ponto $A=(1,2,3)$ e \'e perpendicular ao plano}

            \[\pi:\left\{\begin{array}{l}
		    		x = 1+s-2t \\
		    		y = 1-t \\
		    		z = 4+2s-2t
            \end{array}\right., s, t\in{} \mathbb{R}.\]
        \q{Descreva em palavras o efeito geom\'etrico de multiplicar um vetor
           $(x,y)$ pela matriz $A$.}
	    	\begin{enumerate}
	    		\vspace{0.3cm}
	    		\item $A = \left[\begin{array}{cc}
	    					2 & 0 \\
	    					0 & 2
	    				\end{array}\right]$
	    		\vspace{0.3cm}
	    		\item $A = \left[\begin{array}{cc}
	    					1 & 0 \\
	    					0 & -1
	    				\end{array}\right]$
	    		\vspace{0.3cm}
	    		\item $A = \left[\begin{array}{cc}
	    					2 & 0 \\
	    					0 & 2
	    				\end{array}\right]\cdot
	    				\left[\begin{array}{cc}
	    					1 & 0 \\
	    					0 & -1
	    				\end{array}\right]$
	    	\end{enumerate}
        \q{Seja}
	    	\[A = \left[\begin{array}{ccc}
	    			0 & 1 & 0 \\
	    			0 & 0 & 1 \\
	    			-1 & 0 & 0
	    		\end{array}\right]\]
	    	\begin{enumerate}
	    		\vspace{0.3cm}
	    		\item Quantos autovalores reais a matriz $A$ possui?
	    		\vspace{0.3cm}
	    		\item Calcule os autovetores de $A$.
	    		%\vspace{0.3cm}
	    		%\item A matriz $A$ \'e diagonaliz\'avel? Justifique.
	    	\end{enumerate}
    \end{questionario}
\end{document}
