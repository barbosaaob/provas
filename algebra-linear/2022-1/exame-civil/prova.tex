\documentclass{prova}

\usepackage{amsmath}
\usepackage{amsfonts}
\usepackage[inline]{enumitem}

\setlength{\textheight}{25cm}

\DeclareMathOperator{\sen}{sen}
\DeclareMathOperator{\tg}{tg}
\DeclareMathOperator{\tr}{tr}
\newcommand{\ds}{\displaystyle}

\professor{Prof.\@ Adriano Barbosa}
\disciplina{\'Algebra Linear e Geometria Anal\'{\i}tica}
\avaliacao{Final}
\curso{Eng. Civil}
\data{07/11/2022}

\begin{document}
	\cabecalho{5}  % o numero 5 indica a qnt de quadros na tabela de nota

    \textbf{Todas as respostas devem ser justificadas.}

    \begin{questionario}
        \q{Determine $k$ para que o sistema admita \'unica solu\c{c}\~ao.
            \[\left\{\begin{array}{rcl}
                -4x + 3y & = & 2 \\
                5x - 4y & = & 0 \\
                2x - y & = & k
            \end{array}\right.\]}
        \q{Encontre a equa\c{c}\~ao param\'etrica do plano $\pi$ que passa pelo ponto
            $A=(2,0,0)$ e tem vetor normal $n=(3,-2,-1)$.}
        \q{Combine as matrizes de rota\c{c}\~ao de $30^{\circ}$ e $45^{\circ}$ no
            sentido anti-hor\'ario para obter a matriz de rota\c{c}\~ao de $75^{\circ}$ no
            sentido anti-hor\'ario. N\~ao utilize calculadora!}
        \q{Determine se a transforma\c{c}\~ao linear associada a matriz can\^onica
            $A=\left[\begin{array}{cc} 1 & -1 \\ 2 & 0 \\ 3 & -4\end{array}\right]$ \'e
            injetiva.}
        \q{Dada a matriz $A=\left[\begin{array}{cc} 0 & 2 \\ 1 &
            0\end{array}\right]$, calcule os autovalores de $AA^T$ e $A^TA$.}
    \end{questionario}
\end{document}
