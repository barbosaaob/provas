\documentclass{prova}

\usepackage{amsmath}
\usepackage{amsfonts}
\usepackage[inline]{enumitem}

\setlength{\textheight}{25cm}

\DeclareMathOperator{\sen}{sen}
\DeclareMathOperator{\tg}{tg}
\DeclareMathOperator{\tr}{tr}
\newcommand{\ds}{\displaystyle}

\professor{Prof.\@ Adriano Barbosa}
\disciplina{\'Algebra Linear e Geometria Anal\'{\i}tica}
\avaliacao{P1}
\curso{Eng. de Alimentos}
\data{22/08/2022}

\begin{document}
	\cabecalho{5}  % o numero 5 indica a qnt de quadros na tabela de nota

    \textbf{Todas as respostas devem ser justificadas.}

    \begin{questionario}
        \q{Usando as matrizes abaixo, resolva as opera\c{c}\~oes abaixo:}
    	\[A = \begin{bmatrix}
    			3 & 1 \\
    			2 & 2 \\
    			1 & 0
    		\end{bmatrix},
    	B = \begin{bmatrix}
    			4 & 2 \\
    			0 & 1
    		\end{bmatrix},
    	C = \begin{bmatrix}
    			1 & 5 & 2 \\
    			-1 & 0 & 1 \\
    			3 & 2 & 4
        \end{bmatrix}\]
            \begin{enumerate*}
                \qq{$A^T$}
                \hspace{0.5cm}
                \hspace{0.5cm}
                \qq{$AA^T$}
                \hspace{0.5cm}
                \hspace{0.5cm}
                \qq{$B^{-1}$}
                \hspace{0.5cm}
                \hspace{0.5cm}
                \qq{$\tr(AA^T+C)$}
            \end{enumerate*}
        \q{Sabendo que}
            $\begin{vmatrix}
    			a & b & c \\
    			d & e & f \\
    			g & h & i
    		\end{vmatrix}=-1$,
    		encontre

            \begin{enumerate*}
                \qq{}
    				$\begin{vmatrix}
    					d & e & f \\
    					g & h & i \\
    					a & b & c
    				\end{vmatrix}$
                \hspace{0.5cm}
                \hspace{0.5cm}
                \qq{}
    				$\begin{vmatrix}
    					a+2g & b+2h & c+2i \\
    					d & e & f \\
    					g & h & i
    				\end{vmatrix}$
                \hspace{0.5cm}
                \hspace{0.5cm}
                \qq{}
    				$\begin{vmatrix}
    					a & b & c \\
    					3d & 3e & 3f \\
    					g & h & i
    				\end{vmatrix}$
                \hspace{0.5cm}
                \hspace{0.5cm}
                \qq{}
    				$\begin{vmatrix}
    					a & a & c \\
    					d & d & f \\
    					g & g & i
    				\end{vmatrix}$
            \end{enumerate*}

            justificando sua resposta.
        \q{Suponha $\langle u, v \times w \rangle = 7$. Encontre}

	    	\begin{enumerate*}
	    		\item $\langle u, w \times v \rangle$
                \hspace{0.5cm}
                \hspace{0.5cm}
	    		\item $\langle v \times w, u \rangle$
                \hspace{0.5cm}
                \hspace{0.5cm}
	    		\item $\langle w, u \times v \rangle$
	    	\end{enumerate*}

            justificando sua resposta.
        \q{Dados $A = (0,0,1)$ e 
            $r:\left\{\begin{array}{l}
		        x = 1 + t \\
		        y = 2 + 2t \\
		        z = 4 + 3t
	        \end{array}\right.$.
            Determine a equa\c{c}\~ao param\'etrica da reta que passa por $A$ e \'e
            perpendicular a $r$ e ao eixo $y$.}
        \q{Encontre a equa\c{c}\~ao impl\'{\i}cita do plano paralelo ao plano $yz$ e que
            intersecta o eixo $x$ em $-3$.}
    \end{questionario}
\end{document}
