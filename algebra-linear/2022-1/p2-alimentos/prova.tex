\documentclass{prova}

\usepackage{amsmath}
\usepackage{amsfonts}
\usepackage[inline]{enumitem}

\setlength{\textheight}{25cm}

\DeclareMathOperator{\sen}{sen}
\DeclareMathOperator{\tg}{tg}
\DeclareMathOperator{\tr}{tr}
\newcommand{\ds}{\displaystyle}

\professor{Prof.\@ Adriano Barbosa}
\disciplina{\'Algebra Linear e Geometria Anal\'{\i}tica}
\avaliacao{P2}
\curso{Eng. de Alimentos}
\data{24/10/2022}

\begin{document}
	\cabecalho{5}  % o numero 5 indica a qnt de quadros na tabela de nota

    \textbf{Todas as respostas devem ser justificadas.}

    \begin{questionario}
        \q{Determine uma base para os subespa\c{c}os de $\mathbb{R}^3$ abaixo.}
            \begin{questionario}
                \qq{o plano $z=y-x$.}
                \qq{a reta $x=2t$, $y=-t$, $z=-t$.}
            \end{questionario}
        \q{Seja $T:\mathbb{R}^2\rightarrow\mathbb{R}^4$, $T(x,y) =
           (y-x, x-y, y-2x, 2x-y)$.}
            \begin{questionario}
                \qq{Determine a matriz can\^onica de $T$.}
                \qq{Determine o n\'ucleo de $T$. $T$ \'e injetiva?}
                \qq{Determine a imagem de $T$. $T$ \'e sobrejetiva?}
            \end{questionario}
        \q{Encontre a transforma\c{c}\~ao linear resultante da aplica\c{c}\~ao de uma
           reflex\~ao em torno do eixo $x$, seguida de uma proje\c{c}\~ao ortogonal no
           eixo $y$, seguida de uma escala de raz\~ao $\alpha=2$.}
        \q{Calcule os autovalores e autovetores de
        $T:\mathbb{R}^3\rightarrow\mathbb{R}^3$, $T(x,y,z)=(x+z, 2y, y+z)$.}
        \q{Determine se a matriz $A=\left[\begin{array}{cc} 4 & 3 \\ 2 &
           1\end{array}\right]$ \'e diagonaliz\'avel.}
    \end{questionario}
\end{document}
