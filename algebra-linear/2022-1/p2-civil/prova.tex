\documentclass{prova}

\usepackage{amsmath}
\usepackage{amsfonts}
\usepackage[inline]{enumitem}

\setlength{\textheight}{27cm}
\setlength{\topmargin}{-2.7cm}

\DeclareMathOperator{\sen}{sen}
\DeclareMathOperator{\tg}{tg}
\DeclareMathOperator{\tr}{tr}
\newcommand{\ds}{\displaystyle}

\professor{Prof.\@ Adriano Barbosa}
\disciplina{\'Algebra Linear e Geometria Anal\'{\i}tica}
\avaliacao{P2}
\curso{Eng. Civil}
\data{24/10/2022}

\begin{document}
	\cabecalho{5}  % o numero 5 indica a qnt de quadros na tabela de nota

    \textbf{Todas as respostas devem ser justificadas.}

    \textbf{Escolha cinco exerc\'{\i}cios e anote suas escolhas no quadro de notas acima.}

    \begin{questionario}
        \q{Sejam $u=(1,2,-2)$ e $v=(0,1,-1)$. Determine:}
            \begin{enumerate}
                \qq{$u-2v$.}
                \qq{$\|u-2v\|$.}
                \qq{Se $u$ e $v$ s\~ao ortogonais.}
                \qq{Uma dire\c{c}\~ao ortogonal a $u$ e $v$.}
            \end{enumerate}
        \q{Encontre a reta que passa pelo ponto m\'edio do segmento de extremos
           $A=(2,1,-1)$ e $B=(0,1,0)$ e que seja perpendicular a ele.}
        \q{Determine a interse\c{c}\~ao entre os planos $\pi_1: x+y+z=6$ e $\pi_2:
           y=3-x$.}
        \q{Encontre as coordenadas de $w$ em rela\c{c}\~ao as bases abaixo:}
            \begin{questionario}
                \qq{$\beta=\{(1,0), (1,1)\}$, $w=(1,2)$.}
                \qq{$\beta=\{(1,1,1), (1,0,2), (0,-1,1)\}$, $w=(1,2,3)$.}
            \end{questionario}
        \q{Determine uma base para os subespa\c{c}os de $\mathbb{R}^3$ abaixo.}
            \begin{questionario}
                \qq{o plano $x-y+z=0$.}
                \qq{a reta $x=t$, $y=-t$, $z=0$.}
            \end{questionario}
        \q{Seja $T:\mathbb{R}^2\rightarrow\mathbb{R}^3$, $T(x,y) =
           (y-x, x+y, 3y-x)$.}
            \begin{questionario}
                \qq{Determine a matriz can\^onica de $T$.}
                \qq{Determine o n\'ucleo de $T$. $T$ \'e injetiva?}
                \qq{Determine a imagem de $T$. $T$ \'e sobrejetiva?}
            \end{questionario}
        \q{Encontre a transforma\c{c}\~ao linear resultante da aplica\c{c}\~ao de uma
           reflex\~ao em torno do eixo $y$, seguida de uma rota\c{c}\~ao de
           $\frac{\pi}{2}$ radianos no sentido anti-hor\'ario, seguida de uma
           proje\c{c}\~ao ortogonal no eixo $y$.}
        \q{Calcule os autovalores e autovetores de
           $T:\mathbb{R}^2\rightarrow\mathbb{R}^2$, $T(x,y)=(x+y, -y)$.}
    \end{questionario}
\end{document}
