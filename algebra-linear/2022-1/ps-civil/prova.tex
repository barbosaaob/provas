\documentclass{prova}

\usepackage{amsmath}
\usepackage{amsfonts}
\usepackage[inline]{enumitem}

\setlength{\textheight}{27cm}
\setlength{\topmargin}{-2.7cm}

\DeclareMathOperator{\sen}{sen}
\DeclareMathOperator{\tg}{tg}
\DeclareMathOperator{\tr}{tr}
\newcommand{\ds}{\displaystyle}

\professor{Prof.\@ Adriano Barbosa}
\disciplina{\'Algebra Linear e Geometria Anal\'{\i}tica}
\avaliacao{PS}
\curso{Eng. Civil}
\data{31/11/2022}

\begin{document}
	\cabecalho{5}  % o numero 5 indica a qnt de quadros na tabela de nota

    \textbf{Todas as respostas devem ser justificadas.}

    \begin{questionario}
        \q{Determine a interse\c{c}\~ao entre os planos $\pi_1: x+y=1$ e $\pi_2:
           y+z=1$.}
        \q{Determine uma base para os subespa\c{c}os de $\mathbb{R}^3$ abaixo.}
            \begin{questionario}
                \qq{o plano $x+y-z=0$.}
                \qq{a reta $x=-t$, $y=t$, $z=0$.}
            \end{questionario}
        \q{Seja $T:\mathbb{R}^2\rightarrow\mathbb{R}^3$, $T(x,y) =
           (x+y, x-y, 3x+y)$.}
            \begin{questionario}
                \qq{Determine a matriz can\^onica de $T$.}
                \qq{Determine o n\'ucleo de $T$. $T$ \'e injetiva?}
                \qq{Determine a imagem de $T$. $T$ \'e sobrejetiva?}
            \end{questionario}
        \q{Encontre a transforma\c{c}\~ao linear resultante da aplica\c{c}\~ao de uma
           proje\c{c}\~ao ortogonal no eixo $y$ seguida de uma reflex\~ao em torno do eixo
           $y$ seguida de uma rota\c{c}\~ao de $\frac{\pi}{2}$ radianos no sentido
           anti-hor\'ario.}
        \q{Calcule os autovalores e autovetores de
           $T:\mathbb{R}^2\rightarrow\mathbb{R}^2$, $T(x,y)=(x, x-y)$.}
    \end{questionario}
\end{document}
