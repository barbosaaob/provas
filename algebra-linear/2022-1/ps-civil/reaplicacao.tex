\documentclass{prova}

\usepackage{amsmath}
\usepackage{amsfonts}
\usepackage[inline]{enumitem}

%\setlength{\textheight}{27cm}
%\setlength{\topmargin}{-2.7cm}

\DeclareMathOperator{\sen}{sen}
\DeclareMathOperator{\tg}{tg}
\DeclareMathOperator{\tr}{tr}
\newcommand{\ds}{\displaystyle}

\professor{Prof.\@ Adriano Barbosa}
\disciplina{\'Algebra Linear e Geometria Anal\'{\i}tica}
\avaliacao{PS 2}
\curso{Eng. Civil}
\data{03/11/2022}

\begin{document}
	\cabecalho{5}  % o numero 5 indica a qnt de quadros na tabela de nota

    \textbf{Todas as respostas devem ser justificadas.}

    \begin{questionario}
        \q{Encontre a equa\c{c}\~ao impl\'{\i}cita do plano que cont\'em as retas}

            $r_1:\left\{\begin{array}{l}
                y = 2x - 3 \\
                z = -x + 2
            \end{array}\right.$
            \ \ \ \ \ e\ \ \ \ \ 
            $r_2:\left\{\begin{array}{l}
                \frac{x-1}{3} = z-1 \\
                y = -1
            \end{array}\right.$
        \q{Determine uma base e a dimens\~ao dos subespa\c{c}os de $\mathbb{R}^4$:}
            \begin{questionario}
                \qq{Conjunto dos vetores da forma $(a, b, a, b)$,
                    $a,b\in\mathbb{R}$}
                \qq{Conjunto dos vetores da forma $(a, a-b, b+c, c)$,
                    $a,b,c\in\mathbb{R}$}
            \end{questionario}
        \q{Determine se o operador linear
           $T:\mathbb{R}^2\rightarrow\mathbb{R}^2$, $T(x,y)=(4x+3y, x+2y)$ \'e
           invert\'{\i}vel e calcule sua inversa, se poss\'{\i}vel.}

           Lembre que: $\ds A=\left[\begin{array}{cc} a & b \\ c & d\end{array}\right]
               \Rightarrow A^{-1}=\frac{1}{\det{A}}\left[\begin{array}{cc} d &
               -b \\ -c & a\end{array}\right]$
        \q{Encontre a matriz can\^onica da transforma\c{c}\~ao linear resultante de uma
           rota\c{c}\~ao de $\frac{\pi}{3}$ radianos no sentido anti-hor\'ario seguida de uma
           reflex\~ao em torno do eixo $x$.}
        \q{Calcule os autovalores e autovetores de
           $T:\mathbb{R}^2\rightarrow\mathbb{R}^2$, $T(x,y)=(x, x-y)$.}
    \end{questionario}
\end{document}
