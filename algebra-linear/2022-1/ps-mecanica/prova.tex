\documentclass{prova}

\usepackage{amsmath}
\usepackage{amsfonts}
\usepackage[inline]{enumitem}
\usepackage{systeme}

\setlength{\textheight}{27cm}
\setlength{\topmargin}{-2.5cm}

\DeclareMathOperator{\sen}{sen}
\DeclareMathOperator{\tg}{tg}
\DeclareMathOperator{\tr}{tr}
\newcommand{\ds}{\displaystyle}

\professor{Prof.\@ Adriano Barbosa}
\disciplina{\'Algebra Linear e Geometria Anal\'{\i}tica}
\avaliacao{PS}
\curso{Eng. Mec\^anica}
\data{31/10/2022}

\begin{document}
	\cabecalho{5}  % o numero 5 indica a qnt de quadros na tabela de nota

    \textbf{Todas as respostas devem ser justificadas.}
    \vspace{0.5cm}

    \textbf{Avalia\c{c}\~ao P1:}
    \vspace{-0.5cm}

    \begin{questionario}
        \q{Para quais valores de $a$ o sistema abaixo \textbf{n\~ao} admite solu\c{c}\~ao?}
            \[\systeme[xyz]{
                x + 2y - 3z = 4,
                3x - y - 2z = -5,
                4x + y + {(a^2-14)}z = a+2
            }\]
        \vspace{-1cm}

        \q{Encontre todos os valores de $a$, $b$ e $c$ tais que $A$ \'e sim\'etrica}
            \[A = \begin{bmatrix}
                2 & a-2b+2c & 2a+b+c \\
                2 & -1 & a+c \\
                0 & -1 & 1
            \end{bmatrix}\]
        \vspace{-1cm}

        \q{Dadas constantes reais $a\neq 0$ e $b\neq 0$, explique sem calcular
           o determinante por que quando $x=a$ e $x=0$ a igualdade abaixo
           \'e v\'alida.}
            \[\begin{vmatrix}
                x^2 & x & a \\
                a^2 & a & a \\
                0 & 0 & b
            \end{vmatrix}=0\]
        \vspace{-1cm}

        \q{Determine o valor de $n$ para que o \^angulo entre as retas seja}
            $\frac{\pi}{6}$:

            $r_1:\begin{array}{l}
                \ds\frac{x-2}{4} = \frac{y}{5} = \frac{z}{3}
            \end{array}$
            \ \ \ \ \ e\ \ \ \ \ 
            $r_2:\left\{\begin{array}{l}
                y = nx + 5 \\
                z = 2x - 2
            \end{array}\right.$
        \q{Encontre a equa\c{c}\~ao impl\'{\i}cita do plano que cont\'em as retas}

            $r_1:\left\{\begin{array}{l}
                y = 2x - 3 \\
                z = -x + 2
            \end{array}\right.$
            \ \ \ \ \ e\ \ \ \ \ 
            $r_2:\left\{\begin{array}{l}
                \frac{x-1}{3} = z-1 \\
                y = -1
            \end{array}\right.$
    \end{questionario}
    \vspace{0.5cm}

    \textbf{Avalia\c{c}\~ao P2:}
    \vspace{-0.5cm}


    \begin{questionario}
        \q{Determine uma base e a dimens\~ao dos subespa\c{c}os de $\mathbb{R}^4$:}
            \begin{questionario}
                \qq{Conjunto dos vetores da forma $(a, b, a, b)$,
                    $a,b\in\mathbb{R}$}
                \qq{Conjunto dos vetores da forma $(a, a-b, b+c, c)$,
                    $a,b,c\in\mathbb{R}$}
            \end{questionario}
        \vspace{-0.5cm}

        \q{Encontre a matriz can\^onica da transforma\c{c}\~ao linear resultante de uma
           rota\c{c}\~ao de $\frac{\pi}{3}$ radianos no sentido anti-hor\'ario seguida de uma
           reflex\~ao em torno do eixo $x$.}
        \vspace{-0.5cm}

        \q{Determine se o operador linear
           $T:\mathbb{R}^2\rightarrow\mathbb{R}^2$, $T(x,y)=(3x+4y, 2x+y)$ \'e
           invert\'{\i}vel e calcule sua inversa, se poss\'{\i}vel.}

           Lembre que: $\ds A=\left[\begin{array}{cc} a & b \\ c & d\end{array}\right]
               \Rightarrow A^{-1}=\frac{1}{\det{A}}\left[\begin{array}{cc} d &
               -b \\ -c & a\end{array}\right]$
        \vspace{-0.5cm}

        \q{Calcule os autovalores e os autovetores da matriz $A=\left[
            \begin{array}{ccc}
                3 & 2 & 1 \\
                0 & 2 & 0 \\
                0 & -1 & 2
            \end{array}\right]$.}
        \vspace{-0.5cm}

        \q{Calcule $A^{10}$, onde $A=\left[\begin{array}{cc}1 & 2 \\ 1 &
           0\end{array}\right]$.}
    \end{questionario}
\end{document}
