\documentclass{article}

\usepackage[inline]{enumitem}
\usepackage{gensymb}
\usepackage{amssymb}
\usepackage[brazil]{babel}

\begin{document}
\noindent{}\rule{\textwidth}{0.4pt}
\begin{center}
	\'{A}lgebra Linear\\
	Avalia\c{c}\~ao P2 --- 11/04/2016 \\
	Engenharia Mec\^anica \\
	\vspace{0.2cm}
	% Prof. Adriano Barbosa
\end{center}
Nome: \\
\noindent{}\rule{\textwidth}{0.4pt}

\begin{enumerate}
%%%%%%%%%%%%%%%%%%%%%%%%%%%%%%%%%%%%%%%%%%%%%
\item O conjunto
	$W = \left\{
		\left[\begin{array}{cc}
			a & b \\
			c & d
		\end{array}\right] \mbox{ com }
	a, b, c, d \in \mathbb{R} \mbox{ e } b = c + 1\right\}$
\'e um subespa\c{c}o das matrizes $2\times 2$? Em caso afirmativo, exiba os geradores.

%%%%%%%%%%%%%%%%%%%%%%%%%%%%%%%%%%%%%%%%%%%%%
\item Decida se os conjuntos de vetores s\~ao LI ou LD:
	\begin{enumerate}
		\item $\{(-1, 2, 4), (5, -10, -20)\}$ em $\mathbb{R}^3$.
		\item $\{-x^2+6, 4x^2+x+1\}$ em $\mathcal{P}_2$.
		\item $\left\{
			\left[\begin{array}{cc}
				-3 & 4 \\
				2 & 0
			\end{array}\right], 
			\left[\begin{array}{cc}
				3 & -4 \\
				-2 & 0
			\end{array}\right]\right\}$
		em $M(2, 2)$.
	\end{enumerate}

%%%%%%%%%%%%%%%%%%%%%%%%%%%%%%%%%%%%%%%%%%%%%
\item Decida se o conjunto $\{1+x+x^2, x+x^2, x^2\}$ \'e uma base de $\mathcal{P}_2$.

%%%%%%%%%%%%%%%%%%%%%%%%%%%%%%%%%%%%%%%%%%%%%
\item Encontre o dom\'inio e o contradom\'inio da transforma\c{c}\~ao	$$(x, y, z) \mapsto
	(3x-2y+4z, 5x-8y+z)$$ e determine se ela \'e linear.

%%%%%%%%%%%%%%%%%%%%%%%%%%%%%%%%%%%%%%%%%%%%%
\item Combine as matrizes de rota\c{c}\~ao de $30\degree$, $-30\degree$ e $45\degree$ para calcular a matrizes de
	rota\c{c}\~ao de $15\degree$ e $75\degree$.

%%%%%%%%%%%%%%%%%%%%%%%%%%%%%%%%%%%%%%%%%%%%%
\item Determine se a transforma\c{c}\~ao linear associada a matriz
	$A=\left[\begin{array}{cc}
			1 & -1 \\
			2 & 0 \\
			3 & -4
		\end{array}\right]$
	dada nas bases can\^onicas \'e injetiva.

\end{enumerate}
\end{document}
