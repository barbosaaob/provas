\documentclass[a4paper,5pt]{amsbook}
%%%%%%%%%%%%%%%%%%%%%%%%%%%%%%%%%%%%%%%%%%%%%%%%%%%%%%%%%%%%%%%%%%%%%

\usepackage{booktabs}
\usepackage{graphicx}
% \usepackage[]{float}
\usepackage{amssymb}
% \usepackage{amsfonts}
% \usepackage[]{amsmath}
% \usepackage[]{epsfig}
% \usepackage[brazil]{babel}
% \usepackage[utf8]{inputenc}
% \usepackage{verbatim}
%\usepackage[]{pstricks}
%\usepackage[notcite,notref]{showkeys}
\usepackage{subcaption}
\usepackage[inline]{enumitem}

%%%%%%%%%%%%%%%%%%%%%%%%%%%%%%%%%%%%%%%%%%%%%%%%%%%%%%%%%%%%%%

\newcommand{\sen}{\,\mbox{sen}\,}
\newcommand{\tg}{\,\mbox{tg}\,}
\newcommand{\cosec}{\,\mbox{cosec}\,}
\newcommand{\cotg}{\,\mbox{cotg}\,}
% \newcommand{\det}{\,\mbox{det}\,}
\newcommand{\ds}{\displaystyle}

%%%%%%%%%%%%%%%%%%%%%%%%%%%%%%%%%%%%%%%%%%%%%%%%%%%%%%%%%%%%%%%%%%%%%%%%

\setlength{\textwidth}{16cm} \setlength{\topmargin}{-1cm}
\setlength{\textheight}{25cm}
\setlength{\leftmargin}{1.2cm} \setlength{\rightmargin}{1.2cm}
\setlength{\oddsidemargin}{0cm}\setlength{\evensidemargin}{0cm}

%%%%%%%%%%%%%%%%%%%%%%%%%%%%%%%%%%%%%%%%%%%%%%%%%%%%%%%%%%%%%%%%%%%%%%%%

% \renewcommand{\baselinestretch}{1.6}
% \renewcommand{\thefootnote}{\fnsymbol{footnote}}
% \renewcommand{\theequation}{\thesection.\arabic{equation}}
% \setlength{\voffset}{-50pt}
% \numberwithin{equation}{chapter}

%%%%%%%%%%%%%%%%%%%%%%%%%%%%%%%%%%%%%%%%%%%%%%%%%%%%%%%%%%%%%%%%%%%%%%%

\begin{document}
\thispagestyle{empty}
\hspace{-0.6cm}
\begin{minipage}[p]{0.14\linewidth}
	\includegraphics[scale=0.24]{ufgd.png}
\end{minipage}
\begin{minipage}[p]{0.7\linewidth}
\begin{tabular}{c}
\toprule{}
{{\bf UNIVERSIDADE FEDERAL DA GRANDE DOURADOS}}\\
{{\bf Prof.\ Adriano Barbosa}}\\

{{\bf \'algebra Linear e Geometria Anal\'{\i}tica --- Exame}}\\

\midrule{}
Qu\'{\i}mica\hspace{5cm}13 de Abril de 2017 \\
\bottomrule{}
\end{tabular}
\vspace{-0.45cm}
%
\end{minipage}
\begin{minipage}[p]{0.15\linewidth}
\begin{flushright}
\def\arraystretch{1.2}
\begin{tabular}{|c|c|}  % chktex 44
\hline\hline  % chktex 44
1 & \hspace{1.2cm} \\
\hline  % chktex 44
2& \\
\hline  % chktex 44
3& \\
\hline  % chktex 44
4&  \\
\hline  % chktex 44
5&  \\
\hline  % chktex 44
{\small Total}&  \\
\hline\hline  % chktex 44
\end{tabular}
\end{flushright}
\end{minipage}

%------------------------
\vspace{0.5cm}
{\bf Aluno(a):}\dotfill{}  % chktex 36
%----------------------------

\vspace{0.2cm}
%%%%%%%%%%%%%%%%%%%%%%%%%%%%%%%%   formulario  inicio  %%%%%%%%%%%%%%%%%%%%%%%%%%%%%%%%
\begin{enumerate}
	\vspace{0.5cm}
	\item
		\begin{enumerate}
			\vspace{0.3cm}
			\item Encontre $a$, $b$, $c$ e $d$ tais que
				\[\left[\begin{array}{cc}
						a & b \\
						c & d
					\end{array}\right]
				\left[\begin{array}{cc}
						2 & 3 \\
						3 & 4
					\end{array}\right]
				=
				\left[\begin{array}{cc}
						1 & 0 \\
						0 & 1
					\end{array}\right]\]
			\vspace{0.3cm}
			\item Seja $A = \left[\begin{array}{cc}
						2 & x^2 \\
						2x-1 & 0
					\end{array}\right]$. Se $A^T = A$, qual o valor de $x$?
		\end{enumerate}

	\vspace{0.5cm}
	% \item Sejam $A$ e $B$ matrizes $n\times{}n$. Verifique se as afirma\c{c}\~oes abaixo s\~ao verdadeiras ou faltas.
	% 	\begin{enumerate}
	% 		\vspace{0.3cm}
	% 		\item $\det(AB) = \det(BA)$
	% 		\vspace{0.3cm}
	% 		\item $\det(A^T) = \det(A)$
	% 		\vspace{0.3cm}
	% 		\item $\det(A^2) = (\det(A))^2$
	% 	\end{enumerate}
	\item Determine $k$ para que o sistema admita solu\c{c}\~ao:
		\[\left\{\begin{array}{rcl}
				-4x + 3y & = & 2 \\
				 5x - 4y & = & 0 \\
				 2x -\ y & = & k
			\end{array}\right.\]

	\vspace{0.5cm}
	\item Encontre a equa\c{c}\~ao da reta que passa pelo ponto $(1,2,3)$ e \'e perpendicular ao plano
		$\left\{\begin{array}{l}
				x = 1+s-2t \\
				y = 1-t \\
				z = 4+2s-2t
			\end{array}\right.$, $s, t\in{} \mathbb{R}$.

	\vspace{0.5cm}
	\item Seja
		\[A = \left[\begin{array}{ccc}
				0 & 1 & 0 \\
				0 & 0 & 1 \\
				-1 & 0 & 0
			\end{array}\right]\]
		\begin{enumerate}
			\vspace{0.3cm}
			\item Quantos autovalores distintos a matriz $A$ possui?
			\vspace{0.3cm}
			\item Calcule os autovetores de $A$.
			\vspace{0.3cm}
			\item A matriz $A$ \'e diagonaliz\'avel? Justifique.
		\end{enumerate}

	\vspace{0.5cm}
	\item Descreva em palavras o efeito geom\'etrico de multiplicar um vetor $(x,y)$ pela matriz $A$.
		\begin{enumerate}
			\vspace{0.3cm}
			\item $A = \left[\begin{array}{cc}
						2 & 0 \\
						0 & 2
					\end{array}\right]$
			\vspace{0.3cm}
			\item $A = \left[\begin{array}{cc}
						1 & 0 \\
						0 & -1
					\end{array}\right]$
			\vspace{0.3cm}
			\item $A = \left[\begin{array}{cc}
						2 & 0 \\
						0 & 2
					\end{array}\right]\cdot
					\left[\begin{array}{cc}
						1 & 0 \\
						0 & -1
					\end{array}\right]$
		\end{enumerate}
\end{enumerate}

% \vfill{}
% F\'ormulas \'uteis:
% \[\begin{array}{llll}
% 	\vspace{0.3cm}
% 	\cosec(x) = \displaystyle\frac{1}{\sen(x)}, & \sec(x) = \displaystyle\frac{1}{\cos(x)}, & \cotg(x) = \displaystyle\frac{\cos(x)}{\sen(x)}, & \ds\tg(x) = \frac{\sen(x)}{\cos(x)}\\
% 	\vspace{0.3cm}
% 	\sen^2(x) + \cos^2(x) = 1, & \tg^2(x) + 1 = \sec^2(x), & 1 + \cotg^2(x) = \cosec^2(x) & \\
% 	\vspace{0.3cm}
% 	\sen^2(x) = \displaystyle\frac{1 - \cos(2x)}{2}, & \cos^2(x) = \displaystyle\frac{1 + \cos(2x)}{2} & & \\
% 	\multicolumn{2}{l}{\sen(x+y) = \displaystyle\sen(x)\cos(y) + \sen(y)\cos(x),} & \multicolumn{2}{l}{\cos(x+y) = \displaystyle\cos(x)\cos(y) - \sen(x)\sen(y)} \\
% 	& & & \\
% 	\multicolumn{2}{l}{\sen(x-y) = \displaystyle\sen(x)\cos(y) - \sen(y)\cos(x),} & \multicolumn{2}{l}{\cos(x-y) = \displaystyle\cos(x)\cos(y) + \sen(x)\sen(y)}
% \end{array}\]

\begin{flushright}
	\vspace{1cm}
	\textit{Boa Prova!}
\end{flushright}

\end{document}
