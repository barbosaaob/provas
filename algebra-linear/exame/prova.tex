\documentclass{article}

\usepackage[inline]{enumitem}
\usepackage{graphicx}
\usepackage{gensymb}
\usepackage{amssymb}
\usepackage[brazil]{babel}

\begin{document}
\noindent{}\rule{\textwidth}{0.4pt}
\begin{center}
	\'{A}lgebra Linear\\
	Exame --- 19/05/2016 \\
	Engenharia Mec\^anica \\
	\vspace{0.2cm}
	% Prof. Adriano Barbosa
\end{center}
Nome: \\
\noindent{}\rule{\textwidth}{0.4pt}

\begin{enumerate}
%%%%%%%%%%%%%%%%%%%%%%%%%%%%%%%%%%%%%%%%%%%%%
\item Calcule o posto da matriz 
$\left[\begin{array}{ccc}
	2 & -1 & 3 \\
	1 & 4 & 2 \\
	1 & -5 & 1 \\
	4 & 16 & 8
\end{array}\right]$.

%%%%%%%%%%%%%%%%%%%%%%%%%%%%%%%%%%%%%%%%%%%%%
\item Encontre a equa\c{c}\~ao impl\'icita do plano
$\left\{\begin{array}{l}
	x = 1+s-2t \\
	y = 1-t \\
	z = 4+2s-2t
\end{array}\right.$

%%%%%%%%%%%%%%%%%%%%%%%%%%%%%%%%%%%%%%%%%%%%%
\item Verifique se $\{(1-t)^2, 1-t, 1\}$ \'e uma base de $\mathcal{P}_2(t)$.

%%%%%%%%%%%%%%%%%%%%%%%%%%%%%%%%%%%%%%%%%%%%%
\item Dado o subespa\c{c}o
	$W = \left\{
		\left[\begin{array}{cc}
			2a & a+2b \\
			0 & a-b
		\end{array}\right] \mbox{ com }
	a, b \in \mathbb{R}\right\}$,
	verifique se as matrizes abaixo pertencem a $W$:
	\begin{enumerate}
		\item
			$\left[\begin{array}{cc}
				0 & -2 \\
				0 & 1
			\end{array}\right]$
		\item
			$\left[\begin{array}{cc}
				0 & 2 \\
				3 & 1
			\end{array}\right]$
	\end{enumerate}

%%%%%%%%%%%%%%%%%%%%%%%%%%%%%%%%%%%%%%%%%%%%%
\item Encontre a transforma\c{c}\~ao linear $T:\mathbb{R}^2 \rightarrow \mathbb{R}^2$
	tal que $T(1, 1) = (3, 2)$ e $T(0, -2) = (0, 1)$.

%%%%%%%%%%%%%%%%%%%%%%%%%%%%%%%%%%%%%%%%%%%%%
\item Seja
$A=\left[\begin{array}{ccc}
	0 & 1 & 0 \\
	0 & 0 & 1 \\
	-1 & 0 & 0
\end{array}\right]$:
\begin{enumerate}
	\item Calcule o polin\^omio caracter\'istico de $A$.
	\item Quantos autovalores distintos a matriz $A$ possui? Quais s\~ao eles?
	\item Calcule os autovetores de $A$.
\end{enumerate}

\end{enumerate}
\end{document}
