\documentclass{article}

\usepackage[inline]{enumitem}
\usepackage{graphicx}

\begin{document}
\noindent{}\rule{\textwidth}{0.4pt}
\begin{center}
	\'{A}lgebra Linear\\
	Avalia\c{c}\~ao P1 --- 18/02/2016 \\
	Engenharia Mec\^anica \\
	\vspace{0.2cm}
	% Prof. Adriano Barbosa
\end{center}
Nome: \\
\noindent{}\rule{\textwidth}{0.4pt}

\begin{enumerate}
%%%%%%%%%%%%%%%%%%%%%%%%%%%%%%%%%%%%%%%%%%%%%
\item Se $A_{n \times n}$ \'e uma matriz sim\'etrica, qual o resultado de $A^T-A$? Justifique.

%%%%%%%%%%%%%%%%%%%%%%%%%%%%%%%%%%%%%%%%%%%%%
\item Determine $k$, para que o sistema admita infinitas solu\c{c}\~oes

$\left\{\begin{array}{ccccccc}
	2x & - & 5y & + & 2z & = & 0 \\
	x & + & y & + & z & = & 0 \\
	2x & & & + & kz & = & 0
\end{array}\right.$

%%%%%%%%%%%%%%%%%%%%%%%%%%%%%%%%%%%%%%%%%%%%%
\item Verifique se as retas $r_1$ e $r_2$ possuem interse\c{c}\~ao:

$r_1:\left\{\begin{array}{l}
	y = 2x - 3 \\
	z = -x
\end{array}\right.$
\ \ \ \ \ e\ \ \ \ \ 
$r_2:\left\{\begin{array}{l}
	x = -t \\
	y = 4 - t \\
	z = 2 + 2t
\end{array}\right.$

%%%%%%%%%%%%%%%%%%%%%%%%%%%%%%%%%%%%%%%%%%%%%
\item Encontre a equa\c{c}\~ao param\'etrica da reta que passa pelo ponto $A = (3, 2, -1)$ e \'e simultaneamente perpendicular as retas $r_1$ e $r_2$:

$r_1:\left\{\begin{array}{l}
	x = 3 \\
	y = -1
\end{array}\right.$
\ \ \ \ \ e\ \ \ \ \ 
$r_2:\left\{\begin{array}{l}
	y = x - 3 \\
	z = -2x + 3
\end{array}\right.$

%%%%%%%%%%%%%%%%%%%%%%%%%%%%%%%%%%%%%%%%%%%%%
\item Encontre as equa\c{c}\~oes param\'etrica e geral do plano paralelo ao plano $yz$
	e que intersecta o eixo $x$ em $2$.

%%%%%%%%%%%%%%%%%%%%%%%%%%%%%%%%%%%%%%%%%%%%%
\item Encontre a equa\c{c}\~ao geral do plano que passa pelos pontos $(1, 0, 0)$, $(0, 1, 0)$ e $(0, 0, 1)$.

\end{enumerate}
\end{document}
