\documentclass[a4paper,5pt]{amsbook}
%%%%%%%%%%%%%%%%%%%%%%%%%%%%%%%%%%%%%%%%%%%%%%%%%%%%%%%%%%%%%%%%%%%%%

\usepackage{booktabs}
\usepackage{graphicx}
\usepackage{multicol}
\usepackage{textcomp}
% \usepackage[]{float}
\usepackage{amssymb}
% \usepackage{amsfonts}
% \usepackage[]{amsmath}
% \usepackage[]{epsfig}
% \usepackage[brazil]{babel}
% \usepackage[utf8]{inputenc}
% \usepackage{verbatim}
%\usepackage[]{pstricks}
%\usepackage[notcite,notref]{showkeys}
\usepackage{subcaption}
\usepackage[inline]{enumitem}

%%%%%%%%%%%%%%%%%%%%%%%%%%%%%%%%%%%%%%%%%%%%%%%%%%%%%%%%%%%%%%

\newcommand{\sen}{\,\mbox{sen}\,}
\newcommand{\tg}{\,\mbox{tg}\,}
\newcommand{\cosec}{\,\mbox{cosec}\,}
\newcommand{\cotg}{\,\mbox{cotg}\,}
\newcommand{\ds}{\displaystyle}

%%%%%%%%%%%%%%%%%%%%%%%%%%%%%%%%%%%%%%%%%%%%%%%%%%%%%%%%%%%%%%%%%%%%%%%%

\setlength{\textwidth}{16cm} % \setlength{\topmargin}{-2cm}
% \setlength{\textheight}{25cm}
\setlength{\leftmargin}{1.2cm} \setlength{\rightmargin}{1.2cm}
\setlength{\oddsidemargin}{0cm}\setlength{\evensidemargin}{0cm}

%%%%%%%%%%%%%%%%%%%%%%%%%%%%%%%%%%%%%%%%%%%%%%%%%%%%%%%%%%%%%%%%%%%%%%%%

% \renewcommand{\baselinestretch}{1.6}
% \renewcommand{\thefootnote}{\fnsymbol{footnote}}
% \renewcommand{\theequation}{\thesection.\arabic{equation}}
% \setlength{\voffset}{-50pt}
% \numberwithin{equation}{chapter}

%%%%%%%%%%%%%%%%%%%%%%%%%%%%%%%%%%%%%%%%%%%%%%%%%%%%%%%%%%%%%%%%%%%%%%%

\begin{document}
\thispagestyle{empty}
\hspace{-0.6cm}
\begin{minipage}[p]{0.14\linewidth}
	\includegraphics[scale=0.24]{ufgd.png}
\end{minipage}
\begin{minipage}[p]{0.7\linewidth}
\begin{tabular}{c}
\toprule{}
{{\bf UNIVERSIDADE FEDERAL DA GRANDE DOURADOS}}\\
{{\bf Prof.\ Adriano Barbosa}}\\

{{\bf Geometria Anal\'{\i}tica e \'{A}lgebra Linear --- Avalia\c{c}\~ao P2}}\\

\midrule{}
Qu\'{\i}mica\hspace{5cm}30 de Mar\c{c}o de 2017 \\
\bottomrule{}
\end{tabular}
\vspace{-0.45cm}
%
\end{minipage}
\begin{minipage}[p]{0.15\linewidth}
\begin{flushright}
\def\arraystretch{1.2}
\begin{tabular}{|c|c|}  % chktex 44
\hline\hline  % chktex 44
1 & \hspace{1.2cm} \\
\hline  % chktex 44
2& \\
\hline  % chktex 44
3& \\
\hline  % chktex 44
4&  \\
\hline  % chktex 44
5&  \\
\hline  % chktex 44
{\small Total}&  \\
\hline\hline  % chktex 44
\end{tabular}
\end{flushright}
\end{minipage}

%------------------------
\vspace{0.5cm}
{\bf Aluno(a):}\dotfill{}  % chktex 36
%----------------------------

\vspace{0.2cm}
%%%%%%%%%%%%%%%%%%%%%%%%%%%%%%%%   formulario  inicio  %%%%%%%%%%%%%%%%%%%%%%%%%%%%%%%%
\begin{enumerate}
	\vspace{0.5cm}
	\item Encontre a equa\c{c}\~ao param\'etrica da reta que passa por $A = (0,0,0)$ e
		\'e ortogonal as retas $\ds r_1: \frac{x}{2} = y = \frac{z-3}{2}$ e
		$r_2: \left\{\begin{array}{lll}
				x = 3t \\
				y = -t + 1 \\
				z = 2
			\end{array}\right.$

	\vspace{0.5cm}
	\item Encontre a equa\c{c}\~ao param\'etrica do plano que passa pelo ponto
		$(2, 0, 0)$ e tem vetor normal $(3, -2, -1)$.

	\vspace{0.5cm}
	\item Combine as matrizes de rota\c{c}\~ao de $30$\textdegree\ e $45$\textdegree\ no
		sentido anti-hor\'ario para obter a matriz de rota\c{c}\~ao de $75$\textdegree\ no
		sentido anti-hor\'ario.

	\vspace{0.5cm}
	\item Calcule os autovalores e autovetores da matriz
		\[A = \left[\begin{array}{rrr}
			1 & 0 & 3 \\
			1 & 2 & 1 \\
			0 & 0 & -2
		\end{array}\right].\]

	\vspace{0.5cm}
	\item Usando a matriz $A$ da quest\~ao anterior:
		\begin{enumerate}
			% \vspace{0.3cm}
			% \item Sabendo que $\{(-1,0,1),(-1,1,0),(0,1,0)\}$ \'e base de
			% 	$\mathbb{R}^3$, verifique que $A$ \'e diagonaliz\'avel.
			\vspace{0.3cm}
			\item Dada a matriz
				$P=\left[\begin{array}{ccc}
						-1 & -1 & 0 \\
						0 & 1 & 1 \\
						1 & 0 & 0
					\end{array}\right]$,
						verifique que
				$\left[\begin{array}{ccc}
						0 & 0 & 1 \\
						-1 & 0 & -1 \\
						1 & 1 & 1
					\end{array}\right]$
				\'e sua matriz inversa.
			\vspace{0.2cm}
			\item Calcule a matriz $D = P^{-1}AP$. A matriz $A$ \'e diagonaliz\'avel?
			\vspace{0.5cm}
			\item Calcule $A^{10}$.
		\end{enumerate}
\end{enumerate}

\begin{flushright}
	\vspace{1cm}
	\textit{Boa Prova!}
\end{flushright}

\end{document}
