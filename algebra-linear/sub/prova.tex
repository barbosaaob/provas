\documentclass{article}

\usepackage[inline]{enumitem}
\usepackage{graphicx}
\usepackage{gensymb}
\usepackage{amssymb}
\usepackage[brazil]{babel}

\begin{document}
\noindent{}\rule{\textwidth}{0.4pt}
\begin{center}
	\'{A}lgebra Linear\\
	Avalia\c{c}\~ao PS --- 12/05/2016 \\
	Engenharia Mec\^anica \\
	\vspace{0.2cm}
	% Prof. Adriano Barbosa
\end{center}
Nome: \\
Avalia\c{c}\~ao respondida: \\
\noindent{}\rule{\textwidth}{0.4pt}

\begin{center}
VOC\^E DEVE RESPONDER APENAS AS QUEST\~OES REFERENTES A SUA MENOR NOTA OU APENAS AS QUEST\~OES MARCADAS COM (*)!\@
\end{center}

\noindent{}\rule{\textwidth}{0.4pt}

{\bf Avalia\c{c}\~ao P1:}
\begin{enumerate}
%%%%%%%%%%%%%%%%%%%%%%%%%%%%%%%%%%%%%%%%%%%%%
\item Defina e d\^e um exemplo de:
	\begin{enumerate}
		\item Matriz sim\'etrica
		\item Matriz transposta
	\end{enumerate}

%%%%%%%%%%%%%%%%%%%%%%%%%%%%%%%%%%%%%%%%%%%%%
\item Determine $k$, para que o sistema admita \'unica solu\c{c}\~ao

$\left\{\begin{array}{ccccc}
	-4x & + & 3y & = & 2 \\
	5x & - & 4y & = & 0 \\
	2x & - & y & = & k
\end{array}\right.$

%%%%%%%%%%%%%%%%%%%%%%%%%%%%%%%%%%%%%%%%%%%%%
\item Verifique se as retas $r_1$ e $r_2$ possuem interse\c{c}\~ao:

$r_1:\left\{\begin{array}{l}
	y = 2x - 3 \\
	z = x
\end{array}\right.$
\ \ \ \ \ e\ \ \ \ \ 
$r_2:\left\{\begin{array}{l}
	x = -t \\
	y = 4 - t \\
	z = 2 + 2t
\end{array}\right.$

%%%%%%%%%%%%%%%%%%%%%%%%%%%%%%%%%%%%%%%%%%%%%
\item (*) Encontre a equa\c{c}\~ao param\'etrica da reta que passa pelo ponto $A = (3, 2, 1)$ e \'e simultaneamente perpendicular as retas $r_1$ e $r_2$:

$r_1:\left\{\begin{array}{l}
	x = 3 \\
	y = 1
\end{array}\right.$
\ \ \ \ \ e\ \ \ \ \ 
$r_2:\left\{\begin{array}{l}
	y = x - 3 \\
	z = 2x + 3
\end{array}\right.$

%%%%%%%%%%%%%%%%%%%%%%%%%%%%%%%%%%%%%%%%%%%%%
\item (*) Encontre a equa\c{c}\~ao geral do plano que passa pelos pontos $(1, 0, 0)$, $(0, 2, 0)$ e $(0, 0, 3)$.

\end{enumerate}
\noindent{}\rule{\textwidth}{0.4pt}

{\bf Avalia\c{c}\~ao P2:}
\begin{enumerate}
%%%%%%%%%%%%%%%%%%%%%%%%%%%%%%%%%%%%%%%%%%%%%
\item O conjunto
	$W = \left\{
		\left[\begin{array}{cc}
			a & b \\
			c & d
		\end{array}\right] \mbox{ com }
	a, b, c, d \in \mathbb{R} \mbox{ e } b = c\right\}$
\'e um subespa\c{c}o das matrizes $2\times 2$?

%%%%%%%%%%%%%%%%%%%%%%%%%%%%%%%%%%%%%%%%%%%%%
\item Decida se os conjuntos de vetores s\~ao LI ou LD:
	\begin{enumerate}
		\item $\{(1, 2, 4), (5, -10, -20)\}$ em $\mathbb{R}^3$.
		\item $\{-x^2+6x, x^2+x+1\}$ em $\mathcal{P}_2$.
		\item $\left\{
			\left[\begin{array}{cc}
				-3 & 4 \\
				2 & 0
			\end{array}\right], 
			\left[\begin{array}{cc}
				3 & -4 \\
				-2 & 0
			\end{array}\right]\right\}$
		em $M(2, 2)$.
	\end{enumerate}

%%%%%%%%%%%%%%%%%%%%%%%%%%%%%%%%%%%%%%%%%%%%%
\item (*) Decida se o conjunto $\{1+x+x^2, x+x^2, 1+x^2\}$ \'e uma base de $\mathcal{P}_2$.

%%%%%%%%%%%%%%%%%%%%%%%%%%%%%%%%%%%%%%%%%%%%%
\item Combine as matrizes de rota\c{c}\~ao no sentido anti-hor\'ario de $60\degree$ e $-45\degree$ para
	calcular a matrizes de rota\c{c}\~ao no sentido anti-hor\'ario de $15\degree$.

%%%%%%%%%%%%%%%%%%%%%%%%%%%%%%%%%%%%%%%%%%%%%
\item (*) Determine se a transforma\c{c}\~ao linear associada a matriz
	$A=\left[\begin{array}{cc}
			1 & 1 \\
			-2 & 0 \\
			-3 & 4
		\end{array}\right]$
	dada nas bases can\^onicas \'e injetiva.
\end{enumerate}

\noindent{}\rule{\textwidth}{0.4pt}

{\bf Avalia\c{c}\~ao P3:}
\begin{enumerate}
%%%%%%%%%%%%%%%%%%%%%%%%%%%%%%%%%%%%%%%%%%%%%
\item Encontre o polin\^omio caracter\'istico das matrizes:
	\begin{enumerate}
		\item $\left[\begin{array}{cc}
					1 & 2 \\
					0 & -1
			\end{array}\right]$
		\item $\left[\begin{array}{cc}
					-2 & 4 \\
					-9 & -10
			\end{array}\right]$
	\end{enumerate}

%%%%%%%%%%%%%%%%%%%%%%%%%%%%%%%%%%%%%%%%%%%%%
\item Calcule os autovalores e autovetores das matrizes da quest\~ao acima.

%%%%%%%%%%%%%%%%%%%%%%%%%%%%%%%%%%%%%%%%%%%%%
\item Dada a matriz
	$A=\left[
		\begin{array}{cc}
			0 & 2 \\
			1 & 0
		\end{array}
	\right]$
	calcule os autovalores de $AA^T$ e $A^TA$.

%%%%%%%%%%%%%%%%%%%%%%%%%%%%%%%%%%%%%%%%%%%%%
\item Se o polin\^omio caracter\'istico da matriz $A$ \'e $p(\lambda) = (-1-\lambda)(-2-\lambda)^2(3-\lambda)^2$, responda:
	\begin{enumerate}
		\item Quais as dimens\~oes da matriz $A$?
		\item Quantos autovalores distintos a matriz $A$ possui?
		\item Quais s\~ao os autovalores de $A$?
	\end{enumerate}

%%%%%%%%%%%%%%%%%%%%%%%%%%%%%%%%%%%%%%%%%%%%%
\item (*) Dada a matriz
	$A=\left[\begin{array}{ccc}
			1 & 2 & 3 \\
			0 & 2 & -1 \\
			0 & 0 & 3
		\end{array}\right]$
	\begin{enumerate}
		\item Exiba uma base de $\mathbb{R}^3$ formada por autoveroes de $A$.
		\item Diagonalize a matriz $A$.
	\end{enumerate}
\end{enumerate}

\end{document}
