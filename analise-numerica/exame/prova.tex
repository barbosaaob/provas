\documentclass[a4paper,5pt]{amsbook}
%%%%%%%%%%%%%%%%%%%%%%%%%%%%%%%%%%%%%%%%%%%%%%%%%%%%%%%%%%%%%%%%%%%%%

\usepackage{booktabs}
\usepackage{graphicx}
% \usepackage[]{float}
\usepackage{amssymb}
% \usepackage{amsfonts}
% \usepackage[]{amsmath}
% \usepackage[]{epsfig}
% \usepackage[brazil]{babel}
% \usepackage[utf8]{inputenc}
% \usepackage{verbatim}
%\usepackage[]{pstricks}
%\usepackage[notcite,notref]{showkeys}
\usepackage{subcaption}

%%%%%%%%%%%%%%%%%%%%%%%%%%%%%%%%%%%%%%%%%%%%%%%%%%%%%%%%%%%%%%

\newcommand{\sen}{\text{sen}}
\newcommand{\ds}{\displaystyle}

%%%%%%%%%%%%%%%%%%%%%%%%%%%%%%%%%%%%%%%%%%%%%%%%%%%%%%%%%%%%%%%%%%%%%%%%

\setlength{\textwidth}{16cm} %\setlength{\topmargin}{-0.1cm}
\setlength{\leftmargin}{1.2cm} \setlength{\rightmargin}{1.2cm}
\setlength{\oddsidemargin}{0cm}\setlength{\evensidemargin}{0cm}

%%%%%%%%%%%%%%%%%%%%%%%%%%%%%%%%%%%%%%%%%%%%%%%%%%%%%%%%%%%%%%%%%%%%%%%%

% \renewcommand{\baselinestretch}{1.6}
% \renewcommand{\thefootnote}{\fnsymbol{footnote}}
% \renewcommand{\theequation}{\thesection.\arabic{equation}}
% \setlength{\voffset}{-50pt}
% \numberwithin{equation}{chapter}

%%%%%%%%%%%%%%%%%%%%%%%%%%%%%%%%%%%%%%%%%%%%%%%%%%%%%%%%%%%%%%%%%%%%%%%

\begin{document}
\thispagestyle{empty}
\hspace{-0.6cm}
\begin{minipage}[p]{0.14\linewidth}
	\includegraphics[scale=0.24]{ufgd.png}
\end{minipage}
\begin{minipage}[p]{0.7\linewidth}
\begin{tabular}{c}
\toprule{}
{{\bf UNIVERSIDADE FEDERAL DA GRANDE DOURADOS}}\\
{{\bf Prof.\ Adriano Barbosa}}\\

{{\bf An\'alise Num\'erica --- Exame}}\\

\midrule{}
Eng.\ Mec\^anica\hspace{5cm}13 de Abril de 2017 \\
\bottomrule{}
\end{tabular}
\vspace{-0.45cm}
%
\end{minipage}
\begin{minipage}[p]{0.15\linewidth}
\begin{flushright}
\def\arraystretch{1.2}
\begin{tabular}{|c|c|}  % chktex 44
\hline\hline  % chktex 44
1 & \hspace{1.2cm} \\
\hline  % chktex 44
2(a)& \\
\hline  % chktex 44
2(b)& \\
\hline  % chktex 44
3(a)&  \\
\hline  % chktex 44
3(b)&  \\
\hline  % chktex 44
{\small Total}&  \\
\hline\hline  % chktex 44
\end{tabular}
\end{flushright}
\end{minipage}

%------------------------
\vspace{0.5cm}
{\bf Aluno(a):}\dotfill{} % chktex 36
%----------------------------

\vspace{1cm}
%%%%%%%%%%%%%%%%%%%%%%%%%%%%%%%%   formulario  inicio  %%%%%%%%%%%%%%%%%%%%%%%%%%%%%%%%
\begin{enumerate}
	\vspace{0.5cm}
	\item Dada $\ds f(x) = \frac{\sen{(5x)}}{x}$:
		\begin{enumerate}
			\vspace{0.3cm}
			\item \'{E} poss\'{\i}vel garantir um zero de $f(x)$ no intervalo $[1,2]$? E no intervalo $[2,3]$? Justifique.
			\vspace{0.3cm}
			\item Calcule tr\^es itera\c{c}\~oes do m\'etodo da bisse\c{c}\~ao para $f(x)$ nos intervalos acima, se poss\'{\i}vel.
			\vspace{0.3cm}
			\item O m\'etodo de Newton com $p_0=1$ converge para o valor $p
				\approx 1.256637$. Verifique que $p$ \'e a aproxima\c{c}\~ao de um zero
				de $f(x)$ no intervalo $[1,2]$. Por que isso n\~ao invalida sua
				resposta no item (a)?
		\end{enumerate}

	\vspace{0.5cm}
	\item Dado o problema de valor inicial $y'=y$, $y(0)=2$, $x\in[0,1]$.
		\begin{enumerate}
			\vspace{0.3cm}
			\item Resolva o problema acima usando o m\'etodo do ponto m\'edio e $h=0.2$.
			\vspace{0.3cm}
			\item Use interpola\c{c}\~ao de Lagrange de grau 2 para calcular $y(0.1)$.
		\end{enumerate}

	\vspace{0.5cm}
	\item Seja
		\[A = \left[\begin{array}{ccc}
					10 & -1 & 0 \\
					-1 & 10 & -2 \\
					0 & -2 & 10
				\end{array}\right]\]
		\begin{enumerate}
			\vspace{0.3cm}
			\item Use o m\'etodo de Gauss-Seidel para resolver o sistema $Ax = b$, com $b=(-1, 10, -2)$, $x^{(0)}= (0, 0, 0)$ e precis\~ao de $10^{-3}$.
			\vspace{0.3cm}
			\item Compare o resultado do item (a) com a solu\c{c}\~ao exata do sistema.
		\end{enumerate}
\end{enumerate}

\begin{flushright}
	\vspace{1cm}
	\textit{Boa Prova!}
\end{flushright}

\end{document}
