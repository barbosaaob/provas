\documentclass[a4paper,5pt]{amsbook}
%%%%%%%%%%%%%%%%%%%%%%%%%%%%%%%%%%%%%%%%%%%%%%%%%%%%%%%%%%%%%%%%%%%%%

\usepackage{booktabs}
\usepackage{graphicx}
% \usepackage[]{float}
\usepackage{amssymb}
% \usepackage{amsfonts}
% \usepackage[]{amsmath}
% \usepackage[]{epsfig}
% \usepackage[brazil]{babel}
% \usepackage[utf8]{inputenc}
% \usepackage{verbatim}
%\usepackage[]{pstricks}
%\usepackage[notcite,notref]{showkeys}
\usepackage{subcaption}

%%%%%%%%%%%%%%%%%%%%%%%%%%%%%%%%%%%%%%%%%%%%%%%%%%%%%%%%%%%%%%

\newcommand{\sen}{\text{sen}}
\newcommand{\ds}{\displaystyle}

%%%%%%%%%%%%%%%%%%%%%%%%%%%%%%%%%%%%%%%%%%%%%%%%%%%%%%%%%%%%%%%%%%%%%%%%

\setlength{\textwidth}{16cm} %\setlength{\topmargin}{-0.1cm}
\setlength{\leftmargin}{1.2cm} \setlength{\rightmargin}{1.2cm}
\setlength{\oddsidemargin}{0cm}\setlength{\evensidemargin}{0cm}

%%%%%%%%%%%%%%%%%%%%%%%%%%%%%%%%%%%%%%%%%%%%%%%%%%%%%%%%%%%%%%%%%%%%%%%%

% \renewcommand{\baselinestretch}{1.6}
% \renewcommand{\thefootnote}{\fnsymbol{footnote}}
% \renewcommand{\theequation}{\thesection.\arabic{equation}}
% \setlength{\voffset}{-50pt}
% \numberwithin{equation}{chapter}

%%%%%%%%%%%%%%%%%%%%%%%%%%%%%%%%%%%%%%%%%%%%%%%%%%%%%%%%%%%%%%%%%%%%%%%

\begin{document}
\thispagestyle{empty}
\hspace{-0.6cm}
\begin{minipage}[p]{0.14\linewidth}
	\includegraphics[scale=0.24]{ufgd.png}
\end{minipage}
\begin{minipage}[p]{0.7\linewidth}
\begin{tabular}{c}
\toprule{}
{{\bf UNIVERSIDADE FEDERAL DA GRANDE DOURADOS}}\\
{{\bf Prof.\ Adriano Barbosa}}\\

{{\bf An\'alise Num\'erica --- Avalia\c{c}\~ao P2}}\\

\midrule{}
Eng.\ Mec\^anica\hspace{5cm}27 de Mar\c{c}o de 2017 \\
\bottomrule{}
\end{tabular}
\vspace{-0.45cm}
%
\end{minipage}
\begin{minipage}[p]{0.15\linewidth}
\begin{flushright}
\def\arraystretch{1.2}
\begin{tabular}{|c|c|}  % chktex 44
\hline\hline  % chktex 44
1 & \hspace{1.2cm} \\
\hline  % chktex 44
2& \\
\hline  % chktex 44
3& \\
\hline  % chktex 44
% 4&  \\
% \hline  % chktex 44
% 5&  \\
% \hline  % chktex 44
{\small Total}&  \\
\hline\hline  % chktex 44
\end{tabular}
\end{flushright}
\end{minipage}

%------------------------
\vspace{0.5cm}
{\bf Aluno(a):}\dotfill{}  % chktex 36
%----------------------------

\vspace{0.2cm}
%%%%%%%%%%%%%%%%%%%%%%%%%%%%%%%%   formulario  inicio  %%%%%%%%%%%%%%%%%%%%%%%%%%%%%%%%
\vspace{0.5cm}
Dada a matriz
\[A = \left[
		\begin{array}{rrr}
			4 & 1 & -1 \\
			-1 & 3 & 1 \\
			2 & 2 & 5
		\end{array}\right]\]
\begin{enumerate}
\vspace{0.5cm}

	\item Sem fazer permuta\c{c}\~ao de linhas:
			\vspace{0.2cm}
		\begin{enumerate}
			\item Calcule a decomposi\c{c}\~ao $A = LU$.
			\vspace{0.2cm}
			\item Encontre a solu\c{c}\~ao exata do sistema $Ax=b$ com $b = (4, 3, 9)$.
			\vspace{0.2cm}
			\item Encontre a solu\c{c}\~ao exata do sistema $Ax=b$ com $b = (3, 0, 7)$.
		\end{enumerate}
	\vspace{0.5cm}

	\item 
		\begin{enumerate}
			\item Calcule tr\^es itera\c{c}\~oes do m\'etodo de Gauss-Seidel para o
				sistema $Ax = b$ usando $x^{(0)}~=~(0,0,0)$ e $b=(4,3,9)$. (Use
				5 casas decimais)
			\vspace{0.2cm}
			\item Compare o resultado das tr\^es itera\c{c}\~oes com o resultado obtido no item (b) da quest\~ao anterior.
		\end{enumerate}
	\vspace{0.5cm}

	\item Dado o PVI $y' = 1 + t + y$, $y(0) = 1$ e $t\in{}[0, 1.5]$:
		\begin{enumerate}
			\vspace{0.2cm}
			\item Mostre que o m\'etodo de Euler Modificado com $h=0.25$ aplicado a EDO acima pode ser simplificado usando a recorr\^encia:
				\[w_{i+1} = 1.28125 w_i + 0.0703125 i + 0.3125\]
			\vspace{0.2cm}
			\item Resolva o PVI utilizando o m\'etodo de Euler Modificado com $h=0.25$.
		\end{enumerate}
	\vspace{0.5cm}
\end{enumerate}

\begin{flushright}
	\vspace{1cm}
	\textit{Boa Prova!}
\end{flushright}

\end{document}
