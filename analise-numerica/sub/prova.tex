\documentclass[a4paper,5pt]{amsbook}
%%%%%%%%%%%%%%%%%%%%%%%%%%%%%%%%%%%%%%%%%%%%%%%%%%%%%%%%%%%%%%%%%%%%%

\usepackage{booktabs}
\usepackage{graphicx}
% \usepackage[]{float}
\usepackage{amssymb}
% \usepackage{amsfonts}
% \usepackage[]{amsmath}
% \usepackage[]{epsfig}
% \usepackage[brazil]{babel}
% \usepackage[utf8]{inputenc}
% \usepackage{verbatim}
%\usepackage[]{pstricks}
%\usepackage[notcite,notref]{showkeys}
\usepackage{subcaption}

%%%%%%%%%%%%%%%%%%%%%%%%%%%%%%%%%%%%%%%%%%%%%%%%%%%%%%%%%%%%%%

\newcommand{\sen}{\text{sen}}
\newcommand{\ds}{\displaystyle}

%%%%%%%%%%%%%%%%%%%%%%%%%%%%%%%%%%%%%%%%%%%%%%%%%%%%%%%%%%%%%%%%%%%%%%%%

\setlength{\textwidth}{16cm} %\setlength{\topmargin}{-0.1cm}
\setlength{\leftmargin}{1.2cm} \setlength{\rightmargin}{1.2cm}
\setlength{\oddsidemargin}{0cm}\setlength{\evensidemargin}{0cm}

%%%%%%%%%%%%%%%%%%%%%%%%%%%%%%%%%%%%%%%%%%%%%%%%%%%%%%%%%%%%%%%%%%%%%%%%

% \renewcommand{\baselinestretch}{1.6}
% \renewcommand{\thefootnote}{\fnsymbol{footnote}}
% \renewcommand{\theequation}{\thesection.\arabic{equation}}
% \setlength{\voffset}{-50pt}
% \numberwithin{equation}{chapter}

%%%%%%%%%%%%%%%%%%%%%%%%%%%%%%%%%%%%%%%%%%%%%%%%%%%%%%%%%%%%%%%%%%%%%%%

\begin{document}
\thispagestyle{empty}
\hspace{-0.6cm}
\begin{minipage}[p]{0.14\linewidth}
	\includegraphics[scale=0.24]{ufgd.png}
\end{minipage}
\begin{minipage}[p]{0.7\linewidth}
\begin{tabular}{c}
\toprule{}
{{\bf UNIVERSIDADE FEDERAL DA GRANDE DOURADOS}}\\
{{\bf Prof.\ Adriano Barbosa}}\\

{{\bf An\'alise Num\'erica --- Avalia\c{c}\~ao PS}}\\

\midrule{}
Eng.\ Mec\^anica\hspace{5cm}3 de Abril de 2017 \\
\bottomrule{}
\end{tabular}
\vspace{-0.45cm}
%
\end{minipage}
\begin{minipage}[p]{0.15\linewidth}
\begin{flushright}
\def\arraystretch{1.2}
\begin{tabular}{|c|c|}  % chktex 44
\hline\hline  % chktex 44
1 & \hspace{1.2cm} \\
\hline  % chktex 44
2& \\
\hline  % chktex 44
3& \\
\hline  % chktex 44
4&  \\
\hline  % chktex 44
5&  \\
\hline  % chktex 44
{\small Total}&  \\
\hline\hline  % chktex 44
\end{tabular}
\end{flushright}
\end{minipage}

%------------------------
\vspace{0.5cm}
{\bf Aluno(a):}\dotfill{} \textbf{Avalia\c{c}\~ao:} \ldots{}\ldots{}\ldots{} % chktex 36
%----------------------------

\vspace{1cm}
%%%%%%%%%%%%%%%%%%%%%%%%%%%%%%%%   formulario  inicio  %%%%%%%%%%%%%%%%%%%%%%%%%%%%%%%%
\textbf{Avalia\c{c}\~ao P1:}
\begin{enumerate}
	\item Calcule quatro itera\c{c}\~oes do m\'etodo da bisse\c{c}\~ao para aproximar
		$\sqrt[3]{25}$ no intervalo $[0, 3]$.
	\item Calcule quatro itera\c{c}\~oes do m\'etodo de Newton para aproximar
		$\sqrt[3]{25}$ com $p_0=1.5$.
	\item A tabela abaixo apresenta a popula\c{c}\~ao dos Estados Unidos entre os
		anos $1950$ e $2000$.
		\begin{table}[h]
			\begin{tabular}{c|c|c|c|c|c|c}  % chktex 44
				Ano & 1950 & 1960 & 1970 & 1980 & 1990 & 2000 \\
				\hline{}  % chktex 44
				Popula\c{c}\~ao (mil) & 151326 & 179323 & 203302 &
				226542 & 249633 & 281422
			\end{tabular}
		\end{table}

		\noindent{}Use o polin\^omio interpolador de Lagrange de grau 2 para aproximar os
		valores da popula\c{c}\~ao no ano de $1940$.
	\item Uma spline c\'ubica com fronteira amarrada para uma fun\c{c}\~ao $f$ \'e
		definida em $[1,3]$ por:
		\[s(x) = \left\{\begin{array}{l}
					3(x-1) + 2(x-1)^2 - (x-1)^3,\ \text{ se } 1\le x\le 2 \\
					a + b(x-2) + c(x-2)^2 + d(x-2)^3, \text{ se } 2\le x\le 3
				\end{array}\right.\]
		Dado que $f'(1)=f'(3)$, encontre $a$, $b$, $c$ e $d$.
	\item Use o m\'etodo de Horner para avaliar a fun\c{c}\~ao $f(x) = x^5 - x^4 +
		2x^3 - 3x^2 + x - 4$ em $\pi$ utilizando aritm\'etica computacional de 3
		d\'{\i}tigos e arredondamento.
\end{enumerate}

\vspace{1cm}
\textbf{Avalia\c{c}\~ao P2:}
\begin{enumerate}
	\item Seja $A = \left[\begin{array}{ccc}
				1 & -1 & 3 \\
				3 & -3 & 1 \\
				1 & 1 & 0
			\end{array}	\right]$:
		\begin{enumerate}
			\item Calcule a decomposi\c{c}\~ao $A = LU$.
			\item Encontre a solu\c{c}\~ao exata do sistema $Ax=b$, com $b = (8, 0, 3)$.
		\end{enumerate}
	\item Seja $A = \left[\begin{array}{ccc}
				10 &  -1 &    0 \\
				-1 &  10 &  -2 \\
				0 &  -2 &  10
			\end{array}	\right]$. Usando 5 casas decimais, calcule quatro
		itera\c{c}\~oes dos m\'etodos abaixo para resolver o sistema $Ax = b$ usando
		$x^{(0)}~=~(0,0,0)$ e $b=(9, 7, 8)$.
		\begin{enumerate}
			\item Jacobi.
			\item Gauss-Seidel.
			\item Compare os resultados com a solu\c{c}\~ao exata $(1, 1, 1)$. Qual
				m\'etodo obt\'em melhor resultado?
		\end{enumerate}
	\item Resolva o PVI $y' = 1 + t - y$, $y(0) = 1$ e $t\in{}[0, 1]$ usando o
		m\'etodo de Euler Modificado com $h=0.2$.
\end{enumerate}

\begin{flushright}
	\vspace{1cm}
	\textit{Boa Prova!}
\end{flushright}

\end{document}
