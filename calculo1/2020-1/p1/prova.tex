\documentclass{prova}

\usepackage{amsmath}
\usepackage{amsfonts}

\setlength{\textheight}{25cm}

\renewcommand{\sin}{\,\mbox{sen}\,}
\newcommand{\ds}{\displaystyle}

\professor{Prof.\@ Adriano Barbosa}
\disciplina{C\'alculo Diferencial e Integral}
\avaliacao{P1}
\curso{Engenharia de Produção}
\data{13/04/2021}

\begin{document}
	\cabecalho{5}  % o numero 5 indica a qnt de quadros na tabela de nota

    \textbf{Todas as respostas devem ser justificadas.}

    \begin{questionario}
        \q{Esboce o gráfico de uma função que atenda as condições abaixo:}
            \begin{questionario}
                \qq{}
                    \begin{itemize}
                        \item[] O domínio de $f$ é $[-2, 1]$;
                        \item[] $f(-2) = f(0) = f(1) = 0$;
                        \item[] $\ds\lim_{x\rightarrow -2^+} f(x) = 2$,
                                $\ds\lim_{x\rightarrow 0} f(x) = 0$ e
                                $\ds\lim_{x\rightarrow 1^-} f(x) = 1$.
                    \end{itemize}
                \qq{}
                    \begin{itemize}
                        \item[] $f(-1) = 0$, $f(0) = 1$, $f(1) = 0$;
                        \item[] $\ds\lim_{x\rightarrow -1^-} f(x) = 0$ e
                                $\ds\lim_{x\rightarrow -1^+} f(x) = +\infty$;
                        \item[] $\ds\lim_{x\rightarrow 1^-} f(x) = 1$ e
                                $\ds\lim_{x\rightarrow 1^+} f(x) = -\infty$;
                    \end{itemize}
            \end{questionario}
        \q{Determine o valor de $k$ de modo que $f$ seja contínua para todo
           $x\in\mathbb{R}$}
            \[f(x) = \left\{
                \begin{array}{ll}
                    7x-2, & x \le 1 \\
                    kx^2, & x > 1
                \end{array}
            \right.\]
        \q{Sem transformar ou manipular a função $f$:}
            \begin{questionario}
                \qq{$\ds f(x) = x^{-3} + \frac{1}{x^7}$, calcule $f'(x)$.}
                \qq{$f(x) = x^2(x^4-2)$, calcule $f''(x)$.}
            \end{questionario}
        \q{Encontre a equação da reta tangente a $y=x \cos{x}$ em $x=\pi$.}
        \q{Seja $s(t) = 4,5\, t^2$ a equação que descreve a posição (m) de uma
           partícula em função do tempo (s) no intervalo $0\le t\le 10$:}
            \begin{questionario}
                \qq{Calcule a velocidade média da partícula no intervalo $[0,
                    10]$.}
                \qq{Determine a velocidade instantânea da partícula em $t=4$.}
            \end{questionario}
    \end{questionario}
\end{document}
