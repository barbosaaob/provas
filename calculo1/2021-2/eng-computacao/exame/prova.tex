\documentclass{prova}

\usepackage{amsmath}
\usepackage{amsfonts}

\setlength{\textheight}{25cm}

\DeclareMathOperator{\sen}{sen}
\DeclareMathOperator{\tg}{tg}
\newcommand{\ds}{\displaystyle}
\newcommand{\ra}{\rightarrow}

\professor{Prof.\@ Adriano Barbosa}
\disciplina{C\'alculo Diferencial e Integral}
\avaliacao{Final}
\curso{Engenharia de Computa\c{c}\~ao}
\data{20/06/2022}

\begin{document}
	\cabecalho{5}  % o numero 5 indica a qnt de quadros na tabela de nota

    \textbf{Todas as respostas devem ser justificadas.}

    \begin{questionario}
        \q{O volume de um cubo est\'a crescendo a uma taxa de 10 cm$^3$/min.
           Qu\~ao r\'apido a \'area da surperf\'{i}cie do cubo est\'a aumentando
           quando sua aresta mede 30 cm?}
        \q{Um time de futebol joga num est\'adio com capacidade para 15.000
           espectadores. Com o ingresso custando R\$12,00, a m\'edia de p\'ublico
           nos jogos \'e de 11.000 pessoas. Uma pesquisa de mercado indicou que o
           p\'ublico aumentaria em 1.000 pessoas em cada jogo para cada R\$1,00
           diminuido no valor do ingresso. Qual deve ser o pre\c{c}o do ingresso
           para que o faturamento do time com a venda de ingressos seja a maior
           poss\'ivel?}
        \q{Dada $f(t) = \ds 1+\frac{1}{2}t^4 - \frac{2}{5} t^9$:}
            \begin{questionario}
                \qq{Calcule a derivada de $g(x) = \ds\int_0^{-x} f(t)\ dt$.}
                \qq{Encontre uma antiderivada de $f$.}
                \qq{Calcule $\ds\int_0^1 f(t)\ dt$.}
            \end{questionario}
        \q{\'E poss\'{\i}vel encontrar uma fun\c{c}\~ao tal que $f'(0)=1$,
           $f'(1)=0$ e que $f'(x)>0$ para todo $x\in\mathbb{R}$? Exiba a
           fun\c{c}\~ao ou prove que n\~ao existe.}
        \q{}
        \begin{questionario}
            \qq{Calcule $\ds\lim_{x\ra3} \left(\frac{x}{x-3} \int_3^x
                \frac{\sen{t}}{t}\ dt\right)$.}
            \qq{Sejam $f(x) = \ds\int_0^{\sen{x}} 1+\cos{\left(t^2\right)}\ dt$ e
                $g(x) = \ds\int_0^{f(x)} \frac{x^2}{\sqrt{1+t^3}}\ dt$. Calcule
                $g'(\pi)$.}
        \end{questionario}
    \end{questionario}
\end{document}
