\documentclass{prova}

\usepackage{amsmath}
\usepackage{amsfonts}

\setlength{\textheight}{25cm}

\DeclareMathOperator{\sen}{sen}
\DeclareMathOperator{\tg}{tg}
\newcommand{\ds}{\displaystyle}

\professor{Prof.\@ Adriano Barbosa}
\disciplina{C\'alculo Diferencial e Integral}
\avaliacao{P1}
\curso{Engenharia de Computa\c{c}\~ao}
\data{06/04/2022}

\begin{document}
	\cabecalho{5}  % o numero 5 indica a qnt de quadros na tabela de nota

    \textbf{Todas as respostas devem ser justificadas.}

    \begin{questionario}
        \q{Calcule as derivadas abaixo:}
            \begin{questionario}
                \qq{$f'(x)$, onde $f(x)=10^{\sqrt{x}}$.}
                \qq{$f''(x)$, onde $f(x)=\ln(x^2)$.}
            \end{questionario}

        \q{Encontre a equa\c{c}\~ao da reta tangente ao gr\'afico de
        $f(x)=\ds\frac{1}{x^2+2x+1}$ no ponto $(0,1)$.}

        \q{Seja $s(t)=t^3-3t^2+2t-1$ a fun\c{c}\~ao que descreve o deslocamento de
        uma part\'{\i}cula em fun\c{c}\~ao do tempo.}
            \begin{questionario}
                \qq{Determine a velocidade instant\^anea da part\'{\i}cula em $t=1$.}
                \qq{Determine a acelera\c{c}\~ao da part\'{\i}cula em fun\c{c}\~ao do tempo.}
                \qq{Determine o intervalo onde a acelera\c{c}\~ao \'e positiva.}
            \end{questionario}

        \q{Para quais valores de $x$ no intervalo $[0,\pi]$ a tangente ao
        gr\'afico de $f(x)=\sen(x)\cos(x)$ \'e horizontal?}
        [Use a identidade $\sen^2{x}+\cos^2{x}=1$ se achar necess\'ario.]
        
        \q{Seja $f(x)=g(x+g(x))$.}
            \begin{questionario}
                \qq{Calcule $f'$ em fun\c{c}\~ao de $g$ e $g'$.}
                \qq{Se $g(0)=g'(0)=0$, calcule $f'(0)$.}
            \end{questionario}
    \end{questionario}
\end{document}
