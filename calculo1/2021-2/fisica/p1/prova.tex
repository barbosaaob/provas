\documentclass{prova}

\usepackage{amsmath}
\usepackage{amsfonts}

\setlength{\textheight}{25cm}

\DeclareMathOperator{\sen}{sen}
\DeclareMathOperator{\tg}{tg}
\newcommand{\ds}{\displaystyle}

\professor{Prof.\@ Adriano Barbosa}
\disciplina{C\'alculo Diferencial e Integral}
\avaliacao{P1}
\curso{F\'{\i}sica}
\data{06/04/2022}

\begin{document}
	\cabecalho{5}  % o numero 5 indica a qnt de quadros na tabela de nota

    \textbf{Todas as respostas devem ser justificadas.}

    \begin{questionario}
        \q{Calcule as derivadas abaixo:}
            \begin{questionario}
                \qq{$f'(x)$, onde $f(x)=\ds e^{x^3+1}$.}
                \qq{$f''(x)$, onde $f(x)=x\ln{x}-x$.}
            \end{questionario}

        \q{Encontre a equa\c{c}\~ao da reta tangente ao gr\'afico de
        $f(x)=\ds\frac{\cos{x}+1}{\sen{x}}$ no ponto
        $\left(\ds\frac{\pi}{2},1\right)$.}

        \q{Seja $s(t)=t^2-2t++1$ a fun\c{c}\~ao que descreve o deslocamento de
        uma part\'{\i}cula em fun\c{c}\~ao do tempo.}
            \begin{questionario}
                \qq{Determine a velocidade instant\^anea da part\'{\i}cula em $t=2$.}
                \qq{Determine o intervalo onde a velocidade \'e positiva.}
                \qq{Determine a acelara\c{c}\~ao da part\'{\i}cula em fun\c{c}\~ao do tempo.}
            \end{questionario}

        \q{Para quais valores de $x$ a tangente ao gr\'afico de $f(x)=x^2\ln{x}$
        \'e horizontal?}
        
        \q{Sejam $f(x)=[g(x^2)]^3$ e $g(4)=g'(4)=1$, calcule $f'(2)$.}
    \end{questionario}
\end{document}
