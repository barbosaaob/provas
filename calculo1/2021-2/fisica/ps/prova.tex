\documentclass{prova}

\usepackage{amsmath}
\usepackage{amsfonts}

\setlength{\textheight}{25cm}

\DeclareMathOperator{\sen}{sen}
\DeclareMathOperator{\tg}{tg}
\newcommand{\ds}{\displaystyle}

\professor{Prof.\@ Adriano Barbosa}
\disciplina{C\'alculo Diferencial e Integral}
\avaliacao{PS}
\curso{F\'{\i}sica}
\data{13/06/2022}

\begin{document}
	\cabecalho{5}  % o numero 5 indica a qnt de quadros na tabela de nota

    \textbf{Todas as respostas devem ser justificadas.}

    \vspace{0.5cm}
    \textbf{Avalia\c{c}\~ao P1:}
    \begin{questionario}
        \q{Encontre a equa\c{c}\~ao da reta tangente a
        $f(x)=\displaystyle\frac{x^2-1}{x^2+1}$ no ponto $(0,-1)$.}
        \q{Seja $f(x)=x\sen{x}$. Calcule $f''(x)$.}
        \q{Se $g(x)=f(x)+x\cos(f(x))$ e $f'(0)=f(0)=0$, calcule $g'(0)$.}
        \q{Se $f(x)=e^{2x}$, encontre a f\'ormula para $f^{(n)}(x)$ (derivada de
           ordem $n$) em fun\c{c}\~ao de $n$.}
        \q{Determine os pontos onde a tangente a $f(x)=x-x\ln{x}$ \'e
           horizontal.}
    \end{questionario}

    \vspace{0.5cm}
    \textbf{Avalia\c{c}\~ao P2:}
    \begin{questionario}
        \q{Encontre o erro no c\'alculo abaixo e calcule o limite corretamente.}
            \begin{align*}
                \lim_{x\rightarrow\pi^{-}}\frac{\sen{x}}{1-\cos{x}}
                =\lim_{x\rightarrow\pi^{-}}\frac{\cos{x}}{\sen{x}}
                =-\infty
            \end{align*}
        \q{Dado que $x^2+y^2=2x+4y$, onde $x$ e $y$ s\~ao fun\c{c}\~oes de $t$, calcule
           $\displaystyle\frac{dx}{dt}$ sabendo que
           $\displaystyle\frac{dy}{dt}=6$ quando $(x,y)=(2,3)$.}
        \q{Mostre que entre todos os ret\^angulos de \'area $A$, o quadrado \'e
           o que tem o menor per\'{\i}metro.}
        \q{Calcule a \'area da regi\~ao delimitada pelas curvas $x=1-y^2,
           x=y^2-1$.}
        \q{Utilizando integrais, calcule o volume da pir\^amide de altura $H$ e
           base quadrada de lado $L$.}
    \end{questionario}
\end{document}
