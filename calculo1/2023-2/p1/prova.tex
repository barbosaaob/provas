\documentclass{prova}

\usepackage{amsmath}
\usepackage{amsfonts}

\setlength{\textheight}{25cm}

\DeclareMathOperator{\sen}{sen}
\DeclareMathOperator{\tg}{tg}
\newcommand{\ds}{\displaystyle}

\professor{Prof.\@ Adriano Barbosa}
\disciplina{C\'alculo Diferencial e Integral I}
\avaliacao{P1}
\curso{Qu\'{\i}mica}
\data{21/11/2023}

\begin{document}
	\cabecalho{5}  % o numero 5 indica a qnt de quadros na tabela de nota

    \textbf{Todas as respostas devem ser justificadas.}

    \begin{questionario}
        \q{(2 pts) Calcule os limites abaixo:}
            \begin{questionario}
                \qq{$\ds\lim_{x\rightarrow -3} \frac{x^2-9}{2x^2+7x+3}$}
                \qq{$\ds\lim_{x\rightarrow \pi/4} x\sen^2(x)$}
            \end{questionario}
        \q{(2 pts) O deslocamento de um m\'ovel \'e dado pela fun\c{c}\~ao $s(t) =
           5t-9t^2$.}
            \begin{questionario}
                \qq{Determine a velocidade instant\^anea do m\'ovel em $t=1$.}
                \qq{Determine o intervalo onde a velocidade instant\^anea \'e
                    positiva.}
                \qq{Determine a acelera\c{c}\~ao do m\'ovel em fun\c{c}\~ao do tempo.}
            \end{questionario}
        \q{(2 pts) Encontre a equa\c{c}\~ao da reta tangente ao gr\'afico de $f(x) =
           \sen(2\ln(x))$ no ponto $(1,0)$.}
        \q{(2 pts) Derive as fun\c{c}\~oes abaixo:}
            \begin{questionario}
                \q{$\ds f(x) = \frac{x\sen(x)}{1+x}$}
                \q{$\ds f(x) = x\ln(x) - x$}
            \end{questionario}
        \q{(2 pts) Seja $f(x) = \ds\frac{x^2+4x+3}{\sqrt{x}}$. Mostre que
           $f'(x) = \ds\frac{3x^2+4x-3}{2x^{3/2}}$.}
    \end{questionario}
\end{document}
