\documentclass{prova}

\usepackage{amsmath}
\usepackage{amsfonts}

\setlength{\textheight}{25cm}

\DeclareMathOperator{\sen}{sen}
\DeclareMathOperator{\tg}{tg}
\newcommand{\ds}{\displaystyle}

\professor{Prof.\@ Adriano Barbosa}
\disciplina{C\'alculo Diferencial e Integral I}
\avaliacao{P2}
\curso{Qu\'{\i}mica}
\data{20/02/2024}

\begin{document}
	\cabecalho{5}  % o numero 5 indica a qnt de quadros na tabela de nota

    \textbf{Todas as respostas devem ser justificadas.}

    \begin{questionario}
        \q{(2 pts) Calcule o limite: $\ds\lim_{x\rightarrow 0}
           \frac{\sqrt{1+3x}-\sqrt{1-9x}}{x}$.}
        \q{(2 pts) Dado que $4x^2+9y^2=36$, onde $x$ e $y$ s\~ao fun\c{c}\~oes de $t$,
           calcule $x'(t)$ quando $x=1$, $y=\frac{2}{3}\sqrt{5}$ e
           $y'(t)=\frac{1}{3}$.}
        \q{(2 pts) Uma loja tem vendido 200 aparelhos de som por semana a R\$
           350,00 cada. Uma pesquisa de mercado indicou que para cada R\$ 10,00 de
           desconto oferecido aos compradores, o n\'umero de unidades vendidas
           aumenta em 20 por semana. Determine qual o desconto que a loja deve
           oferecer para maximizar seu faturamento.}
        \q{(2 pts) Esboce o gr\'afico de $f$ tal que $f(0) = 0$, $f'(0) = 0$,
           $f''(0) = 0$, $f'(x) > 0$ para $x<0$ e para $x>0$, $f''(x) < 0$ para
           $x<0$ e $f''(x) > 0$ para $x>0$.}
        \q{}
            \begin{questionario}
                \qq{(1 pt) Determine uma primitiva de $f(x)=\ds\frac{x-1}{\sqrt{x}}$.}
                \qq{(1 pt) Calcule $\ds\int_1^9 \frac{x-1}{\sqrt{x}}\ dx$.}
            \end{questionario}
    \end{questionario}
\end{document}
