\documentclass{prova}

\usepackage{amsmath}
\usepackage{amsfonts}

\setlength{\textheight}{25cm}

\DeclareMathOperator{\sen}{sen}
\DeclareMathOperator{\tg}{tg}
\newcommand{\ds}{\displaystyle}

\professor{Prof.\@ Adriano Barbosa}
\disciplina{C\'alculo Diferencial e Integral I}
\avaliacao{PS}
\curso{Qu\'{\i}mica}
\data{27/02/2024}

\begin{document}
	\cabecalho{5}  % o numero 5 indica a qnt de quadros na tabela de nota

	\textbf{Todas as respostas devem ser justificadas.}

    \vspace{0.5cm}

    {\bf Avalia\c{c}\~ao P1:}
    \begin{questionario}
        \q{Determine o maior dom\'{\i}nio de $f(x)=\ds\frac{\sen{(6x)}}{x}$ e calcule $\ds\lim_{x\rightarrow 0} f(x)$.}
        \q{Mostre que $\ds\lim_{x\rightarrow 0} x^2\cos{\left(\frac{1}{x^2}\right)}=0$.}
        \q{Dados $f(x)=x^3-2x-\cos{x}$ e $I=\left(0,\frac{\pi}{2}\right)$:}
            \begin{questionario}
                \qq{Determine se a fun\c{c}\~ao $f$ \'e cont\'{\i}nua no intervalo $I$.}
                \qq{Mostre que a fun\c{c}\~ao $f$ possui uma raiz no intervalo $I$.}
            \end{questionario}
        \q{Dada a equa\c{c}\~ao impl\'{\i}cita $x^4(x+y)=y^2(3x-y)$, calcule $\ds\frac{dy}{dx}$.}
        \q{Para quais valores de $x$ no intervalo $[0,\pi]$ a tangente ao
           gr\'afico de $f(x)=\sen(x)\cos(x)$ \'e horizontal?}
    \end{questionario}

    \vspace{1cm}

    {\bf Avalia\c{c}\~ao P2:}
    \begin{questionario}
        \q{(2 pts) Calcule o limite $\ds\lim_{x\rightarrow \frac{\pi}{2}^+}
           \frac{\cos{x}}{1-\sen{x}}$.}
        \q{(2 pts) Um tanque cil\'{\i}ndrico com raio 5m est\'a enchendo com \'agua a
           uma taxa de 3m$^3$/min. Qu\~ao r\'apido a altura da \'agua est\'a
           aumentando?}
        \q{(2 pts) Uma sorveteria vende $130$ picol\'es por dia por R\$ $5,00$
           cada.  Observou-se que, durante uma promo\c{c}\~ao de ver\~ao, cada vez que
           diminuia R\$ $0,50$ no pre\c{c}o do picol\'e, vendia $20$ unidades a mais por
           dia. Qual deve ser o pre\c{c}o do picol\'e para que a receita da sorveteria
           seja m\'axima?}
        \q{(2 pts) Uma part\'{\i}cula de move com velocidade $v(t) = \sen(t) -
           \cos(t)$. Determine a posi\c{c}\~ao da part\'{\i}cula em fun\c{c}\~ao do tempo
           sabendo que $s(0)=0$.}
        \q{(2 pts) Calcule a integral definida $\ds\int_0^4
           \frac{4+6u}{\sqrt{u}}\ du$.}
    \end{questionario}
\end{document}
