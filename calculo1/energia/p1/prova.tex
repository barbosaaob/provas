\documentclass{prova}

\usepackage{amssymb}

\newcommand{\sen}{\,\mbox{sen}\,}
\newcommand{\tg}{\,\mbox{tg}\,}
\newcommand{\cosec}{\,\mbox{cosec}\,}
\newcommand{\cotg}{\,\mbox{cotg}\,}
\newcommand{\tr}{\,\mbox{tr}\,}
\newcommand{\ds}{\displaystyle}
\newcommand{\ra}{\rightarrow}

\professor{Prof.\@ Adriano Barbosa}
\disciplina{C\'alculo Diferencial e Integral}
\avaliacao{P1}
\curso{Eng.\@ de Energia}
\data{17/05/2018}

\begin{document}
	\cabecalho{5}  % o numero 5 indica a qnt de quadros na tabela de nota

	\textbf{Todas as respostas devem ser justificadas.}
    \vspace{1cm}

	\begin{questionario}
        \q{Determine o dom\'{\i}nio das fun\c{c}\~oes e calcule os limites abaixo:}
            \begin{questionario}
                \qq{$f(x)=\ds\frac{\sqrt{1+x}-\sqrt{1-x}}{x}$, $\ds\lim_{x\ra 0} f(x)$.}
                \qq{$f(x)=\ds\frac{\sen{(6x)}}{x}$, $\ds\lim_{x\ra 0} f(x)$.}
            \end{questionario}

        \q{Mostre que $\ds\lim_{x\ra 0} x^2\cos{\left(\frac{1}{x^2}\right)}=0$.}

        [Lembre que $-1\le \cos{x}\le 1$, que $x^2\ge0$ e use o Teorema do
        Confronto.]

        \q{Dados $f(x)= x^4-4x^3+2$ e $I=\left(0,1\right)$:}
            \begin{questionario}
                \qq{Determine se a fun\c{c}\~ao $f$ \'e cont\'{\i}nua no intervalo $I$.}
                \qq{Mostre que a fun\c{c}\~ao $f$ possui uma raiz no intervalo $I$.}
            \end{questionario}

        \q{Calcule a derivada das fun\c{c}\~oes abaixo:}
            \begin{questionario}
                \qq{$f(x)=\ds\frac{2x}{5-\cos{x}}$}
                \qq{$g(x)=\ds\ln{\left(x e^x\right)}$}
            \end{questionario}

        \q{Dada a equa\c{c}\~ao impl\'{\i}cita $x^3(x-y)=y^2(x+2y)$, calcule $\ds\frac{dy}{dx}$.}
	\end{questionario}

\end{document}
