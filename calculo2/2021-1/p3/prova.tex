\documentclass{prova}

\usepackage{amsmath}
\usepackage{amsfonts}

\setlength{\textheight}{25cm}

\renewcommand{\sin}{\,\mbox{sen}\,}
\newcommand{\ds}{\displaystyle}

\professor{Prof.\@ Adriano Barbosa}
\disciplina{C\'alculo Diferencial e Integral II}
\avaliacao{P3}
\curso{Engenharia de Produ\c{c}\~ao}
\data{22/11/2021}

\begin{document}
	\cabecalho{5}  % o numero 5 indica a qnt de quadros na tabela de nota

    \textbf{Todas as respostas devem ser justificadas.}

    \begin{questionario}
        \q{Determine se as sequ\^encias abaixo s\~ao convergentes ou divergentes:}
            \begin{questionario}
                \qq{$x_n = \ds\frac{\ln{n}}{n^{0,5}}$}
                \qq{$x_n = \ds\frac{n^{2021}-n^{2020}}{3n^{2022}-1}$}
            \end{questionario}
        \q{Calcule a soma da s\'erie $\ds\sum_{n=0}^{\infty} \frac{1}{(3n+1)(3n+4)}.$}
        \q{Determine se a s\'erie $\ds\sum_{k=1}^{\infty} \frac{k^k}{k!}$ \'e convergente ou divergente.}
        \q{Determine o invervalo de converg\^encia da s\'erie $\ds\sum_{n=1}^{\infty}\frac{x^n 5^{-n}}{n}$.}
        \q{}
            \begin{questionario}
                \qq{Encontre a s\'erie de Maclaurin de $f(x)=\cos{x}$.}
                \qq{Encontre a s\'erie de Maclaurin de $g(x) = f(x^4)$.}
                \qq{Calcule $g^{(48)}(0)$. (derivada de ordem 48)}
            \end{questionario}
    \end{questionario}
\end{document}
