\documentclass{prova}

\usepackage{amsmath}
\usepackage{amsfonts}

\setlength{\textheight}{25cm}

\renewcommand{\sin}{\,\mbox{sen}\,}
\newcommand{\ds}{\displaystyle}

\professor{Prof.\@ Adriano Barbosa}
\disciplina{C\'alculo Diferencial e Integral II}
\avaliacao{PS}
\curso{Engenharia de Produ\c{c}\~ao}
\data{29/11/2021}

\begin{document}
	\cabecalho{5}  % o numero 5 indica a qnt de quadros na tabela de nota

    \textbf{Todas as respostas devem ser justificadas.}

    \vspace{0.5cm}
    \textbf{Avalia\c{c}\~ao P1:}
    \begin{questionario}
        \q{Calcule a integral definida $\ds\int_0^1 (x+2)(x+1)^5\ dx$.}
        \q{Calcule a integral indefinida $\ds\int e^{3x} \cos(2x)\ dx$.}
        \q{Calcule a integral $\ds\int_0^1 \frac{e^{1/x}}{x^2}\ dx$.}
        \q{Calcule a integral impr\'opria $\ds\int_1^\infty x e^{-x}\ dx$.}
        \q{Encontre uma primitiva para a fun\c{c}\~ao $f(x) = \ds\frac{1}{x^3+9x^2}$.}
    \end{questionario}

    \vspace{0.5cm}
    \textbf{Avalia\c{c}\~ao P2:}
    \begin{questionario}
        \q{Verifique se cada fun\c{c}\~ao \'e solu\c{c}\~ao da equa\c{c}\~ao dada:}
            \begin{enumerate}
                 \qq{(1 pt) $y = -\ds\frac{2}{x^2+1}$, $y'=xy^2$.}
                 \qq{(1 pt) $y = \sin(\ln{x})$, $x^2y''+xy'+y=0$}
            \end{enumerate}
        \q{(2 pt) Encontre a solu\c{c}\~ao do PVI}
            \[y''-2y-3y = 6x+4, y(1)=4, y'(1)=-2\]
        \q{(2 pt) Encontre a solu\c{c}\~ao geral da equa\c{c}\~ao diferencial $y' =
        \ds\frac{t}{ye^{y+t^2}}$.}
        \q{(2 pt) Resolva a EDO $ts' = t^2\cos{(t)} - s$, $t>0$.}
        \q{(2 pt) Resolva a equa\c{c}\~ao $y''+16y=0$.}
    \end{questionario}

    \vspace{0.5cm}
    \textbf{Avalia\c{c}\~ao P3:}
    \begin{questionario}
        \q{(2 pt) Encontre o termo geral das sequ\^encias e determine se s\~ao
        convergentes ou divergentes:}
            \begin{questionario}
                \qq{$\ds\frac{3}{4-1}, \frac{4}{9-4}, \frac{5}{16-9}, \ldots$}
                \qq{$\ds\frac{1}{3}, -\frac{2}{5}, \frac{3}{7}, -\frac{4}{9}, \ldots$}
            \end{questionario}
        \q{Determine se as s\'eries s\~ao convergentes ou divergentes.}
            \begin{questionario}
                \qq{(1 pt) $\ds\frac{1}{3}+\frac{1}{6}+\frac{1}{9}+\frac{1}{12}+\frac{1}{15}+\cdots$}
                \qq{(1 pt) $\ds\sum_{n=1}^\infty \frac{1+3^n}{2^n}$}
                \qq{(1 pt) $\ds\sum_{n=1}^\infty \left(\frac{1}{n}\right)^{\sqrt{2}}$}
                \qq{(1 pt) $\ds\sum_{n=1}^\infty \left(\frac{1}{\sqrt{n}} - \frac{1}{\sqrt{n+1}}\right)$}
            \end{questionario}
        \q{(2 pt) Determine o raio e o intervalo de converg\^encia a da s\'erie
        $\ds\sum_{n=1}^\infty \frac{n}{4^n}(x+1)^n$.}
        \q{(2 pt) Encontre a s\'erie de Maclaurin da fun\c{c}\~ao $f(x) =
        \ds\frac{1}{(1-x)^2}$ e calcule seu intervalo de converg\^encia.}
    \end{questionario}
\end{document}
