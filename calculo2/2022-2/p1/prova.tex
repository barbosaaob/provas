\documentclass{prova}

\usepackage{amsmath}
\usepackage{amsfonts}

\setlength{\textheight}{25cm}

\renewcommand{\sin}{\,\mbox{sen}\,}
\newcommand{\ds}{\displaystyle}

\professor{Prof.\@ Adriano Barbosa}
\disciplina{C\'alculo Diferencial e Integral II}
\avaliacao{P1}
\curso{Engenharia Civil}
\data{24/01/2023}

\begin{document}
	\cabecalho{5}  % o numero 5 indica a qnt de quadros na tabela de nota

    \textbf{Todas as respostas devem ser justificadas.}

    \begin{questionario}
        \q{Calcule os limites abaixo:}
            \begin{questionario}
                \qq{$\ds\lim_{n\rightarrow\infty} \frac{n^{2023}+n^{2022}}{4n^{2024}+2n^{2023}}$}
                \qq{$\ds\lim_{n\rightarrow\infty} \frac{2^n}{3^n}$}
            \end{questionario}
        \q{Determine se a s\'erie abaixo \'e convergente ou divergente e calcule
           sua coma caso seja convergente:}
           \[3-4+\frac{16}{3}-\frac{64}{9}+\cdots\]
        \q{Determine se as s\'eries s\~ao convergentes ou divergentes:}
            \begin{questionario}
                \qq{$\ds\sum_{n=1}^{\infty} \frac{n^3}{5^n}$}
                \qq{$\ds\sum_{n=1}^{\infty} \frac{n^{2n}}{(1+2n^2)^n}$}
            \end{questionario}
        \q{Determine o raio e o intervalo de converg\^encia da s\'erie $\ds\sum_{n=1}^{\infty} \frac{3^n(x+4)^n}{\sqrt{n}}$.}
        \q{Calcule a s\'erie de Taylor da fun\c{c}\~ao $f(x)=\ln{x}$ centrada em $a=1$.}
    \end{questionario}
\end{document}
