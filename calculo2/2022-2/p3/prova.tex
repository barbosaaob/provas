\documentclass{prova}

\usepackage{amsmath}
\usepackage{amsfonts}

\setlength{\textheight}{25cm}

\renewcommand{\sin}{\,\mbox{sen}\,}
\newcommand{\ds}{\displaystyle}

\professor{Prof.\@ Adriano Barbosa}
\disciplina{C\'alculo Diferencial e Integral II}
\avaliacao{P3}
\curso{Engenharia Civil}
\data{18/04/2023}

\begin{document}
	\cabecalho{5}  % o numero 5 indica a qnt de quadros na tabela de nota

    \textbf{Todas as respostas devem ser justificadas.}

    \begin{questionario}
        \q{Encontre a solu\c{c}\~ao da EDO de primeira ordem $x\ln{x} =
           y(1+\sqrt{3+y^2})y'$.}
        \q{Resolva a EDO $xy''-6x^2=-y'$ fazendo a substitui\c{c}\~ao $u=y'$ para
           $x>0$.}
        \q{Encontre a solu\c{c}\~ao da equa\c{c}\~ao de Bernoulli $y'=-y+2xy^2$.}
        \q{Encontre a solu\c{c}\~ao geral da equa\c{c}\~ao $2y''=y'$.}
        \q{Resolva a equa\c{c}\~ao diferencial $\ds 2ye^{y^2}y' = 2x+3\sqrt{x}$.}
        %\q{Quando um cabo flex\'{\i}vel de densidade uniforme \'e suspenso entre dois
        %   pontos fixos e fica pendurado \`a merc\^e de seu pr\'oprio peso, a curva $y =
        %   f(x)$ desenhada pelo cabo satisfaz uma equa\c{c}\~ao diferencial do tipo}
        %   \[\frac{d^2y}{dx^2} = k\sqrt{1+\left(\frac{dy}{dx}\right)^2}\]
        %   onde $k$ \'e uma constante positiva. Resolva a equa\c{c}\~ao reduzindo-a a
        %   uma equa\c{c}\~ao de primeira ordem utilizando a mudan\c{c}a de vari\'aveis
        %   $z=\ds\frac{dy}{dx}$.
        %\q{Um estudante esqueceu a Regra do Produto para a derivada e cometeu o
        %   erro de pensar que $(fg)'=f'g'$. Contudo, ele teve sorte e obteve a
        %   resposta certa. A fun\c{c}\~ao $f$ que ele usou era $f(x)=e^{x^2}$ e o
        %   dom\'{\i}nio de seu problema era o intervalo
        %   $\left(\frac{1}{2},\infty\right)$. Qual era a fun\c{c}\~ao $g$?}
    \end{questionario}
\end{document}
