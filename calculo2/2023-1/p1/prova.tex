\documentclass{prova}

\usepackage{amsmath}
\usepackage{amsfonts}

\setlength{\textheight}{25cm}

\renewcommand{\sin}{\,\mbox{sen}\,}
\newcommand{\ds}{\displaystyle}

\professor{Prof.\@ Adriano Barbosa}
\disciplina{C\'alculo Diferencial e Integral II}
\avaliacao{P1}
\curso{Qu\'{\i}mica}
\data{04/07/2023}

\begin{document}
	\cabecalho{5}  % o numero 5 indica a qnt de quadros na tabela de nota

    \textbf{Todas as respostas devem ser justificadas.}

    \begin{questionario}
        \q{Calcule a integral indefinida $\ds\int \sqrt{x}\sin{(1+x^{3/2})}\
           dx$.}
        \q{Calcule a integral $\ds\int_0^9 \frac{1}{\sqrt[3]{x-1}}\ dx$.}
        \q{Determine o valor da integral definida $\ds\int_1^5 x^2\ln{x}\ dx$.}
        \q{Determine se as afirma\c{c}\~oes abaixo s\~ao verdadeiras ou falsas.
           Reescreva a soma de fra\c{c}\~oes parciais correta para as falsas. N\~ao \'e
           necess\'ario calcular as constantes $A$, $B$ e $C$.}
           \begin{questionario}
               \qq{$\ds\frac{x(x^2+4)}{x^2-4}$ pode ser escrita como soma de
                   fra\c{c}\~oes parciais da forma $\ds\frac{A}{x+2} +
                   \frac{B}{x-2}$.}
               \qq{$\ds\frac{x^2+4}{x(x^2-4)}$ pode ser escrita como soma de
                   fra\c{c}\~oes parciais da forma $\ds\frac{A}{x} +
                   \frac{B}{x+2} + \frac{C}{x-2}$.}
               \qq{$\ds\frac{x^2+4}{x^2(x-4)}$ pode ser escrita como soma de
                   fra\c{c}\~oes parciais da forma $\ds\frac{A}{x^2} +
                   \frac{B}{x-4}$.}
               \qq{$\ds\frac{x^2-4}{x(x^2+4)}$ pode ser escrita como soma de
                   fra\c{c}\~oes parciais da forma $\ds\frac{A}{x} +
                   \frac{B}{x^2+4}$.}
           \end{questionario}
        \q{Determine se a integral impr\'opria $\ds\int_2^\infty e^{-5p}\ dp$ \'e
           convergente ou divergente e calcule seu valor se for convergente.}
    \end{questionario}
\end{document}
