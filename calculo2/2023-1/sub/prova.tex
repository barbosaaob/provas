\documentclass{prova}

\usepackage{amsmath}
\usepackage{amsfonts}

\setlength{\textwidth}{18cm}
\setlength{\textheight}{27cm}
\setlength{\topmargin}{-2.3cm}
\setlength{\oddsidemargin}{-1cm}

\renewcommand{\sin}{\,\mbox{sen}\,}
\newcommand{\ds}{\displaystyle}

\professor{Prof.\@ Adriano Barbosa}
\disciplina{C\'alculo Diferencial e Integral II}
\avaliacao{PS}
\curso{Qu\'{\i}mica}
\data{05/09/2023}

\begin{document}
	\cabecalho{5}  % o numero 5 indica a qnt de quadros na tabela de nota

    \textbf{Todas as respostas devem ser justificadas.}

    \vspace{0.5cm}
    \textbf{Avalia\c{c}\~ao P1:}
    \begin{questionario}
        \q{Calcule a integral indefinida $\ds\int x^{1/2}\cos{(1+x^{3/2})}\
           dx$.}
        \q{Calcule a integral $\ds\int_1^3 \frac{1}{\sqrt{x-1}}\ dx$.}
        \q{Determine o valor da integral definida $\ds\int_1^2 x\ln{x}\ dx$.}
        \q{Determine se as afirma\c{c}\~oes abaixo s\~ao verdadeiras ou falsas.
           Reescreva a soma de fra\c{c}\~oes parciais correta para as falsas. N\~ao \'e
           necess\'ario calcular as constantes $A$, $B$ e $C$.}
           \begin{questionario}
               \qq{$\ds\frac{x(x^2+9)}{x^2-9}$ pode ser escrita como soma de
                   fra\c{c}\~oes parciais da forma $\ds\frac{A}{x+3} +
                   \frac{B}{x-3}$.}
               \qq{$\ds\frac{x^2+9}{x(x^2-9)}$ pode ser escrita como soma de
                   fra\c{c}\~oes parciais da forma $\ds\frac{A}{x} +
                   \frac{B}{x+3} + \frac{C}{x-3}$.}
               \qq{$\ds\frac{x^2+9}{x^2(x-9)}$ pode ser escrita como soma de
                   fra\c{c}\~oes parciais da forma $\ds\frac{A}{x^2} +
                   \frac{B}{x-9}$.}
               \qq{$\ds\frac{x^2-9}{x(x^2+9)}$ pode ser escrita como soma de
                   fra\c{c}\~oes parciais da forma $\ds\frac{A}{x} +
                   \frac{B}{x^2+9}$.}
           \end{questionario}
        \q{Determine se a integral impr\'opria $\ds\int_1^\infty e^{-3x}\ dx$ \'e
           convergente ou divergente e calcule seu valor se for convergente.}
    \end{questionario}

    \textbf{Avalia\c{c}\~ao P2:}
    \begin{questionario}
        \q{Use a mudan\c{c}a de vari\'aveis $u=y/x$ para resolver a EDO $xy' =
           y+xe^{y/x}$.}
        \q{Resolva a equa\c{c}\~ao diferencial $y'+y = \cos(e^x)$.}
        \q{Resolva o problema de valor inicial $y''-3y'+2y = 0$, $y(0)=2$ e
           $y'(0)=1$.}
        \q{Determine se a s\'erie $-3+4-\ds\frac{16}{3}+\frac{64}{9}+\ldots$ \'e
           convergente e calcule sua soma, se poss\'{\i}vel.}
        \q{Encontre a s\'erie de Maclaurin de $f(x) = \cos(x)$ e determine seu
           intervalo de converg\^encia.}
    \end{questionario}
\end{document}
