\documentclass[a4paper,5pt]{amsbook}
%%%%%%%%%%%%%%%%%%%%%%%%%%%%%%%%%%%%%%%%%%%%%%%%%%%%%%%%%%%%%%%%%%%%%

\usepackage{booktabs}
% \usepackage{graphics}
% \usepackage[]{float}
% \usepackage{amssymb}
% \usepackage{amsfonts}
% \usepackage[]{amsmath}
% \usepackage[]{epsfig}
% \usepackage[brazil]{babel}
% \usepackage[utf8]{inputenc}
% \usepackage{verbatim}
%\usepackage[]{pstricks}
%\usepackage[notcite,notref]{showkeys}

%%%%%%%%%%%%%%%%%%%%%%%%%%%%%%%%%%%%%%%%%%%%%%%%%%%%%%%%%%%%%%

\newcommand{\sen}{\text{sen}}
\newcommand{\ds}{\displaystyle}

%%%%%%%%%%%%%%%%%%%%%%%%%%%%%%%%%%%%%%%%%%%%%%%%%%%%%%%%%%%%%%%%%%%%%%%%

\setlength{\textwidth}{16cm} %\setlength{\topmargin}{-0.1cm}
\setlength{\leftmargin}{1.2cm} \setlength{\rightmargin}{1.2cm}
\setlength{\oddsidemargin}{0cm}\setlength{\evensidemargin}{0cm}

%%%%%%%%%%%%%%%%%%%%%%%%%%%%%%%%%%%%%%%%%%%%%%%%%%%%%%%%%%%%%%%%%%%%%%%%

% \renewcommand{\baselinestretch}{1.6}
% \renewcommand{\thefootnote}{\fnsymbol{footnote}}
% \renewcommand{\theequation}{\thesection.\arabic{equation}}
% \setlength{\voffset}{-50pt}
% \numberwithin{equation}{chapter}

%%%%%%%%%%%%%%%%%%%%%%%%%%%%%%%%%%%%%%%%%%%%%%%%%%%%%%%%%%%%%%%%%%%%%%%

\begin{document}
\thispagestyle{empty}
\begin{minipage}[b]{0.45\linewidth}
\begin{tabular}{c}
\toprule{}
{{\bf UNIVERSIDADE FEDERAL DA GRANDE DOURADOS}}\\
{{\bf Prof.\ Adriano Barbosa}}\\



{{\bf C\'alculo II}}\\

\midrule{}
\hspace{8cm}28 de Setembro de 2016 \\
\bottomrule{}
\end{tabular}
%
\end{minipage} \hfill
\begin{minipage}[b]{0.58\linewidth}
\begin{flushright}
\def\arraystretch{1.2}
\begin{tabular}{|c|c|}  % chktex 44
\hline\hline  % chktex 44
1 & \hspace{1.2cm} \\
\hline  % chktex 44
2& \\
\hline  % chktex 44
3& \\
\hline  % chktex 44
4&  \\
\hline  % chktex 44
5&  \\
\hline  % chktex 44
{\small Total}&  \\
\hline\hline  % chktex 44
\end{tabular}
\end{flushright}
\end{minipage} \hfill
%------------------------
\vspace{0.3cm}\\
{\bf Aluno(a):}\dotfill{} \\  % chktex 36
%----------------------------


\vspace{0.2cm}
%%%%%%%%%%%%%%%%%%%%%%%%%%%%%%%%   formulario  in\'icio  %%%%%%%%%%%%%%%%%%%%%%%%%%%%%%%%
\begin{enumerate}

\item Classifique a EDO $y^{(3)} - y' = 0$ quando a ordem e verifique se $y(t) =
	c_1 e^t + c_2 e^{-t}$ \'e uma solu\c{c}\~ao.
\vspace{0.8cm}

\noindent{}\textbf{Nos itens 2 e 3 classifique as equa\c{c}\~oes em separ\'avel e linear (ou ambas) e resolva o PVI.}
\vspace{0.2cm}

\item $\ds y' = \frac{\ln(x)}{xy}$, $y(1)=2$.
\vspace{1cm}

\item $x^2y' + 2xy = \ln(x)$, $y(1)=2$.
\vspace{1cm}

\item
	\begin{enumerate}
		\item Encontre a solu\c{c}\~ao geral da EDO $y''-6y'+8y=0$.
		\item Encontre a solu\c{c}\~ao da EDO acima que satisfaz $y(0)=2$ e
			$y'(0)=2$.
	\end{enumerate}
\vspace{1cm}

\item Encontre a solu\c{c}\~ao geral da EDO n\~ao homog\^enea $y''+y = \text{tg}(x)$,
	$0<x<\frac{\pi}{2}$.

\end{enumerate}

\begin{flushright}
	\textit{Boa Prova!}
\end{flushright}

\end{document}
