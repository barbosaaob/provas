\documentclass[a4paper,5pt]{amsbook}
%%%%%%%%%%%%%%%%%%%%%%%%%%%%%%%%%%%%%%%%%%%%%%%%%%%%%%%%%%%%%%%%%%%%%

\usepackage{booktabs}
\usepackage{graphicx}
% \usepackage[]{float}
\usepackage{amssymb}
% \usepackage{amsfonts}
% \usepackage[]{amsmath}
% \usepackage[]{epsfig}
% \usepackage[brazil]{babel}
% \usepackage[utf8]{inputenc}
% \usepackage{verbatim}
%\usepackage[]{pstricks}
%\usepackage[notcite,notref]{showkeys}
\usepackage{subcaption}

%%%%%%%%%%%%%%%%%%%%%%%%%%%%%%%%%%%%%%%%%%%%%%%%%%%%%%%%%%%%%%

\newcommand{\sen}{\,\mbox{sen}\,}
\newcommand{\tg}{\,\mbox{tg}\,}
\newcommand{\cosec}{\,\mbox{cosec}\,}
\newcommand{\cotg}{\,\mbox{cotg}\,}
\newcommand{\ds}{\displaystyle}

%%%%%%%%%%%%%%%%%%%%%%%%%%%%%%%%%%%%%%%%%%%%%%%%%%%%%%%%%%%%%%%%%%%%%%%%

\setlength{\textwidth}{16cm} \setlength{\topmargin}{-1cm}
\setlength{\textheight}{25cm}
\setlength{\leftmargin}{1cm} \setlength{\rightmargin}{1cm}
\setlength{\oddsidemargin}{0cm}\setlength{\evensidemargin}{0cm}

%%%%%%%%%%%%%%%%%%%%%%%%%%%%%%%%%%%%%%%%%%%%%%%%%%%%%%%%%%%%%%%%%%%%%%%%

% \renewcommand{\baselinestretch}{1.6}
% \renewcommand{\thefootnote}{\fnsymbol{footnote}}
% \renewcommand{\theequation}{\thesection.\arabic{equation}}
% \setlength{\voffset}{-50pt}
% \numberwithin{equation}{chapter}

%%%%%%%%%%%%%%%%%%%%%%%%%%%%%%%%%%%%%%%%%%%%%%%%%%%%%%%%%%%%%%%%%%%%%%%

\begin{document}
\thispagestyle{empty}
\hspace{-0.6cm}
\begin{minipage}[p]{0.14\linewidth}
	\includegraphics[scale=0.24]{ufgd.png}
\end{minipage}
\begin{minipage}[p]{0.7\linewidth}
\begin{tabular}{c}
\toprule{}
{{\bf UNIVERSIDADE FEDERAL DA GRANDE DOURADOS}}\\
{{\bf Prof.\ Adriano Barbosa}}\\

{{\bf C\'alculo 2 --- Avalia\c{c}\~ao P3}}\\

\midrule{}
Eng.\ Mec\^anica\hspace{5cm}31 de Mar\c{c}o de 2017 \\
\bottomrule{}
\end{tabular}
\vspace{-0.45cm}
%
\end{minipage}
\begin{minipage}[p]{0.15\linewidth}
\begin{flushright}
\def\arraystretch{1.2}
\begin{tabular}{|c|c|}  % chktex 44
\hline\hline  % chktex 44
1 & \hspace{1.2cm} \\
\hline  % chktex 44
2& \\
\hline  % chktex 44
3& \\
\hline  % chktex 44
4&  \\
\hline  % chktex 44
5&  \\
\hline  % chktex 44
{\small Total}&  \\
\hline\hline  % chktex 44
\end{tabular}
\end{flushright}
\end{minipage}

%------------------------
\vspace{0.5cm}
{\bf Aluno(a):}\dotfill{}  % chktex 36
%----------------------------

\vspace{0.2cm}
%%%%%%%%%%%%%%%%%%%%%%%%%%%%%%%%   formulario  inicio  %%%%%%%%%%%%%%%%%%%%%%%%%%%%%%%%
\begin{enumerate}
	\vspace{0.5cm}
	\item Resolva o PVI\@: $\ds y' = \frac{xy\sen(x)}{y+1}$, $y(0)=1$.

	\vspace{0.5cm}
	\item Resolva o problema de valor inicial $\ds t u' = t^2 + 3u$, $t>0$, $u(2)=4$.

	\vspace{0.5cm}
	\item A fun\c{c}\~ao $y(x) = \sen(x)$ \'e solu\c{c}\~ao da EDO ${(y''' - y)}^2 = 2\sen(x)\cos(x) + 1$?

	\vspace{0.5cm}
	\item Encontre a solu\c{c}\~ao do problema de valor inicial $4y'' - 20y' + 25y = 0$, $y(0) = 2$, $y'(0) = -3$.

	\vspace{0.5cm}
	\item Encontre a solu\c{c}\~ao geral da EDO $y''' - y' = 0$. [Dica: use a substitui\c{c}\~ao $u = y'$]

	\vspace{0.5cm}
	\item[(B\^onus)] Uma equa\c{c}\~ao diferencial da forma
		\[y' + P(x) y = Q(x) y^n\]
		\'e chamada equa\c{c}\~ao de Bernoulli. Essa equa\c{c}\~ao pode ser resolvida usando a
		substitui\c{c}\~ao $u = y^{1-n}$, que transforma a equa\c{c}\~ao de Bernoulli em
		\[u'+(1-n)P(x)u=(1-n)Q(x).\]
	   	Resolva a equa\c{c}\~ao diferencial $xy' + y = -xy^2$.
\end{enumerate}

\begin{flushright}
	\textit{Boa Prova!}
\end{flushright}

\end{document}
