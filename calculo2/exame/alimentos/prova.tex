\documentclass{article}

\usepackage[inline]{enumitem}
\usepackage{graphicx}
\usepackage{amssymb}

\newcommand{\ds}{\displaystyle}
\newcommand{\sen}{\mbox{sen}}
\newcommand{\tg}{\mbox{tg}}
\newcommand{\cosec}{\mbox{cosec}}
\newcommand{\cotg}{\mbox{cotg}}

\begin{document}
\noindent{}\rule{\textwidth}{0.4pt}
\begin{center}
	C\'alculo 2\\
	Exame --- 19/05/2016 \\
	Engenharia de Alimentos \\
	\vspace{0.2cm}
	% Prof. Adriano Barbosa
\end{center}
Nome: \\
\noindent{}\rule{\textwidth}{0.4pt}

\begin{enumerate}
%%%%%%%%%%%%%%%%%%%%%%%%%%%%%%%%%%%%%%%%%%%%%
\item Calcule a integral indefinida $\ds\int x^2\cos(x)\ dx$.

%%%%%%%%%%%%%%%%%%%%%%%%%%%%%%%%%%%%%%%%%%%%%
\item Calcule a integral $\ds\int \frac{dx}{x\ln(x)}\ dx$.

%%%%%%%%%%%%%%%%%%%%%%%%%%%%%%%%%%%%%%%%%%%%%
\item Calcule o limites abaixo:
	\begin{enumerate}
		\item $\ds\lim_{n\rightarrow\infty}\frac{n^2}{n^2+3n}$
		\item $\ds\lim_{n\rightarrow\infty}\frac{e^n+e^{-n}}{e^{2n}-1}$
	\end{enumerate}

%%%%%%%%%%%%%%%%%%%%%%%%%%%%%%%%%%%%%%%%%%%%%
\item Calcule o raio e o intervalo de converg\^encia da s\'erie $\ds\sum_{n=1}^\infty \frac{x^n}{n(n+1)}$.

[A s\'erie $\sum\frac{(-1)^n}{n(n+1)}$ \'e convergente.]

%%%%%%%%%%%%%%%%%%%%%%%%%%%%%%%%%%%%%%%%%%%%%
\item Resolva a equa\c{c}\~ao $\ds y'+4y=e^{-3x}$.

%%%%%%%%%%%%%%%%%%%%%%%%%%%%%%%%%%%%%%%%%%%%%
\item Resolva o problema de valor inicial $\ds(1+y^2)y'=e^{x}y$, $y(0)=1$.

\end{enumerate}
\end{document}
