\documentclass[a4paper,5pt]{amsbook}
%%%%%%%%%%%%%%%%%%%%%%%%%%%%%%%%%%%%%%%%%%%%%%%%%%%%%%%%%%%%%%%%%%%%%

\usepackage{booktabs}
\usepackage{graphicx}
% \usepackage[]{float}
\usepackage{amssymb}
% \usepackage{amsfonts}
% \usepackage[]{amsmath}
% \usepackage[]{epsfig}
% \usepackage[brazil]{babel}
% \usepackage[utf8]{inputenc}
% \usepackage{verbatim}
%\usepackage[]{pstricks}
%\usepackage[notcite,notref]{showkeys}
\usepackage{subcaption}
\usepackage[inline]{enumitem}

%%%%%%%%%%%%%%%%%%%%%%%%%%%%%%%%%%%%%%%%%%%%%%%%%%%%%%%%%%%%%%

\newcommand{\sen}{\,\mbox{sen}\,}
\newcommand{\tg}{\,\mbox{tg}\,}
\newcommand{\cosec}{\,\mbox{cosec}\,}
\newcommand{\cotg}{\,\mbox{cotg}\,}
\newcommand{\ds}{\displaystyle}

%%%%%%%%%%%%%%%%%%%%%%%%%%%%%%%%%%%%%%%%%%%%%%%%%%%%%%%%%%%%%%%%%%%%%%%%

\setlength{\textwidth}{16cm} \setlength{\topmargin}{-1cm}
\setlength{\textheight}{25cm}
\setlength{\leftmargin}{1.2cm} \setlength{\rightmargin}{1.2cm}
\setlength{\oddsidemargin}{0cm}\setlength{\evensidemargin}{0cm}

%%%%%%%%%%%%%%%%%%%%%%%%%%%%%%%%%%%%%%%%%%%%%%%%%%%%%%%%%%%%%%%%%%%%%%%%

% \renewcommand{\baselinestretch}{1.6}
% \renewcommand{\thefootnote}{\fnsymbol{footnote}}
% \renewcommand{\theequation}{\thesection.\arabic{equation}}
% \setlength{\voffset}{-50pt}
% \numberwithin{equation}{chapter}

%%%%%%%%%%%%%%%%%%%%%%%%%%%%%%%%%%%%%%%%%%%%%%%%%%%%%%%%%%%%%%%%%%%%%%%

\begin{document}
\thispagestyle{empty}
\hspace{-0.6cm}
\begin{minipage}[p]{0.14\linewidth}
	\includegraphics[scale=0.24]{ufgd.png}
\end{minipage}
\begin{minipage}[p]{0.7\linewidth}
\begin{tabular}{c}
\toprule{}
{{\bf UNIVERSIDADE FEDERAL DA GRANDE DOURADOS}}\\
{{\bf Prof.\ Adriano Barbosa}}\\

{{\bf C\'alculo 2 --- Exame}}\\

\midrule{}
Eng. Mec\^anica\hspace{5cm}13 de Abril de 2017 \\
\bottomrule{}
\end{tabular}
\vspace{-0.45cm}
%
\end{minipage}
\begin{minipage}[p]{0.15\linewidth}
\begin{flushright}
\def\arraystretch{1.2}
\begin{tabular}{|c|c|}  % chktex 44
\hline\hline  % chktex 44
1 & \hspace{1.2cm} \\
\hline  % chktex 44
2& \\
\hline  % chktex 44
3& \\
\hline  % chktex 44
4&  \\
\hline  % chktex 44
5&  \\
\hline  % chktex 44
{\small Total}&  \\
\hline\hline  % chktex 44
\end{tabular}
\end{flushright}
\end{minipage}

%------------------------
\vspace{0.5cm}
{\bf Aluno(a):}\dotfill{}  % chktex 36
%----------------------------

\vspace{0.2cm}
%%%%%%%%%%%%%%%%%%%%%%%%%%%%%%%%   formulario  inicio  %%%%%%%%%%%%%%%%%%%%%%%%%%%%%%%%
\begin{enumerate}
	\vspace{0.5cm}
	\item Calcule a integral definida $\ds \int_1^2 x^4{(\ln{x})}^2\ dx$.

	\vspace{0.5cm}
	\item Determine se as s\'eries abaixo s\~ao convergentes ou divergentes:
		\begin{enumerate}
			\vspace{0.3cm}
			\item $\ds\sum_{n=0}^\infty \left(\frac{\pi}{3}\right)^n$
			\vspace{0.3cm}
			\item $\ds\sum_{n=1}^\infty \frac{(-2)^n}{n^n}$
		\end{enumerate}

	\vspace{0.5cm}
	\item Escreva a s\'erie de Taylor da fun\c{c}\~ao $f(x) = \sen{x}$ centrada em $a =
		\frac{\pi}{2}$ e calcule seu intervalo de converg\^encia.

	\vspace{0.5cm}
	\item Calcule o problema de valor inicial: $\ds (1+\cos{x}) y' = \frac{e^y+1}{e^y} \sen{x}$, $y(0) = 0$.

	\vspace{0.5cm}
	\item Resolva o PVI: $xy' - y = x \ln{x}$, $y(1) = 0$.
\end{enumerate}

% \vfill{}
% F\'ormulas \'uteis:
% \[\begin{array}{llll}
% 	\vspace{0.3cm}
% 	\cosec(x) = \displaystyle\frac{1}{\sen(x)}, & \sec(x) = \displaystyle\frac{1}{\cos(x)}, & \cotg(x) = \displaystyle\frac{\cos(x)}{\sen(x)}, & \ds\tg(x) = \frac{\sen(x)}{\cos(x)}\\
% 	\vspace{0.3cm}
% 	\sen^2(x) + \cos^2(x) = 1, & \tg^2(x) + 1 = \sec^2(x), & 1 + \cotg^2(x) = \cosec^2(x) & \\
% 	\vspace{0.3cm}
% 	\sen^2(x) = \displaystyle\frac{1 - \cos(2x)}{2}, & \cos^2(x) = \displaystyle\frac{1 + \cos(2x)}{2} & & \\
% 	\multicolumn{2}{l}{\sen(x+y) = \displaystyle\sen(x)\cos(y) + \sen(y)\cos(x),} & \multicolumn{2}{l}{\cos(x+y) = \displaystyle\cos(x)\cos(y) - \sen(x)\sen(y)} \\
% 	& & & \\
% 	\multicolumn{2}{l}{\sen(x-y) = \displaystyle\sen(x)\cos(y) - \sen(y)\cos(x),} & \multicolumn{2}{l}{\cos(x-y) = \displaystyle\cos(x)\cos(y) + \sen(x)\sen(y)}
% \end{array}\]

\begin{flushright}
	\vspace{1cm}
	\textit{Boa Prova!}
\end{flushright}

\end{document}
