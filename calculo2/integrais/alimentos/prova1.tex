\documentclass{article}

\usepackage[inline]{enumitem}
\usepackage{graphicx}

\newcommand{\sen}{\mbox{sen}}
\newcommand{\tg}{\mbox{tg}}
\newcommand{\cosec}{\mbox{cosec}}
\newcommand{\cotg}{\mbox{cotg}}

\begin{document}
\noindent{}\rule{\textwidth}{0.4pt}
\begin{center}
	C\'alculo 2\\
	Avalia\c{c}\~ao P1 --- 10/03/2016 \\
	Engenharia de Alimentos \\
	\vspace{0.2cm}
	% Prof. Adriano Barbosa
\end{center}
Nome: \\
\noindent{}\rule{\textwidth}{0.4pt}

\begin{enumerate}
%%%%%%%%%%%%%%%%%%%%%%%%%%%%%%%%%%%%%%%%%%%%%
\item Calcule a integral $\displaystyle\int \frac{x}{1-x^2}\ dx$.

%%%%%%%%%%%%%%%%%%%%%%%%%%%%%%%%%%%%%%%%%%%%%
\item
	\begin{enumerate}
		\item Calcule a integral indefinida $\displaystyle\int \frac{x}{x^2+1}\ dx$.
		\item Calcule a integral impr\'opria $\displaystyle\int_1^\infty \frac{x}{x^2+1}\ dx$.
	\end{enumerate}

%%%%%%%%%%%%%%%%%%%%%%%%%%%%%%%%%%%%%%%%%%%%%
\item Utilize substitui\c{c}\~ao trigonom\'etrica para calcular a integral $\displaystyle\int_0^1 \sqrt{1-x^2}\ dx$.

%%%%%%%%%%%%%%%%%%%%%%%%%%%%%%%%%%%%%%%%%%%%%
\item Utilize uma substitui\c{c}\~ao e ent\~ao utilize integra\c{c}\~ao por partes para calcular a integral $\displaystyle\int\cos(\sqrt{x})\ dx$.

%%%%%%%%%%%%%%%%%%%%%%%%%%%%%%%%%%%%%%%%%%%%%
\item Dados os polin\^omios $p(x) = x^3+4x$ e $q(x)=x^2-4$:
	\begin{enumerate}
		\item Fa\c{c}a a divis\~ao de $p(x)$ por $q(x)$.
		\item Fatore o polin\^omio $q(x)$.
		\item Escreva $\displaystyle\frac{8x}{q(x)}$ como soma de fra\c{c}\~oes parciais.
		\item Calcule $\displaystyle\int\frac{p(x)}{q(x)}\ dx$.
	\end{enumerate}

\end{enumerate}
F\'ormulas:

$\begin{array}{ccc}
	\vspace{0.5cm}
	\cosec(x) = \displaystyle\frac{1}{\sen(x)} & \sec(x) = \displaystyle\frac{1}{\cos(x)} & \cotg(x) = \displaystyle\frac{\cos(x)}{\sen(x)} \\
	\vspace{0.5cm}
	\sen^2(x) + \cos^2(x) = 1 & \tg^2(x) + 1 = \sec^2(x) & 1 + \cotg^2(x) = \cosec^2(x) \\
	\vspace{0.5cm}
	\sen^2(x) = \displaystyle\frac{1 - \cos(2x)}{2} & \cos^2(x) = \displaystyle\frac{1 + \cos(2x)}{2} &
\end{array}$
\end{document}
