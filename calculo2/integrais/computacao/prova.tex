\documentclass[a4paper,5pt]{amsbook}
%%%%%%%%%%%%%%%%%%%%%%%%%%%%%%%%%%%%%%%%%%%%%%%%%%%%%%%%%%%%%%%%%%%%%

\usepackage{booktabs}
% \usepackage{graphics}
% \usepackage[]{float}
% \usepackage{amssymb}
% \usepackage{amsfonts}
% \usepackage[]{amsmath}
% \usepackage[]{epsfig}
% \usepackage[brazil]{babel}
% \usepackage[utf8]{inputenc}
% \usepackage{verbatim}
%\usepackage[]{pstricks}
%\usepackage[notcite,notref]{showkeys}

%%%%%%%%%%%%%%%%%%%%%%%%%%%%%%%%%%%%%%%%%%%%%%%%%%%%%%%%%%%%%%

\newcommand{\sen}{\text{sen}}
\newcommand{\tg}{\mbox{tg}}
\newcommand{\cosec}{\mbox{cosec}}
\newcommand{\cotg}{\mbox{cotg}}
\newcommand{\ds}{\displaystyle}

%%%%%%%%%%%%%%%%%%%%%%%%%%%%%%%%%%%%%%%%%%%%%%%%%%%%%%%%%%%%%%%%%%%%%%%%

\setlength{\textwidth}{16cm} %\setlength{\topmargin}{-0.1cm}
\setlength{\leftmargin}{1.2cm} \setlength{\rightmargin}{1.2cm}
\setlength{\oddsidemargin}{0cm}\setlength{\evensidemargin}{0cm}

%%%%%%%%%%%%%%%%%%%%%%%%%%%%%%%%%%%%%%%%%%%%%%%%%%%%%%%%%%%%%%%%%%%%%%%%

% \renewcommand{\baselinestretch}{1.6}
% \renewcommand{\thefootnote}{\fnsymbol{footnote}}
% \renewcommand{\theequation}{\thesection.\arabic{equation}}
% \setlength{\voffset}{-50pt}
% \numberwithin{equation}{chapter}

%%%%%%%%%%%%%%%%%%%%%%%%%%%%%%%%%%%%%%%%%%%%%%%%%%%%%%%%%%%%%%%%%%%%%%%

\begin{document}
\thispagestyle{empty}
\begin{minipage}[b]{0.45\linewidth}
\begin{tabular}{c}
\toprule{}
{{\bf UNIVERSIDADE FEDERAL DA GRANDE DOURADOS}}\\
{{\bf Prof.\ Adriano Barbosa}}\\



{{\bf C\'alculo II}}\\

\midrule{}
\hspace{8cm}03 de Agosto de 2016  \\
\bottomrule{}
\end{tabular}
%
\end{minipage} \hfill
\begin{minipage}[b]{0.58\linewidth}
\begin{flushright}
\def\arraystretch{1.2}
\begin{tabular}{|c|c|}
\hline\hline
1 & \hspace{1.2cm} \\
\hline
2& \\
\hline
3& \\
\hline
4&  \\
\hline
5&  \\
\hline
{\small Total}&  \\
\hline\hline
\end{tabular}
\end{flushright}
\end{minipage} \hfill
%------------------------
\vspace{0.3cm}\\
{\bf Aluno(a):}\dotfill{} \\
%----------------------------


\vspace{0.2cm}
%%%%%%%%%%%%%%%%%%%%%%%%%%%%%%%%   formulario  in\'icio  %%%%%%%%%%%%%%%%%%%%%%%%%%%%%%%%
\begin{enumerate}

\item Calcule a integral indefinida $\ds\int x{(2x-5)}^8\ dx$.
\vspace{0.5cm}

\item Calcule a integral definida $\ds\int_1^3\ r^3 \ln(r)\ dr$.
\vspace{0.5cm}

\item Calcule as integrais:
	\begin{enumerate}
		\item $\ds\int \frac{1}{x^4}\ dx$
		\vspace{0.2cm}
		\item $\ds\int_{-2}^3 \frac{1}{x^4}\ dx$
	\end{enumerate}
\vspace{0.5cm}

\item Utilize substitui\c{c}\~ao trigonom\'etrica para calcular a integral $\ds\int \frac{x^2}{\sqrt{1-x^2}}\ dx$.
\vspace{0.5cm}

\item Dados os polin\^omios $p(x) = x^3 + 4$ e $q(x) = x^2+4$:
	\vspace{0.2cm}
	\begin{enumerate}
		\item Divida $p(x)$ por $q(x)$.
		\vspace{0.2cm}
		\item Fatore $q(x)$.
		\vspace{0.1cm}
		\item Calcule a integral $\ds\int \frac{p(x)}{q(x)}\ dx$.
	\end{enumerate}

\end{enumerate}

\vspace{1cm}
F\'ormulas \'uteis:

$\begin{array}{ccc}
	\vspace{0.5cm}
	\cosec(x) = \displaystyle\frac{1}{\sen(x)} & \sec(x) = \displaystyle\frac{1}{\cos(x)} & \cotg(x) = \displaystyle\frac{\cos(x)}{\sen(x)} \\
	\vspace{0.5cm}
	\sen^2(x) + \cos^2(x) = 1 & \tg^2(x) + 1 = \sec^2(x) & 1 + \cotg^2(x) = \cosec^2(x) \\
	\vspace{0.5cm}
	\sen^2(x) = \displaystyle\frac{1 - \cos(2x)}{2} & \cos^2(x) = \displaystyle\frac{1 + \cos(2x)}{2} & \ds\int\frac{1}{x^2+a^2}\ dx = \frac{1}{a}\ \mbox{arctg}\left(\frac{x}{a}\right) + c
\end{array}$

$\begin{array}{cc}
	\sen(x+y) = \sen(x)\cos(y) + \sen(y)\cos(x) & \sen(x-y) = \sen(x)\cos(y) - \sen(y)\cos(x)
\end{array}$
\vspace{0.5cm}

$\begin{array}{cc}
	\cos(x+y) = \cos(x)\cos(y) - \sen(x)\sen(y) & \cos(x-y) = \cos(x)\cos(y) + \sen(x)\sen(y)
\end{array}$

\begin{flushright}
	\vspace{0.5cm}
	\textit{Boa Prova!}
\end{flushright}

\end{document}
