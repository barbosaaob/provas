\documentclass{prova}

\renewcommand{\sin}{\mbox{sen}}

\professor{Prof. Adriano Barbosa}
\disciplina{C\'alculo Diferencial e Integral II}
\avaliacao{P1}
\curso{Engenharia de Energia}
\data{21/09/2018}

\begin{document}
	\cabecalho{5}  % o numero 5 indica a qnt de quadros na tabela de nota

	\textbf{Todas as respostas devem ser justificadas.}
	\begin{questionario}
        \q{Calcule a integral definida $\displaystyle\int_1^e x^2 \ln{x}\ dx$.}
        \q{Resolva a integral indefinida $\displaystyle\int \frac{\ln{x}}{x}\
        dx$.}
        \q{Calcule as integrais:}
            \begin{questionario}
                \qq{$\displaystyle\int \frac{1}{x^3}\ dx$}
                \qq{$\displaystyle\int_{-1}^1 \frac{1}{x^3}\ dx$}
            \end{questionario}
        \q{Sejam $p(x)=10$ e $q(x)=5x-2x^2$.}
            \begin{questionario}
                \qq{Fatore o polin\^omio $q(x)$.}
                \qq{Escreva $\displaystyle\frac{p(x)}{q(x)}$ como soma de
                fra\c{c}\~oes parciais.}
                \qq{Calcule a integral $\displaystyle\int \frac{p(x)}{q(x)}\
                dx$.}
            \end{questionario}
        \q{Calcule a integral $\displaystyle\int \cos{(\sqrt{x})}\ dx$. [Sugest\~ao:
        fa\c{c}a uma substitui\c{c}\~ao e em seguida use integra\c{c}\~ao por partes.]}
	\end{questionario}
\end{document}
