\documentclass{prova}

\usepackage{amsmath}
\usepackage{amsfonts}

\setlength{\textheight}{25cm}

\DeclareMathOperator{\sen}{sen}
\DeclareMathOperator{\tg}{tg}
\newcommand{\ds}{\displaystyle}

\professor{Prof.\@ Adriano Barbosa}
\disciplina{C\'alculo 2}
\avaliacao{Final}
\curso{Matem\'atica}
\data{09/11/2022}

\begin{document}
	\cabecalho{5}  % o numero 5 indica a qnt de quadros na tabela de nota

    \textbf{Todas as respostas devem ser justificadas.}

    \begin{questionario}
        \q{Sejam $f$ e $g$ fun\c{c}\~oes cont\'{\i}nuas em $[a,b]$. Determine se as
           afirma\c{c}\~oes s\~ao verdadeiras ou falsas justificando ou
           apresentando contra-exemplos.}
           \begin{questionario}
               \qq{$\ds\int_a^b f(x)g(x)\ dx = \left(\int_a^b f(x)\ dx\right) \left(\int_a^b
                   g(x)\ dx\right)$.}
                \qq{$\ds\int_a^b xf(x)\ dx = x\int_a^b f(x)\ dx$.}
                \qq{$\ds\int_a^b f'(x)\ dx = f(b)-f(a)$}
           \end{questionario}
        \q{Determine o valor de}
            \begin{questionario}
                \qq{$\ds\int_0^{\pi/2}
                \frac{d}{dx}\left[\sen\left(\frac{x}{2}\right)\cos\left(\frac{x}{3}\right)\right]\
                dx$.}
                \qq{$\ds\frac{d}{dx}\left(\int_0^{\pi/2}
                \sen\left(\frac{x}{2}\right)\cos\left(\frac{x}{3}\right)\
                dx\right)$.}
            \end{questionario}
        \q{Encontre uma primitiva para $f(x)=e^x\cos{x}$.}
        \q{Calcule a \'area delimitada pelas curvas $y=e^{\sqrt{x}}$, $y=0$,
           $x=0$ e $x=1$.}
        \q{Determine se a integral impr\'opria $\ds\int_1^{\infty}
           \frac{\ln{x}}{x}\ dx$ \'e convergente ou divergente.}
    \end{questionario}
\end{document}
