\documentclass{prova}

\usepackage{amsmath}
\usepackage{amsfonts}

\setlength{\textheight}{25.5cm}
\setlength{\topmargin}{-2.5cm}

\DeclareMathOperator{\sen}{sen}
\DeclareMathOperator{\tg}{tg}
\newcommand{\ds}{\displaystyle}

\professor{Prof.\@ Adriano Barbosa}
\disciplina{C\'alculo 2}
\avaliacao{P2}
\curso{Matem\'atica}
\data{19/10/2022}

\begin{document}
	\cabecalho{5}  % o numero 5 indica a qnt de quadros na tabela de nota

    \textbf{Todas as respostas devem ser justificadas.}

    \textbf{Escolha cinco exerc\'{\i}cios e anote suas escolhas no quadro de notas acima.}

    \begin{questionario}
        \q{Calcule a integral indefinida $\ds\int\cos\left(\sqrt{x}\right)\
           dx$.}
        \q{Encontre uma primitiva para $f(x)=\ds\frac{x^3+2x}{x^4+4x^2+3}$.}
        \q{Calcue a integral $\ds\int\frac{1}{\sqrt{t^2+9}}\ dt$.}

        (Use $\int \sec(x)\ dx = \ln|\sec(x)+\tg(x)|+c$ se achar necess\'ario.)
        \q{Calcule a integral impr\'opria $\ds\int_{-\infty}^{\infty} xe^{-x^2}\
           dx$.}
        \q{}
            \begin{questionario}
                \q{Encontre uma primitiva para $f(x)=\ds\frac{x}{\sqrt{x^2-1}}$.}
                \q{Determine o valor da integral $\ds\int_1^2 f(x)\ dx$.}
            \end{questionario}
        \q{Sejam $p(x)=ax+b$, $q(x)=x^2-cx$ e $f(x)=\ds\frac{p(x)}{q(x)}$.}
            \begin{questionario}
                \qq{Fatore $q(x)$.}
                \qq{Escreva $f(x)$ como soma de fra\c{c}\~oes parciais.}
                \qq{Calcule a integral $\ds\int f(x)\ dx$.}
            \end{questionario}
        \q{Calcule a integral indefinida $\ds\int x\ln(1+x)\ dx$.}
        \q{Encontre uma primitiva para $f(x)=\sqrt{1-4x^2}$.}

        (Use $\cos^2(x)=\ds\frac{1+\cos(2x)}{2}$ e $\sen(2x) = 2\sen(x)\cos(x)$
        se achar necess\'ario.)
    \end{questionario}
\end{document}
