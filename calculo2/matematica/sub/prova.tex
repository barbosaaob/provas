\documentclass{prova}

\usepackage{amsmath}
\usepackage{amsfonts}

\setlength{\textheight}{25cm}

\DeclareMathOperator{\sen}{sen}
\DeclareMathOperator{\tg}{tg}
\newcommand{\ds}{\displaystyle}

\professor{Prof.\@ Adriano Barbosa}
\disciplina{C\'alculo 2}
\avaliacao{PS}
\curso{Matem\'atica}
\data{26/10/2022}

\begin{document}
	\cabecalho{5}  % o numero 5 indica a qnt de quadros na tabela de nota

    \textbf{Todas as respostas devem ser justificadas.}
    \vspace{0.5cm}

    \textbf{Avalia\c{c}\~ao P1:}

    \begin{questionario}
        \q{Estime a \'area abaixo do gr\'afico de $f(x)=\sen(x)$ de $x=0$ at\'e
           $x=2\pi$ usando seis ret\^angulos aproximantes e extremos esquerdos
           dos subintervalos.}
        \q{Encontre $f$ tal que $f''(x)=3-\sen(x)$, $f(0)=1$ e $f(\pi/2)=0$.}
        \q{Derive $F(x) = \ds\int_1^{1/x} \sen^4(t)\ dt$ com rela\c{c}\~ao a
           $x$.}
        \q{Calcule a \'are da regi\~ao delimitada pelas curvas $x=1-y^2$ e
           $x=y^2-1$.}
        \q{Resolva a integral definida $\ds\int_0^{\pi/2} \sen(x)
           \cos(\cos x)\ dx$.}
    \end{questionario}
    \vspace{0.5cm}

    \textbf{Avalia\c{c}\~ao P2:}

    \begin{questionario}
        \q{Resolva a integral indefinida $\ds\int (x^2+2x)\cos x\ dx$.}
        \q{Resolva a integral pelo m\'etodo das fra\c{c}\~oes parciais $\ds\int
           \frac{3x+1}{(x+1)(x-1)}\ dx$.}
        \q{Calcule a integral $\ds\int \frac{1}{\sqrt{x^2+4}}\ dx$.}
        \q{Determine, se poss\'{\i}vel, o valor da integral $\ds\int_0^1
           \frac{5}{x^5}\ dx$.}
        \q{Determine os valores de $p$ para os quais a integral $\ds\int_1^\infty
           \frac{1}{x^p}\ dx$ \'e convergente.}
    \end{questionario}
\end{document}
