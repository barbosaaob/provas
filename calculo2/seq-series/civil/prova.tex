\documentclass{prova}

\renewcommand{\sin}{\mbox{sen}\,}
\newcommand{\ds}{\displaystyle}

\professor{Prof.\@ Adriano Barbosa}
\disciplina{C\'alculo Diferencial e Integral II}
\avaliacao{P3}
\curso{Eng. Civil}
\data{22/02/2018}

\begin{document}
	\cabecalho{5}  % o numero 5 indica a qnt de quadros na tabela de nota

	\textbf{Todas as respostas devem ser justificadas.}
	\begin{questionario}
        \q{Calcule o limite das sequ\^encias abaixo:}
            \begin{questionario}
                \qq{$\ds\lim_{n\rightarrow\infty} \frac{n^{41}+n^{40}}{4n^{42}+2n^{41}}$}
                \qq{$\ds\lim_{n\rightarrow\infty} \frac{3^n}{5^n}$}
            \end{questionario}
        \q{Identifique as s\'eries abaixo, determine se s\~ao convergentes ou
            divergentes e calcule sua soma quando poss\'{\i}vel:}
            \begin{questionario}
                \qq{$\ds 1+\frac{1}{2^{0,42}}+\frac{1}{3^{0,42}}+\frac{1}{4^{0,42}}+\cdots$}
                \qq{$\ds -4+3-\frac{9}{4}+\frac{27}{16}-\cdots$}
            \end{questionario}
        \q{Determine se a s\'erie $\ds\sum_{n=1}^\infty
            \frac{{(2n+1)}^n}{n^{2n}}$ \'e convergente ou divergente.}
        \q{Determine para quais valores de $x$ a s\'erie $\ds\sum_{n=1}^\infty
            \frac{{(-1)}^n n^2 x^n}{2^n}$ \'e convergente.}
        \q{Calcule a s\'erie de Taylor da fun\c{c}\~ao $f(x)=\ln(x)$ com $a=1$.}
            % \begin{questionario}
            %     \qq{Calcule a s\'erie de Maclaurin das fun\c{c}\~oes $\sin(x)$ e $\cos(x)$.}
            %     \qq{Sabendo que $i^2=-1$ e $e^x=\ds\sum_{n=0}^\infty \frac{x^n}{n!}$,
            %         verifique a igualdade $e^{ix}=\cos(x)+i\ \sin(x)$.}
            % \end{questionario}
	\end{questionario}
\end{document}
