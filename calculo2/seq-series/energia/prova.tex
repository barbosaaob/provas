\documentclass{prova}

\usepackage{amssymb}

\renewcommand{\sin}{\mbox{sen}\,}
\newcommand{\ds}{\displaystyle}

\professor{Prof.\@ Adriano Barbosa}
\disciplina{C\'alculo Diferencial e Integral II}
\avaliacao{P3}
\curso{Eng. de Energia}
\data{30/11/2018}

\begin{document}
	\cabecalho{5}  % o numero 5 indica a qnt de quadros na tabela de nota

	\textbf{Todas as respostas devem ser justificadas.}
	\begin{questionario}
        \q{Calcule $\ds\lim_{n\rightarrow\infty} x_n$, onde:}
            \begin{questionario}
                \qq{$x_n = \sqrt{n+1}-\sqrt{n}$}
                \qq{$x_n = \ds\int_1^n \frac{1}{x}\ dx$}
            \end{questionario}
        \q{Determine se a s\'erie $2+0,5+0,125+0,03125+\cdots$ \'e convergente e,
        se poss\'{\i}vel, calcule sua soma.}
        \q{Determine se as s\'eries abaixo s\~ao convergentes:}
            \begin{questionario}
                \qq{$\ds\sum_{n=1}^\infty \frac{n^2}{3n^2+2}$}
                \qq{$\ds\sum_{n=0}^\infty \pi^{-n}$}
            \end{questionario}
        \q{Determine para quais valores de $x\in\mathbb{R}$ as s\'eries s\~ao
        convergentes:}
            \begin{questionario}
                \qq{$\ds\sum_{n=1}^\infty n!(2x-1)^n$}
                \qq{$\ds\sum_{n=1}^\infty \frac{(x-2)^n}{n^n}$}
            \end{questionario}
        \q{Encontre a s\'erie de Taylor da fun\c{c}\~ao $f(x)=\ds\frac{1}{x}$ centrada
        em $a=-3$.}
	\end{questionario}
\end{document}
