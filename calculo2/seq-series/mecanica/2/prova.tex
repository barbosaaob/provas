\documentclass[a4paper,5pt]{amsbook}
%%%%%%%%%%%%%%%%%%%%%%%%%%%%%%%%%%%%%%%%%%%%%%%%%%%%%%%%%%%%%%%%%%%%%

\usepackage{booktabs}
\usepackage{graphicx}
% \usepackage[]{float}
\usepackage{amssymb}
% \usepackage{amsfonts}
% \usepackage[]{amsmath}
% \usepackage[]{epsfig}
% \usepackage[brazil]{babel}
% \usepackage[utf8]{inputenc}
% \usepackage{verbatim}
%\usepackage[]{pstricks}
%\usepackage[notcite,notref]{showkeys}
\usepackage{subcaption}

%%%%%%%%%%%%%%%%%%%%%%%%%%%%%%%%%%%%%%%%%%%%%%%%%%%%%%%%%%%%%%

\newcommand{\sen}{\,\mbox{sen}\,}
\newcommand{\tg}{\,\mbox{tg}\,}
\newcommand{\cosec}{\,\mbox{cosec}\,}
\newcommand{\cotg}{\,\mbox{cotg}\,}
\newcommand{\ds}{\displaystyle}

%%%%%%%%%%%%%%%%%%%%%%%%%%%%%%%%%%%%%%%%%%%%%%%%%%%%%%%%%%%%%%%%%%%%%%%%

\setlength{\textwidth}{16cm} \setlength{\topmargin}{-1cm}
\setlength{\textheight}{25cm}
\setlength{\leftmargin}{1cm} \setlength{\rightmargin}{1cm}
\setlength{\oddsidemargin}{0cm}\setlength{\evensidemargin}{0cm}

%%%%%%%%%%%%%%%%%%%%%%%%%%%%%%%%%%%%%%%%%%%%%%%%%%%%%%%%%%%%%%%%%%%%%%%%

% \renewcommand{\baselinestretch}{1.6}
% \renewcommand{\thefootnote}{\fnsymbol{footnote}}
% \renewcommand{\theequation}{\thesection.\arabic{equation}}
% \setlength{\voffset}{-50pt}
% \numberwithin{equation}{chapter}

%%%%%%%%%%%%%%%%%%%%%%%%%%%%%%%%%%%%%%%%%%%%%%%%%%%%%%%%%%%%%%%%%%%%%%%

\begin{document}
\thispagestyle{empty}
\hspace{-0.6cm}
\begin{minipage}[p]{0.14\linewidth}
	\includegraphics[scale=0.24]{ufgd.png}
\end{minipage}
\begin{minipage}[p]{0.7\linewidth}
\begin{tabular}{c}
\toprule{}
{{\bf UNIVERSIDADE FEDERAL DA GRANDE DOURADOS}}\\
{{\bf Prof.\ Adriano Barbosa}}\\

{{\bf C\'alculo 2 --- Avalia\c{c}\~ao P2}}\\

\midrule{}
Eng.\ Mec\^anica\hspace{5cm}10 de Mar\c{c}o de 2017 \\
\bottomrule{}
\end{tabular}
\vspace{-0.45cm}
%
\end{minipage}
\begin{minipage}[p]{0.15\linewidth}
\begin{flushright}
\def\arraystretch{1.2}
\begin{tabular}{|c|c|}  % chktex 44
\hline\hline  % chktex 44
1 & \hspace{1.2cm} \\
\hline  % chktex 44
2& \\
\hline  % chktex 44
3& \\
\hline  % chktex 44
4&  \\
\hline  % chktex 44
5&  \\
\hline  % chktex 44
{\small Total}&  \\
\hline\hline  % chktex 44
\end{tabular}
\end{flushright}
\end{minipage}

%------------------------
\vspace{0.5cm}
{\bf Aluno(a):}\dotfill{}  % chktex 36
%----------------------------

\vspace{0.2cm}
%%%%%%%%%%%%%%%%%%%%%%%%%%%%%%%%   formulario  inicio  %%%%%%%%%%%%%%%%%%%%%%%%%%%%%%%%
\begin{enumerate}
	\vspace{0.5cm}
	\item Calcule o limite das sequ\^encias abaixo:
		\begin{enumerate}
			\vspace{0.3cm}
			\item $\ds\left\{\frac{(n+1)(n+2)}{2n^2}\right\}_{n=1}^\infty$
			\vspace{0.3cm}
			\item $\ds \left\{\sqrt{n^2+3n} - n\right\}_{n=1}^\infty$
		\end{enumerate}

	\vspace{0.5cm}
	\item Mostre que a s\'erie telesc\'opica, $\ds \sum_{n=1}^\infty
		\frac{1}{n(n+1)}$, \'e convergente e calcule sua soma.

		\noindent{}[Dica: escreva $\ds\frac{1}{n(n+1)}$ como soma de
			fra\c{c}\~oes parciais e em seguida escreva as somas parciais.]

	\vspace{0.5cm}
	\item Determine se a s\'erie abaixo \'e convergente ou divergente
		\[\sum_{k=1}^\infty \left(\frac{1}{2^k} - \frac{1}{2^{k+1}}\right)\]

	\vspace{0.5cm}
	\item Calcule o intervalo de converg\^encia da s\'erie abaixo
		\[\sum_{n=1}^\infty (-1)^n \frac{n^2 x^n}{2^n}\]

	\vspace{0.5cm}
	\item Calcule a s\'erie de Maclaurin da fun\c{c}\~ao $\sen(x)$ \textbf{ou} da fun\c{c}\~ao
		$\cos(x)$.
\end{enumerate}

\vspace{1cm}
\begin{flushright}
	\textit{Boa Prova!}
\end{flushright}

\end{document}
