\documentclass{article}

\usepackage[inline]{enumitem}
\usepackage{graphicx}
\usepackage{amssymb}

\newcommand{\ds}{\displaystyle}
\newcommand{\sen}{\mbox{sen}}
\newcommand{\tg}{\mbox{tg}}
\newcommand{\cosec}{\mbox{cosec}}
\newcommand{\cotg}{\mbox{cotg}}

\begin{document}
\noindent{}\rule{\textwidth}{0.4pt}
\begin{center}
	C\'alculo 2\\
	Avalia\c{c}\~ao P2 --- 03/05/2016 \\
	Engenharia Mec\^anica \\
	\vspace{0.2cm}
	% Prof. Adriano Barbosa
\end{center}
Nome: \\
\noindent{}\rule{\textwidth}{0.4pt}

\begin{enumerate}
%%%%%%%%%%%%%%%%%%%%%%%%%%%%%%%%%%%%%%%%%%%%%
\item Calcule $\displaystyle\lim_{n\rightarrow\infty}\ x_n$, com $x_n$ igual a:
	\begin{enumerate}
		\item $\ds\frac{2+n^3}{1+2n^3}$
		\item $\ds\frac{n\ \sen(n)}{n^2+1}$
	\end{enumerate}

%%%%%%%%%%%%%%%%%%%%%%%%%%%%%%%%%%%%%%%%%%%%%
\item Determine se as s\'eries s\~ao convergentes ou divergentes:
	\begin{enumerate}
		\item $\ds\sum_{n=1}^\infty \frac{n^3}{5^n}$
		\item $\ds\sum_{n=1}^\infty \frac{n^{2n}}{(1+2n^2)^n}$
	\end{enumerate}

%%%%%%%%%%%%%%%%%%%%%%%%%%%%%%%%%%%%%%%%%%%%%
\item Calcule a soma da s\'erie $\ds\sum_{n=1}^\infty \frac{1}{n(n+1)}$.

%%%%%%%%%%%%%%%%%%%%%%%%%%%%%%%%%%%%%%%%%%%%%
\item Encontre o raio e o intervalo de converg\^encia da s\'erie
$\ds\sum_{n=1}^\infty \frac{(x+2)^n}{n^n}$.

%%%%%%%%%%%%%%%%%%%%%%%%%%%%%%%%%%%%%%%%%%%%%
\item Encontre a s\'erie de Maclaurin da fun\c{c}\~ao $f(x) = \sen(x)$.
	
\end{enumerate}
\end{document}
