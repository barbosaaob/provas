\documentclass{article}

\usepackage[inline]{enumitem}
\usepackage{graphicx}
\usepackage{amssymb}

\newcommand{\ds}{\displaystyle}
\newcommand{\sen}{\mbox{sen}}
\newcommand{\tg}{\mbox{tg}}
\newcommand{\cosec}{\mbox{cosec}}
\newcommand{\cotg}{\mbox{cotg}}

\begin{document}
\noindent{}\rule{\textwidth}{0.4pt}
\begin{center}
	C\'alculo 2\\
	Avalia\c{c}\~ao PS --- 12/05/2016 \\
	Engenharia de Alimentos \\
	\vspace{0.2cm}
	% Prof. Adriano Barbosa
\end{center}
Nome: \\
Avalia\c{c}\~ao respondida: \\
\noindent{}\rule{\textwidth}{0.4pt}

\begin{center}
VOC\^E DEVE RESPONDER APENAS AS QUEST\~OES REFERENTES A SUA MENOR NOTA OU APENAS AS QUEST\~OES MARCADAS COM (*)!\@
\end{center}

\noindent{}\rule{\textwidth}{0.4pt}

{\bf Avalia\c{c}\~ao P1:}
\begin{enumerate}
%%%%%%%%%%%%%%%%%%%%%%%%%%%%%%%%%%%%%%%%%%%%%
\item Calcule a integral $\displaystyle\int e^x \cos(x)\ dx$.

%%%%%%%%%%%%%%%%%%%%%%%%%%%%%%%%%%%%%%%%%%%%%
\item
	\begin{enumerate}
		\item Calcule a integral indefinida $\displaystyle\int \frac{\ln(x)}{x}\ dx$.
		\item Calcule a integral impr\'opria $\displaystyle\int_1^\infty \frac{\ln(x)}{x}\ dx$.
	\end{enumerate}

%%%%%%%%%%%%%%%%%%%%%%%%%%%%%%%%%%%%%%%%%%%%%
\item Utilize substitui\c{c}\~ao trigonom\'etrica para calcular a integral $\displaystyle\int_0^{1/2} \sqrt{1-4x^2}\ dx$.

%%%%%%%%%%%%%%%%%%%%%%%%%%%%%%%%%%%%%%%%%%%%%
\item (*) Utilize uma substitui\c{c}\~ao e ent\~ao utilize integra\c{c}\~ao por partes para calcular a integral $\displaystyle\int x\ln(x+1)\ dx$.

%%%%%%%%%%%%%%%%%%%%%%%%%%%%%%%%%%%%%%%%%%%%%
\item Calcule $\ds\int\frac{5x+1}{(2x+1)(x-1)}\ dx$.

\end{enumerate}
\noindent{}\rule{\textwidth}{0.4pt}

{\bf Avalia\c{c}\~ao P2:}
\begin{enumerate}
%%%%%%%%%%%%%%%%%%%%%%%%%%%%%%%%%%%%%%%%%%%%%
\item (*) Calcule $\displaystyle\lim_{n\rightarrow\infty}\ x_n$, com $x_n$ igual a:
	\begin{enumerate}
		\item $\ds\frac{n}{n^2+1}$
		\item $\ds\frac{n\ \sen(n)}{n^2+1}$
	\end{enumerate}

%%%%%%%%%%%%%%%%%%%%%%%%%%%%%%%%%%%%%%%%%%%%%
\item Escreva o n\'umero $1,53424242\ldots$ como uma fra\c{c}\~ao.

%%%%%%%%%%%%%%%%%%%%%%%%%%%%%%%%%%%%%%%%%%%%%
\item Calcule a soma da s\'erie $\ds\sum_{n=1}^\infty \left(\frac{1}{e^n} + \frac{1}{n(n+1)} \right)$.

%%%%%%%%%%%%%%%%%%%%%%%%%%%%%%%%%%%%%%%%%%%%%
\item Determine se as s\'eries s\~ao convergentes ou divergentes:
	\begin{enumerate}
		\item $\ds\sum_{n=1}^\infty \frac{n^{2n}}{(1+2n^2)^n}$
		\item $\ds\sum_{n=1}^\infty \frac{(-1)^n\pi^n}{3^{2n}(2n)!}$
	\end{enumerate}

%%%%%%%%%%%%%%%%%%%%%%%%%%%%%%%%%%%%%%%%%%%%%
\item (*) Encontre o raio e o intervalo de converg\^encia da s\'erie
$\ds\sum_{n=1}^\infty \frac{(x+2)^n}{n4^n}$.

[Utilize o fato de que a s\'erie $\sum\frac{(-1)^n}{n}$ \'e convergente.]

%%%%%%%%%%%%%%%%%%%%%%%%%%%%%%%%%%%%%%%%%%%%%
% \item Encontre a s\'erie de Maclaurin da fun\c{c}\~ao $f(x) = e^x$.
\end{enumerate}
\noindent{}\rule{\textwidth}{0.4pt}

{\bf Avalia\c{c}\~ao P3:}
\begin{enumerate}
%%%%%%%%%%%%%%%%%%%%%%%%%%%%%%%%%%%%%%%%%%%%%
\item Classifique as equa\c{c}\~oes abaixo em lineares e separ\'aveis:
	\begin{enumerate}
		\item $\ds y - y' \sec(x) = 0$
		\item $\ds y' - \frac{y^2-y}{\sen(x)} = 0$
		\item $\ds y' = \frac{y}{x}$
	\end{enumerate}

%%%%%%%%%%%%%%%%%%%%%%%%%%%%%%%%%%%%%%%%%%%%%
\item (*) Resolva as equa\c{c}\~oes diferenciais abaixo:
	\begin{enumerate}
		\item $\ds \cos(y)y' = \cos(x)$
		\item $\ds y' = \frac{2t+1}{2y-2}$
	\end{enumerate}

%%%%%%%%%%%%%%%%%%%%%%%%%%%%%%%%%%%%%%%%%%%%%
\item Resolva a equa\c{c}\~ao diferencial $(x^2+4)y' + xy = 0$.

%%%%%%%%%%%%%%%%%%%%%%%%%%%%%%%%%%%%%%%%%%%%%
\item Resolva o problema de valor inicial $\ds y' = \frac{3x^2}{2y+\cos(y)}$, $y(0) = \pi$.

%%%%%%%%%%%%%%%%%%%%%%%%%%%%%%%%%%%%%%%%%%%%%
\item (*) Resolva o problema de valor inicial $xy'-y=x^2$, $y(1)=-1$, $x>0$.

\end{enumerate}
\end{document}
