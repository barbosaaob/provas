\documentclass[a4paper,5pt]{amsbook}
%%%%%%%%%%%%%%%%%%%%%%%%%%%%%%%%%%%%%%%%%%%%%%%%%%%%%%%%%%%%%%%%%%%%%

\usepackage{booktabs}
% \usepackage{graphics}
% \usepackage[]{float}
% \usepackage{amssymb}
% \usepackage{amsfonts}
% \usepackage[]{amsmath}
% \usepackage[]{epsfig}
% \usepackage[brazil]{babel}
% \usepackage[utf8]{inputenc}
% \usepackage{verbatim}
%\usepackage[]{pstricks}
%\usepackage[notcite,notref]{showkeys}

%%%%%%%%%%%%%%%%%%%%%%%%%%%%%%%%%%%%%%%%%%%%%%%%%%%%%%%%%%%%%%

\newcommand{\sen}{\text{sen}}
\newcommand{\ds}{\displaystyle}

%%%%%%%%%%%%%%%%%%%%%%%%%%%%%%%%%%%%%%%%%%%%%%%%%%%%%%%%%%%%%%%%%%%%%%%%

\setlength{\textwidth}{18cm} \setlength{\textheight}{26cm}
\setlength{\topmargin}{-2cm}
% \setlength{\leftmargin}{-5cm} \setlength{\rightmargin}{0cm}
\setlength{\oddsidemargin}{-1.3cm}\setlength{\evensidemargin}{0cm}

%%%%%%%%%%%%%%%%%%%%%%%%%%%%%%%%%%%%%%%%%%%%%%%%%%%%%%%%%%%%%%%%%%%%%%%%

% \renewcommand{\baselinestretch}{1.6}
% \renewcommand{\thefootnote}{\fnsymbol{footnote}}
% \renewcommand{\theequation}{\thesection.\arabic{equation}}
% \setlength{\voffset}{-50pt}
% \numberwithin{equation}{chapter}

%%%%%%%%%%%%%%%%%%%%%%%%%%%%%%%%%%%%%%%%%%%%%%%%%%%%%%%%%%%%%%%%%%%%%%%

\begin{document}
\thispagestyle{empty}
\begin{minipage}[b]{0.45\linewidth}
\begin{tabular}{c}
\toprule{}
{{\bf UNIVERSIDADE FEDERAL DA GRANDE DOURADOS}}\\
{{\bf Prof.\ Adriano Barbosa}}\\

{{\bf PS --- C\'alculo II}}\\

\midrule{}
Eng.\ de Computa\c{c}\~ao\hspace{5cm}5 de Outubro de 2016 \\
\bottomrule{}
\end{tabular}
%
\end{minipage} \hfill
\begin{minipage}[b]{0.58\linewidth}
\begin{flushright}
\def\arraystretch{1.2}
\begin{tabular}{|c|c|}  % chktex 44
\hline\hline  % chktex 44
1 & \hspace{1.2cm} \\
\hline  % chktex 44
2& \\
\hline  % chktex 44
3& \\
\hline  % chktex 44
4&  \\
\hline  % chktex 44
5&  \\
\hline  % chktex 44
{\small Total}&  \\
\hline\hline  % chktex 44
\end{tabular}
\end{flushright}
\end{minipage} \hfill

%------------------------
\vspace{0.3cm}
{\bf Aluno(a):}\dotfill{}  % chktex 36

\vspace{0.3cm}
{\bf Avalia\c{c}\~ao respondida:}\dotfill{}
%----------------------------

\noindent{}\rule{\textwidth}{0.4pt}
\begin{center}
	\textbf{Voc\^e deve responder apenas as quest\~oes referentes a sua menor nota\@!}
\end{center}
\noindent{}\rule{\textwidth}{0.4pt}

\vspace{0.2cm}
%%%%%%%%%%%%%%%%%%%%%%%%%%%%%%%%   formulario  inicio  %%%%%%%%%%%%%%%%%%%%%%%%%%%%%%%%
\textbf{Avalia\c{c}\~ao P1:}
\begin{enumerate}
\item Calcule a integral $\displaystyle\int e^x \cos(x)\ dx$.

\item Calcule a integral indefinida $\displaystyle\int \frac{\ln(x)}{x}\ dx$.

\item Calcule a integral impr\'opria $\displaystyle\int_0^1 \frac{\ln(x)}{x}\ dx$.

\item Utilize substitui\c{c}\~ao trigonom\'etrica para calcular a integral
	$\displaystyle\int \frac{1}{x^2 \sqrt{x^2+4}}\ dx$.

\item Calcule a integral $\ds\int \frac{x}{(x+4)(2x-1)}\ dx$.

\end{enumerate}

\textbf{Avalia\c{c}\~ao P2:}
\begin{enumerate}
\item Calcule $\displaystyle\lim_{n\rightarrow\infty}\ x_n$, com $x_n$ igual a:
	\begin{enumerate}
		\item $\ds\frac{n}{n^2+1}$
		\item $\ds\frac{n\ \sen(n)}{n^2+1}$
	\end{enumerate}

\item Determine se a s\'erie abaixo \'e convergente ou divergente e calcule a soma
	caso seja convergente:
$$\ds3-4+\frac{16}{3}-\frac{64}{9}+\cdots$$

\item Determine se as s\'eries s\~ao convergentes ou divergentes
	\begin{enumerate}
		\item $\ds\sum_{k=1}^\infty k^2 e^{-k}$
		\item $\ds\sum_{n=1}^\infty \frac{\pi}{n^{{\frac{1}{\pi}}}}$
	\end{enumerate}

\item Encontre o intervalo de converg\^encia da s\'erie: $\ds \sum_{n=1}^\infty
	\frac{x^n}{n(n+1)}$.

\item Encontre a s\'erie de Maclaurin da fun\c{c}\~ao $f(x) = \ln(x)$.
\end{enumerate}

\textbf{Avalia\c{c}\~ao P3:}
\begin{enumerate}
\item Classifique as equa\c{c}\~oes abaixo em lineares e separ\'aveis justificando:
	\begin{enumerate}
		\item $\ds y - y' \sec(x) = 0$
		\item $\ds y' - \frac{y^2-y}{\sen(x)} = 0$
		\item $\ds y' = \frac{y}{x}$
	\end{enumerate}

\item Resolva as equa\c{c}\~oes diferenciais abaixo:
	\begin{enumerate}
		\item $(1+x)y' = y$
		\item $(x^2+4)y' + xy = 0$
	\end{enumerate}

\item Resolva o problema de valor inicial $\ds y' = \frac{3x^2}{2y+\cos(y)}$,
	$y(0) = \pi$.

\item Resolva a equa\c{c}\~ao $y''-6y'+25y=0$.

\item Encontre a solu\c{c}\~ao geral da EDO $y''+y=\sec^2(x)$.

\end{enumerate}

\begin{flushright}
	\textit{Boa Prova!}
\end{flushright}

\end{document}
