\documentclass{prova}

\usepackage{amssymb}

\renewcommand{\sin}{\mbox{sen}\,}
\newcommand{\ds}{\displaystyle}

\professor{Prof.\@ Adriano Barbosa}
\disciplina{C\'alculo Diferencial e Integral II}
\avaliacao{PS}
\curso{Eng. de Energia}
\data{07/12/2018}

\begin{document}
	\cabecalho{5}  % o numero 5 indica a qnt de quadros na tabela de nota

	\textbf{Todas as respostas devem ser justificadas.}

    \vspace{0.5cm}
    \textbf{Avalia\c{c}\~ao P1:}
	\begin{questionario}
        \q{Resolva a integral indefinida $\ds\int e^x \cos{x}\ dx$.}
        \q{Resolva a integral definida $\ds\int_0^{\pi/2} \sin{x}\ e^{\cos{x}}\
        dx$.}
        \q{Calcule a integral $\ds\int \frac{x+2}{x^2-9}\ dx$.}
        \q{Determine se a integral impr\'opria $\ds\int_1^\infty
        \frac{\ln{x}}{x}\ dx$ \'e convergente ou divergente.}
        \q{Resolva a integral definida $\ds\int_0^\pi x+x\cos{x}\ dx$.}
	\end{questionario}

    \vspace{0.5cm}
    \textbf{Avalia\c{c}\~ao P2:}
    \begin{questionario}
        \q{Resolva a equa\c{c}\~ao diferencial $y'=xe^{-\cos{x}}+y\sin{x}$.}
        \q{Resolva o problema de valor inicial $y'=3x^2e^y$, $y(0)=1$.}
        \q{Aplique a mudan\c{c}a de vari\'aveis $u=y''$ e resolva a equa\c{c}\~ao
        diferencial $y^{(4)}-y''=0$.}
        \q{Resolva as equa\c{c}\~oes lineares de segunda ordem:}
            \begin{questionario}
                \qq{$y''-2y'-3y=0$}
                \qq{$y''-2y'-3y=x+2$}
            \end{questionario}
        \q{Verifique se a fun\c{c}\~ao $y(x)=c_1e^x + c_2e^{-x}$ \'e solu\c{c}\~ao da equa\c{c}\~ao
        diferencial $y'''-y'=0$.}
    \end{questionario}

    \vspace{0.5cm}
    \textbf{Avalia\c{c}\~ao P3:}
    \begin{questionario}
        \q{Calcule $\ds\lim_{n\rightarrow\infty} x_n$, onde:}
            \begin{questionario}
                \qq{$x_n = \ds\frac{2+n^{2018}}{1+2n^{2019}}$}
                \qq{$x_n = \ds\frac{1}{2^n}$}
            \end{questionario}
        \q{Escreva o n\'umero $4,17326326326...$ como uma fra\c{c}\~ao.}
        \q{Determine se as s\'eries s\~ao convergentes:}
            \begin{questionario}
                \qq{$\ds\sum_{n=1}^\infty \frac{n^3}{n+1}$}
                \qq{$\ds\sum_{n=1}^\infty n^{-\pi}$}
            \end{questionario}
        \q{Determine para quais valores de $x$ a s\'erie $\ds\sum_{n=1}^\infty
        (-1)^n\frac{x^n}{n^25^n}$ \'e convergente.}
        \q{Escreva a s\'erie de Maclaurin da fun\c{c}\~ao $f(x)=\sin{x}$.}
    \end{questionario}

\end{document}
