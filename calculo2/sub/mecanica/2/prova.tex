\documentclass[a4paper,5pt]{amsbook}
%%%%%%%%%%%%%%%%%%%%%%%%%%%%%%%%%%%%%%%%%%%%%%%%%%%%%%%%%%%%%%%%%%%%%

\usepackage{booktabs}
\usepackage{graphicx}
% \usepackage[]{float}
\usepackage{amssymb}
% \usepackage{amsfonts}
% \usepackage[]{amsmath}
% \usepackage[]{epsfig}
% \usepackage[brazil]{babel}
% \usepackage[utf8]{inputenc}
% \usepackage{verbatim}
%\usepackage[]{pstricks}
%\usepackage[notcite,notref]{showkeys}
\usepackage{subcaption}
\usepackage[inline]{enumitem}

%%%%%%%%%%%%%%%%%%%%%%%%%%%%%%%%%%%%%%%%%%%%%%%%%%%%%%%%%%%%%%

\newcommand{\sen}{\,\mbox{sen}\,}
\newcommand{\tg}{\,\mbox{tg}\,}
\newcommand{\cosec}{\,\mbox{cosec}\,}
\newcommand{\cotg}{\,\mbox{cotg}\,}
\newcommand{\ds}{\displaystyle}

%%%%%%%%%%%%%%%%%%%%%%%%%%%%%%%%%%%%%%%%%%%%%%%%%%%%%%%%%%%%%%%%%%%%%%%%

\setlength{\textwidth}{16cm} %\setlength{\topmargin}{-1.5cm}
%\setlength{\textheight}{25cm}
\setlength{\leftmargin}{1.2cm} \setlength{\rightmargin}{1.2cm}
\setlength{\oddsidemargin}{0cm}\setlength{\evensidemargin}{0cm}

%%%%%%%%%%%%%%%%%%%%%%%%%%%%%%%%%%%%%%%%%%%%%%%%%%%%%%%%%%%%%%%%%%%%%%%%

% \renewcommand{\baselinestretch}{1.6}
% \renewcommand{\thefootnote}{\fnsymbol{footnote}}
% \renewcommand{\theequation}{\thesection.\arabic{equation}}
% \setlength{\voffset}{-50pt}
% \numberwithin{equation}{chapter}

%%%%%%%%%%%%%%%%%%%%%%%%%%%%%%%%%%%%%%%%%%%%%%%%%%%%%%%%%%%%%%%%%%%%%%%

\begin{document}
\pagestyle{empty}
\thispagestyle{empty}
\hspace{-0.6cm}
\begin{minipage}[p]{0.14\linewidth}
	\includegraphics[scale=0.24]{ufgd.png}
\end{minipage}
\begin{minipage}[p]{0.7\linewidth}
\begin{tabular}{c}
\toprule{}
{{\bf UNIVERSIDADE FEDERAL DA GRANDE DOURADOS}}\\
{{\bf Prof.\ Adriano Barbosa}}\\

{{\bf C\'alculo 2 --- Avalia\c{c}\~ao PS}}\\

\midrule{}
Eng. Mec\^anica\hspace{5cm}07 de Abril de 2017 \\
\bottomrule{}
\end{tabular}
\vspace{-0.45cm}
%
\end{minipage}
\begin{minipage}[p]{0.15\linewidth}
\begin{flushright}
\def\arraystretch{1.2}
\begin{tabular}{|c|c|}  % chktex 44
\hline\hline  % chktex 44
1 & \hspace{1.2cm} \\
\hline  % chktex 44
2& \\
\hline  % chktex 44
3& \\
\hline  % chktex 44
4&  \\
\hline  % chktex 44
5&  \\
\hline  % chktex 44
{\small Total}&  \\
\hline\hline  % chktex 44
\end{tabular}
\end{flushright}
\end{minipage}

%------------------------
\vspace{0.5cm}
{\bf Aluno(a):}\dotfill{} \textbf{Avalia\c{c}\~ao:}\ldots{}\ldots{}\ldots{}  % chktex 36
%----------------------------

\vspace{0.5cm}
%%%%%%%%%%%%%%%%%%%%%%%%%%%%%%%%   formulario  inicio  %%%%%%%%%%%%%%%%%%%%%%%%%%%%%%%%
\textbf{Avalia\c{c}\~ao P1:}
\begin{enumerate}
	\vspace{0.5cm}
	\item Calcule a integral indefinida $\ds\int t^2{(t^3+2)}^6\ dt$.
	\vspace{0.5cm}
	\item Calcule a integral definida $\ds\int_1^2 x^2\ln{x}\ dx$.
	\vspace{0.5cm}
	\item Resolva a integral $\ds\int \frac{\sqrt{x^2-4}}{x}\ dx$.
	\vspace{0.5cm}
	\item Resolva a integral indefinida $\ds\int \frac{10}{5x^2-2x^3}\ dx$.
	\vspace{0.5cm}
	\item Dada a integral abaixo:
		\[\int_{-1}^1 x^{-2}\ dx = \left.\frac{x^{-1}}{-1}\right|_{-1}^1 = -1 -1 = -2\]
		\begin{enumerate}
			\vspace{0.3cm}
			\item O que est\'a errado?
			\vspace{0.3cm}
			\item Calcule a integral corretamente.
		\end{enumerate}
\end{enumerate}

\vspace{0.5cm}
\textbf{Avalia\c{c}\~ao P2:}
\begin{enumerate}
	\vspace{0.5cm}
	\item Calcule o limite das sequ\^encias abaixo:
		\begin{enumerate}
			\vspace{0.3cm}
			\item $\left\{\ds\int_1^n \frac{1}{x}\ dx\right\}_n$
			\vspace{0.3cm}
			\item $\left\{\ds\frac{\sen{n}}{n}\right\}_n$
		\end{enumerate}
	\vspace{0.5cm}
	\item Determine se as s\'eries abaixo s\~ao convergentes ou divergentes:
		\begin{enumerate}
			\vspace{0.3cm}
			\item $\ds\sum_{n=1}^\infty \frac{n^2}{5n^2+4}$
			\vspace{0.3cm}
			\item $\ds\sum_{n=1}^\infty \frac{n^3}{5^n}$
		\end{enumerate}
	\vspace{0.5cm}
	\item Escreva o n\'umero $0,123123123\ldots$ como fra\c{c}\~ao.
	\vspace{0.5cm}
	\item Determine o intervalo de converg\^encia da s\'erie $\ds\sum_{n=1}^\infty \frac{{(x+2)}^n}{n4^n}$.
	\vspace{0.5cm}
	\item Calcule a s\'erie de Taylor centrada em $a = 1$ da fun\c{c}\~ao $\ds f(x) = \frac{1}{x}$.
\end{enumerate}

\vspace{0.5cm}
\textbf{Avalia\c{c}\~ao P3:}
\begin{enumerate}
	\vspace{0.5cm}
	\item Classifique as equa\c{c}\~oes diferenciais abaixo em linear e separ\'avel:
		\begin{enumerate}
			\vspace{0.3cm}
			\item $y \sen{x} = x^2 y' - x$
			\vspace{0.3cm}
			\item $y' = \ds\frac{1}{x} + \frac{1}{y}$
			\vspace{0.1cm}
			\item $\ds\frac{dv}{ds} = \frac{s+1}{sv+s}$
		\end{enumerate}
	\vspace{0.5cm}
	\item A fun\c{c}\~ao $\ds f(x) = \frac{\ln{x}}{x}$ \'e solu\c{c}\~ao da equa\c{c}\~ao $x^2 y' + xy = 1$?
	\vspace{0.5cm}
	\item Resolva o problema de valor inicial $\ds\frac{dy}{dx} = \frac{\ln{x}}{xy}$, $y(1)=2$.
	\vspace{0.5cm}
	\item Resolva o PVI $2xy' + y = 6x$, $x>0$, $y(4) = 20$.
	\vspace{0.5cm}
	\item Encontre a solu\c{c}\~ao geral da equa\c{c}\~ao diferencial $y'' = y'$. Qual a solu\c{c}\~ao que satisfaz a condi\c{c}\~ao $y(0) = 1$ e $y'(0) = 1$.
\end{enumerate}

\begin{flushright}
	\vspace{1cm}
	\textit{Boa Prova!}
\end{flushright}

\vfill{}
F\'ormulas \'uteis:

\[\begin{array}{llll}
	\vspace{0.3cm}
	\cosec(x) = \displaystyle\frac{1}{\sen(x)}, & \sec(x) = \displaystyle\frac{1}{\cos(x)}, & \cotg(x) = \displaystyle\frac{\cos(x)}{\sen(x)}, & \ds\tg(x) = \frac{\sen(x)}{\cos(x)}\\
	\vspace{0.3cm}
	\sen^2(x) + \cos^2(x) = 1, & \tg^2(x) + 1 = \sec^2(x), & 1 + \cotg^2(x) = \cosec^2(x) & \\
	\vspace{0.3cm}
	\sen^2(x) = \displaystyle\frac{1 - \cos(2x)}{2}, & \cos^2(x) = \displaystyle\frac{1 + \cos(2x)}{2} & & \\
	\multicolumn{2}{l}{\sen(x+y) = \displaystyle\sen(x)\cos(y) + \sen(y)\cos(x),} & \multicolumn{2}{l}{\cos(x+y) = \displaystyle\cos(x)\cos(y) - \sen(x)\sen(y)} \\
	& & & \\
	\multicolumn{2}{l}{\sen(x-y) = \displaystyle\sen(x)\cos(y) - \sen(y)\cos(x),} & \multicolumn{2}{l}{\cos(x-y) = \displaystyle\cos(x)\cos(y) + \sen(x)\sen(y)}
\end{array}\]

\end{document}
