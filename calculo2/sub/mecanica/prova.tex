\documentclass{article}

\usepackage[inline]{enumitem}
\usepackage{graphicx}
\usepackage{amssymb}

\newcommand{\ds}{\displaystyle}
\newcommand{\sen}{\mbox{sen}}
\newcommand{\tg}{\mbox{tg}}
\newcommand{\cosec}{\mbox{cosec}}
\newcommand{\cotg}{\mbox{cotg}}

\begin{document}
\noindent{}\rule{\textwidth}{0.4pt}
\begin{center}
	C\'alculo 2\\
	Avalia\c{c}\~ao PS --- 13/05/2016 \\
	Engenharia Mec\^anica \\
	\vspace{0.2cm}
	% Prof. Adriano Barbosa
\end{center}
Nome: \\
Avalia\c{c}\~ao respondida: \\
\noindent{}\rule{\textwidth}{0.4pt}


\begin{center}
VOC\^E DEVE RESPONDER APENAS AS QUEST\~OES REFERENTES A SUA MENOR NOTA OU APENAS AS QUEST\~OES MARCADAS COM (*)!\@
\end{center}

\noindent{}\rule{\textwidth}{0.4pt}
{\bf Avalia\c{c}\~ao P1:}
\begin{enumerate}
%%%%%%%%%%%%%%%%%%%%%%%%%%%%%%%%%%%%%%%%%%%%%
\item Calcule a integral $\displaystyle\int x^2 \ln(x)\ dx$.

%%%%%%%%%%%%%%%%%%%%%%%%%%%%%%%%%%%%%%%%%%%%%
\item
	\begin{enumerate}
		\item Calcule a integral indefinida $\displaystyle\int \frac{x^2}{\sqrt{1+x^3}}\ dx$.
		\item Calcule a integral impr\'opria $\displaystyle\int_0^\infty \frac{x^2}{\sqrt{1+x^3}}\ dx$.
	\end{enumerate}

%%%%%%%%%%%%%%%%%%%%%%%%%%%%%%%%%%%%%%%%%%%%%
\item (*) Utilize substitui\c{c}\~ao trigonom\'etrica para calcular a integral $\displaystyle\int \frac{1}{x^2 \sqrt{x^2+4}}\ dx$.

%%%%%%%%%%%%%%%%%%%%%%%%%%%%%%%%%%%%%%%%%%%%%
\item Utilize uma substitui\c{c}\~ao e ent\~ao utilize integra\c{c}\~ao por partes para calcular a integral $\displaystyle\int\cos(\sqrt{x})\ dx$.

%%%%%%%%%%%%%%%%%%%%%%%%%%%%%%%%%%%%%%%%%%%%%
\item Calcule a integral $\ds\int \frac{x}{(x+4)(2x-1)}\ dx$.

\end{enumerate}
\noindent{}\rule{\textwidth}{0.4pt}

{\bf Avalia\c{c}\~ao P2:}
\begin{enumerate}
%%%%%%%%%%%%%%%%%%%%%%%%%%%%%%%%%%%%%%%%%%%%%
\item (*) Calcule $\displaystyle\lim_{n\rightarrow\infty}\ x_n$, com $x_n$ igual a:
	\begin{enumerate}
		\item $\ds\frac{9^{n+1}}{10^n}$
		\item $\ds\frac{9^{n+1}\ \sen(n)}{10^n}$
	\end{enumerate}

%%%%%%%%%%%%%%%%%%%%%%%%%%%%%%%%%%%%%%%%%%%%%
\item Escreva o n\'umero $10,135353535\ldots$ como uma fra\c{c}\~ao.

%%%%%%%%%%%%%%%%%%%%%%%%%%%%%%%%%%%%%%%%%%%%%
\item Determine se as s\'eries s\~ao convergentes ou divergentes
	\begin{enumerate}
		\item $\ds\sum_{k=1}^\infty k^2 e^{-k}$
		\item $\ds\sum_{n=1}^\infty \frac{\pi}{n^{{\frac{1}{\pi}}}}$
	\end{enumerate}

%%%%%%%%%%%%%%%%%%%%%%%%%%%%%%%%%%%%%%%%%%%%%
\item (*) Encontre o raio e o intervalo de converg\^encia da s\'erie
$\ds\sum_{n=1}^\infty \frac{(x-2)^n}{n^n}$.

%%%%%%%%%%%%%%%%%%%%%%%%%%%%%%%%%%%%%%%%%%%%%
\item Encontre a s\'erie de Maclaurin da fun\c{c}\~ao $f(x) = e^x$.
\end{enumerate}
\noindent{}\rule{\textwidth}{0.4pt}

{\bf Avalia\c{c}\~ao P3:}
\begin{enumerate}
%%%%%%%%%%%%%%%%%%%%%%%%%%%%%%%%%%%%%%%%%%%%%
\item Classifique as equa\c{c}\~oes abaixo em lineares e separ\'aveis:
	\begin{enumerate}
		\item $\ds y'+xy=e^y$
		\item $\ds y' = xe^{-\sen(x)} - y\cos(x)$
		\item $\ds y' = -xy$
	\end{enumerate}

%%%%%%%%%%%%%%%%%%%%%%%%%%%%%%%%%%%%%%%%%%%%%
\item Resolva as equa\c{c}\~oes diferenciais abaixo:
	\begin{enumerate}
		\item $(1+x)y' = y$
		\item $y' = -4ty^2$
	\end{enumerate}

%%%%%%%%%%%%%%%%%%%%%%%%%%%%%%%%%%%%%%%%%%%%%
\item Resolva a equa\c{c}\~ao diferencial $y'=y+e^{2x}$.

%%%%%%%%%%%%%%%%%%%%%%%%%%%%%%%%%%%%%%%%%%%%%
\item (*) Resolva o problema de valor inicial $\ds y' = \frac{x+2}{y+e^y}$, $y(0)=0$.

%%%%%%%%%%%%%%%%%%%%%%%%%%%%%%%%%%%%%%%%%%%%%
\item (*) Resolva as equa\c{c}\~oes abaixo:
	\begin{enumerate}
		\item $y''-6y'+25y=0$
		\item$y''+6y'=0$
	\end{enumerate}

\end{enumerate}
\end{document}
