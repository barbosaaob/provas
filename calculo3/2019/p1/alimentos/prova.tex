\documentclass{prova}

\renewcommand{\sin}{\,\,\mbox{sen}\,}
\newcommand{\ds}{\displaystyle}

\professor{Prof.\@ Adriano Barbosa}
\disciplina{C\'alculo Diferencial e Integral III}
\avaliacao{P1}
\curso{Eng. de Alimentos}
\data{10/04/2019}

\begin{document}
	\cabecalho{5}  % o numero 5 indica a qnt de quadros na tabela de nota

	\textbf{Todas as respostas devem ser justificadas.}
    \vspace{1cm}

	\begin{questionario}
        \q{Determine o maior dom\'{\i}nio da fun\c{c}\~ao $f(x,y) =
        \ds\frac{1+x^2+y^2}{1-x^2-y^2}$ e os pontos onde ela \'e cont\'{\i}nua.}
        \q{Calcule todas as segundas derivadas da fun\c{c}\~ao $z = xe^{-2y}$.}
        \q{Dada $f(x,y) = x\sin{(x+y)}$:}
            \begin{questionario}
                \qq{Determine a equa\c{c}\~ao do plano tangente a $f$ em $P=(-1,1,0)$.}
                \qq{Determine $k$ para que o ponto $Q=(-1.1,0.9,k)$ perten\c{c}a ao
                plano tangente a $f$ em $P$.}
            \end{questionario}
        \q{Dada $f(x,y) = x^3-6xy+8y^3$:}
            \begin{questionario}
                \qq{Encontre os pontos cr\'{\i}ticos de $f$.}
                \qq{Classifique os pontos cr\'{\i}ticos de $f$ em m\'aximo local,
                m\'{\i}nimo local ou ponto de sela.}
                \qq{Determine se $f$ \'e crescente ou decrescente no ponto
                $(0,1)$ na dire\c{c}\~ao do vetor $(1,0)$.}
            \end{questionario}
        \q{Use o m\'etodo dos Multiplicadores de Lagrange para determinar as
        dimens\~oes da caixa retangular com tampa e volume 125cm$^3$ que tem a
        menor \'area de superf\'{\i}cie poss\'{\i}vel.}
	\end{questionario}
\end{document}
