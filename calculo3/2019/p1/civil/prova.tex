\documentclass{prova}

\renewcommand{\sin}{\mbox{sen}\,}
\newcommand{\ds}{\displaystyle}

\professor{Prof.\@ Adriano Barbosa}
\disciplina{C\'alculo Diferencial e Integral III}
\avaliacao{P1}
\curso{Eng. Civil}
\data{05/04/2019}

\begin{document}
	\cabecalho{5}  % o numero 5 indica a qnt de quadros na tabela de nota

	\textbf{Todas as respostas devem ser justificadas.}
    \vspace{1cm}

	\begin{questionario}
        \q{Determine o maior dom\'{\i}nio das fun\c{c}\~oes e interprete cada conjunto
        geometricamente.}
            \begin{questionario}
                \qq{$f(x,y) = \sqrt{1-x^2}-\sqrt{1-y^2}$}
                \qq{$f(x,y) = \ln(9-x^2-y^2)$}
            \end{questionario}
        \q{Calcule as derivadas parciais pedidas.}
            \begin{questionario}
                \qq{$f(x,y) = x^4y^2 - x^3y$, $\ds\frac{\partial^3 f}{\partial x
                \partial y \partial x}(x,y)$}
                \qq{$w = e^{xy^2z}$, $\ds\frac{\partial^2 w}{\partial y \partial
                x}(x,y,z)$}
            \end{questionario}
        \q{Use a aproxima\c{c}\~ao linear de $f(x,y) = \ds\frac{x}{x+y}$ em $(2,1)$
        para aproximar o valor de $f(2.1, 0.9)$.}
        \q{Dada $f(x,y) = e^{xy}$ e $P=(0,2)$:}
            \begin{questionario}
                \item Calcule a derivada direcional de $f$ no ponto $P$ na
                dire\c{c}\~ao que tem \^angulo $\pi/4$ com rela\c{c}\~ao ao eixo $x$.
                \item Determine a taxa de varia\c{c}\~ao m\'axima de $f$ no ponto
                $P$ e a dire\c{c}\~ao onde ocorre.
                \item Classifique os pontos cr\'{\i}ticos de $f$ em m\'aximo local,
                m\'{\i}nimo local ou sela.
            \end{questionario}
        \q{Deseja-se produzir uma caixa sem tampa com volume de $32000$cm$^3$.
        Utilizando o m\'etodo dos Multiplicadores de Lagrande, determine quais
        devem ser as dimens\~oes da caixa de modo que a quantidade de material 
        utilizada seja a menor poss\'{\i}vel.}
	\end{questionario}
\end{document}
