\documentclass{prova}

\renewcommand{\sin}{\,\,\mbox{sen}\,}
\newcommand{\ds}{\displaystyle}

\professor{Prof.\@ Adriano Barbosa}
\disciplina{C\'alculo Diferencial e Integral III}
\avaliacao{P1}
\curso{Eng. Mec\^anica}
\data{11/04/2019}

\begin{document}
	\cabecalho{5}  % o numero 5 indica a qnt de quadros na tabela de nota

	\textbf{Todas as respostas devem ser justificadas.}
    \vspace{1cm}

	\begin{questionario}
        \q{Determine o maior dom\'{\i}nio de $f(x,y) =
        \sqrt{4-x^2-y^2}+\sqrt{1-x^2}$ e interprete esse conjunto
        geometricamente.}
        %\q{Seja $z = y+f(x^2-y^2)$ diferenci\'avel. Mostre que $\ds
        %y\frac{\partial z}{\partial x}+x\frac{\partial z}{\partial y}=x$.}
        \q{Seja $f(x,y) = x^2+y^2-xy$, onde $x=\cos{t}$ e $y=e^{t}$. Calcule
        $\ds\frac{df}{dt}$ quando $t=0$.}
        \q{Dada $f(x,y) = 1+x\ln(xy-5)$:}
            \begin{enumerate}
                \qq{Encontre a aproxima\c{c}\~ao linear $L(x,y)$ de $f$ no ponto
                $(2,3)$.}
                \qq{Use $L(x,y)$ para aproximar o valor de
                $1+(2.1)\ln{\left((2.1)\cdot(2.9)-5\right)}$.}
            \end{enumerate}
        \q{Dada $f(x,y) = x-x^2y-y+xy^2$:}
            \begin{enumerate}
                \qq{Encontre os pontos cr\'{\i}ticos de $f$.}
                \qq{Classifique os pontos cr\'{\i}ticos de $f$ em m\'aximo local,
                m\'{\i}nimo local ou ponto de sela.}
                \qq{Sabendo que a taxa de varia\c{c}\~ao m\'axima de $f$ em $P$ ocorre
                na dire\c{c}\~ao $(1,-1)$, determine seu valor.}
            \end{enumerate}
        \q{Encontre as dimens\~oes da caixa retangular com volume m\'aximo tal que
        a soma dos comprimentos de suas arestas \'e igual a 4.}
	\end{questionario}
\end{document}
