\documentclass{prova}

\renewcommand{\sin}{\,\,\mbox{sen}\,}
\newcommand{\ds}{\displaystyle}

\professor{Prof.\@ Adriano Barbosa}
\disciplina{C\'alculo Diferencial e Integral III}
\avaliacao{P1}
\curso{Eng. Mec\^anica -- Segunda chamada}
\data{18/04/2019}

\begin{document}
	\cabecalho{5}  % o numero 5 indica a qnt de quadros na tabela de nota

	\textbf{Todas as respostas devem ser justificadas.}
    \vspace{1cm}

	\begin{questionario}
        \q{Encontre o maior dom\'{\i}nio da fun\c{c}\~ao $f(x,y) = \ln{(x+y+1)}$,
        interprete esse conjunto geometricamente e determine os pontos onde $f$
        \'e cont\'{\i}nua.}
        \q{Calcule as derivadas parciais de $f(x,y)=e^{-x}\sin{(\pi y)}$.}
        \q{Dada $f(x,y)=\ds\sqrt{x + \sqrt{y}}$:}
            \begin{questionario}
                \qq{Encontre a aproxima\c{c}\~ao linear $L(x,y)$ de $f$ em $(1,9)$.}
                \qq{Use $L(x,y)$ para aproximar o valor de $\ds\sqrt{1.01 +
                \sqrt{8.99}}$.}
            \end{questionario}
        \q{Dada $f(x,y) = x^2+xy+y^2+y$ e $P=(1,1)$:}
            \begin{questionario}
                \item{Calcule a derivada direcional de $f$ em $P$ na dire\c{c}\~ao
                que faz \^angulo $\pi/3$ com o eixo $x$.}
                \item{Determine de $f$ \'e crescente ou decrescente em $P$ na
                dire\c{c}\~ao do vetor $(1,0)$.}
                \item{Classifique os pontos cr\'{\i}ticos de $f$ em m\'{\i}nimo local,
                m\'aximo local ou sela.}
            \end{questionario}
        \q{Encontre os tr\^es n\'umeros $a$, $b$ e $c$ tais que $a+b+c=60$ e o
        produto $abc$ seja o maior poss\'{\i}vel.}
	\end{questionario}
\end{document}
