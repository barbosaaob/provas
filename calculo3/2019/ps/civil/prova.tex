\documentclass{prova}

\usepackage{amsmath}

\setlength{\textheight}{25cm}

\renewcommand{\sin}{\,\mbox{sen}\,}
\newcommand{\ds}{\displaystyle}

\professor{Prof.\@ Adriano Barbosa}
\disciplina{C\'alculo Diferencial e Integral III}
\avaliacao{PS}
\curso{Engenharia Civil}
\data{13/06/2019}

\begin{document}
	\cabecalho{5}  % o numero 5 indica a qnt de quadros na tabela de nota

    \textbf{Todas as respostas devem ser justificadas.}

    \textbf{Resolva apenas a avalia\c{c}\~ao referente a sua menor nota.}
    \vspace{0.5cm}

    {\bf Avalia\c{c}\~ao P1:}
    \begin{questionario}
        \q{Calcule as tr\^es derivadas parciais da fun\c{c}\~ao $w=\ln{(2x^2+3y-z)}$.}
        \q{Seja $w=xe^{y/z}$, onde $x=t^2$, $y=1-t$, $z=1+2t$. Calcule
        $\ds\frac{dw}{dt}$.}
        \q{Determine a taxa de varia\c{c}\~ao de $T(x,y)=e^x \sin{y}$ no ponto
        $(0,\frac{\pi}{3})$ na dire\c{c}\~ao $(8,-6)$.}
        \q{Determine, caso existam, os pontos de m\'aximo local, m\'{\i}nimo local e
        sela da fun\c{c}\~ao $F(u,v)=(1-uv)(u-v)$.}
        \q{Encontre os pontos do cone $z^2=x^2+y^2$ que est\~ao mais pr\'oximos do
        ponto $(-4,-2,0)$.}
    \end{questionario}

    \vspace{1cm}

    {\bf Avalia\c{c}\~ao P2:}
	\begin{questionario}
        \q{Calcule a integral iterada $\ds\int_0^{\pi} \int_0^1
        \int_0^{\sqrt{1-y^2}} y\sin{x}\ dz\ dy\ dx$.}
        \q{Esboce a regi\~ao cuja \'area \'e dada pela integral
        $\ds\int_{\pi/4}^{3\pi/4} \int_1^2 r\ dr\ d\theta$ e calcule-a.}
        \q{Calcule $\ds\iiint_E x\ dV$, onde $E$ \'e o tetraedro s\'olido limitado
        pelos planos $x=0$, $y=0$, $z=0$ e $x+y+z=1$.}
        \q{Determine o trabalho realizado pelo campo $F(x,y)=(x,y+2)$ ao mover
        uma part\'{\i}cula sobre a curva $r(t)=(t-\sin{t}, 1-\cos{t})$, $0\le t\le
        2\pi$.}
        \q{Calcule a integral de linha $\ds\int_C (y-\cos{y})\ dx + (x\sin{y})\
        dy$, onde $C$ \'e o c\'{\i}rculo de centro em $(3,-4)$ e raio $2$.}
	\end{questionario}
\end{document}
