\documentclass{prova}

\usepackage{amsmath}

\setlength{\textheight}{25cm}

\renewcommand{\sin}{\,\mbox{sen}\,}
\newcommand{\ds}{\displaystyle}

\professor{Prof.\@ Adriano Barbosa}
\disciplina{C\'alculo Diferencial e Integral III}
\avaliacao{PS}
\curso{Engenharia Mec\^anica}
\data{25/06/2019}

\begin{document}
	\cabecalho{5}  % o numero 5 indica a qnt de quadros na tabela de nota

    \textbf{Todas as respostas devem ser justificadas.}

    \textbf{Resolva apenas a avalia\c{c}\~ao referente a sua menor nota.}
    \vspace{0.5cm}

    {\bf Avalia\c{c}\~ao P1:}
    \begin{questionario}
	    \q{Calcule as derivadas parciais de $f(x,y)=1+x\ln{(xy-5)}$.}
	    \q{Dada $z=\sin{\theta}\cos{\phi}$, onde $\theta=st^2$,
	    $\phi=s^2t$, calcule $\ds\frac{\partial z}{\partial s}$ e
	    $\ds\frac{\partial z}{\partial t}$.}
	    \q{Determine a taxa de varia\c{c}\~ao m\'axima de $f(x,y)=4y\sqrt{x}$ no
	    ponto $(4,1)$ e a dire\c{c}\~ao em que isso ocorre.}
	    \q{Encontre os pontos de m\'aximo local, m\'{\i}nimo local e de sela de
	    $f(x,y)=xy(1-x-y)$.}
	    \q{Encontre os tr\^es n\'umeros positivos cuja soma \'e  12 e
	    cuja soma dos quadrados seja a menor poss\'{\i}vel.}
    \end{questionario}

    \vspace{1cm}

    {\bf Avalia\c{c}\~ao P2:}
	\begin{questionario}
            \q{Calcule a integral dupla $\iint_D x\ dA$, onde $D=\{(x,y)\ |\ 0\le
            x\le \pi, 0\le y\le \cos{x}\}$.}
            \q{Descreva o s\'olido cujo volume \'e dado pela integral $\ds\int_0^{\pi}
            \int_0^{\pi} \int_0^1 \rho^2 \sin{\phi}\ d\rho\ d\theta\ d\phi$ e
            determine o valor dessa integral.}
            \q{Calcule a integral $\iiint_E \sqrt{x^2+y^2}\ dV$, onde $E$ \'e a
            regi\~ao delimitada pelo paraboloide $y=x^2+z^2$ e pelo plano $y=4$.}
            \q{Dada $F(x,y)=(3+2xy, x^2-3y^2)$}
                \begin{questionario}
                    \qq{Determine se $F$ \'e conservativo. Caso positivo, determine a
                    fun\c{c}\~ao potencial de $F$.}
                    \qq{Calcule a integral $\int_C F\cdot\ dr$, onde $C$ \'e o
	    	caminho $r(t)=(te^{\sqrt{t}}, te^{t})$, $0\le t\le 1$.}
		\end{questionario}
        \q{Use o Teorema de Green para provar a f\'ormula da \'area do c\'{\i}rculo de
        raio $r$, $x^2+y^2=r^2$.

        [Lembre que: $\iint_D\ dA = \frac{1}{2}\int_C -y\ dx + x\ dy$]
        }
	\end{questionario}
\end{document}
