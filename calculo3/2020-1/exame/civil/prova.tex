\documentclass{prova}

\usepackage{amsmath}
\usepackage{amsfonts}

\setlength{\textheight}{25cm}

\renewcommand{\sin}{\,\mbox{sen}\,}
\newcommand{\ds}{\displaystyle}

\professor{Prof.\@ Adriano Barbosa}
\disciplina{C\'alculo Diferencial e Integral III}
\avaliacao{Final}
\curso{Engenharia Civil}
\data{10/06/2021}

\begin{document}
	\cabecalho{5}  % o numero 5 indica a qnt de quadros na tabela de nota

    \textbf{Todas as respostas devem ser justificadas.}

    \begin{questionario}
        %\q{Um cartão retangular de dimensões $A$ e $L$ é cortado em quatro
        %    partes como na figura abaixo. Determine onde devem ser feitos os
        %    cortes para que a soma dos quadrados das áreas dos retângulos
        %    menores seja mínima. Qual o valor da soma dos quadrados das áreas?}
        %    \begin{figure}[!h]
        %        \centering
        %        \includegraphics[width=0.4\textwidth]{retangulo.pdf}
        %    \end{figure}
        %\q{$\ds\frac{xy^2}{e^{x^2+y^2}}$}
        \q{Dados os pontos $(1,2)$, $(2,5)$, $(4,7)$ e $f(x)=ax+b$:}
            \begin{questionario}
                \qq{(1,0 pt) Encontre os valores de $a$ e $b$ tais que
                    \[(f(1)-2)^2 + (f(2)-5)^2 + (f(4)-7)^2\] seja mínimo.}
                \qq{(0,5 pt) Desenhe os pontos e a função $f$ no plano.}
                \qq{(1,0 pt) Interprete o que significa minimizar o valor de}
                    \[F(a,b) = (f(1)-2)^2 + (f(2)-5)^2 + (f(4)-7)^2\]
            \end{questionario}
        \q{(2,5 pt) Sabendo que a derivada direcional de $f$ no ponto $(3,-2,1)$ na
            direção $(2,-1,-2)$ é $-3$ e que $\|\nabla f(3,-2,1)\|=3$, calcule a
            direção de maior crescimento de $f$ no ponto $(3,-2,1)$. Quem é
            $\nabla f(3,-2,1)$?}
        \q{(2,5 pt)Sendo $f$ contínua, mostre que $\ds\int_0^1 \int_{x^2}^1 \int_0^{1-y} f(t)\ dt\ dy\
            dx = \frac{2}{3}\int_0^1 (1-t)^{3/2}f(t)\ dt$.}
        \q{(2,5 pt) Determine a curva $C$ simples, fechada, suave e orientada
            positivamente tal que\\ $\ds\int_C \left(-2y^3+3y\right)\ dx +
            \left(\frac{x^3}{3}\right)\ dy$ tenha valor mínimo.}
    \end{questionario}
\end{document}
