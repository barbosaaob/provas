\documentclass{prova}

\usepackage{amsmath}
\usepackage{amsfonts}

\setlength{\textheight}{25cm}

\renewcommand{\sin}{\,\mbox{sen}\,}
\newcommand{\ds}{\displaystyle}

\professor{Prof.\@ Adriano Barbosa}
\disciplina{C\'alculo Diferencial e Integral III}
\avaliacao{Final}
\curso{Engenharia Mecânica}
\data{08/06/2021}

\begin{document}
	\cabecalho{5}  % o numero 5 indica a qnt de quadros na tabela de nota

    \textbf{Todas as respostas devem ser justificadas.}

    \begin{questionario}
        \q{Mostre que existe um ponto em comum a todos os planos tangente da
            superfície $z=y\ds f\left(\frac{x}{y}\right)$, onde $f$ é uma função diferenciável de uma
            variável.}
        \q{Sejam $\nabla f(a,b) = (1,-2)$ e $\ds\frac{\partial f}{\partial
            u}(a,b) = -2$, encontre $u$.}
        \q{Sendo $f$ contínua, mostre que}
            \[\int_0^a \int_0^y \int_0^z f(x)\ dx\ dz\ dy = \frac{1}{2}\int_0^a
                f(x) (x-a)^2\ dx\]
        \q{Determine a curva $C$ simples, fechada, suave e orientada
            positivamente tal que\\ $\ds\int_C \left(-3y^3\right)\ dx +
            \left[4x\left(-\frac{x^2}{3}+1\right)\right]\ dy$ tenha valor máximo.}
    \end{questionario}
\end{document}
