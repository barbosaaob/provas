\documentclass{prova}

\usepackage{amsmath}
\usepackage{amsfonts}

\setlength{\textheight}{25cm}

\renewcommand{\sin}{\,\mbox{sen}\,}
\newcommand{\ds}{\displaystyle}

\professor{Prof.\@ Adriano Barbosa}
\disciplina{C\'alculo Diferencial e Integral III}
\avaliacao{P1}
\curso{Engenharia Mecânica}
\data{13/04/2021}

\begin{document}
	\cabecalho{5}  % o numero 5 indica a qnt de quadros na tabela de nota

    \textbf{Todas as respostas devem ser justificadas.}

    \begin{questionario}
        \q{}
            \begin{questionario}
                \qq{Encontre o maior domínio da função $f(x,y) =
                    \ds\frac{\sqrt{4-x^2}}{y^2+3}$ e descreva-o com suas palavras.}
                \qq{Encontre a maior região onde a função $f(x,y,z) =
                    \sin{\left(\sqrt{x^2+y^2+z^2}\right)}$ é contínua e
                    descreva-a com suas palavras.}
            \end{questionario}
        \q{Seja $z=f(2x+y)$ diferenciável. Mostre que}
            \[\frac{\partial z}{\partial x} - 2\frac{\partial z}{\partial y} = 0.\]
        \q{Mostre que a aproximação linear da função $f(u,v) = u^\alpha
           v^\beta$ em $(1,1)$ é}
            \[L(u,v) = 1+\alpha(u-1)+\beta(v-1).\]
        \q{Seja $f(x,y) = x^2-y^2$:}
            \begin{questionario}
                \qq{Calcule a direção de maior crescimento de $f$ em $(1,1)$.}
                \qq{Encontre os valores máximo e mínimo de $f$ restrita ao
                    círculo $x^2+y^2=25$.}
            \end{questionario}
        \q{Seja $\ds f(x,y) = x^2+y^2+\frac{2}{xy}$:}
            \begin{questionario}
                \qq{Calcule os pontos críticos de $f$.}
                \qq{Classifique os pontos críticos de $f$ em ponto de máximo,
                    de mínimo ou de sela.}
            \end{questionario}
    \end{questionario}
\end{document}
