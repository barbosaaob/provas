\documentclass{prova}

\usepackage{amsmath}
\usepackage{amsfonts}
\usepackage{gensymb}

\setlength{\textheight}{25cm}

\renewcommand{\sin}{\,\mbox{sen}\,}
\newcommand{\ds}{\displaystyle}

\professor{Prof.\@ Adriano Barbosa}
\disciplina{C\'alculo Diferencial e Integral III}
\avaliacao{PS}
\curso{Engenharia Civil}
\data{28/05/2021}

\begin{document}
	\cabecalho{5}  % o numero 5 indica a qnt de quadros na tabela de nota

    \textbf{Todas as respostas devem ser justificadas.}

    \textbf{Resolva apenas a avalia\c{c}\~ao referente a sua menor nota.}
    \vspace{0.5cm}

    {\bf Avalia\c{c}\~ao P1:}
    \begin{questionario}
        \q{Determine se as afirmações são verdadeiras ou falsas e justifique
            sua resposta.}
            \begin{questionario}
                \qq{$\ds\lim_{(x,y)\rightarrow (a,b)} f(x,y) = f(a,b)$, para
                    todo $(a,b)\in\mathbb{R}^2$.}
                \qq{Se $\ds\frac{\partial f}{\partial x}(a,b) = \frac{\partial
                    f}{\partial y}(a,b) = 0$, então $(a,b)$ é um ponto de
                    máximo ou de mínimo.}
            \end{questionario}
        \q{A função $f(x,y) = \ln{(e^x + e^y)}$ é solução da equação diferencial
            $\ds\frac{\partial^2 f}{\partial x^2} \frac{\partial^2 f}{\partial y^2}
                - 2\left(\frac{\partial^2 f}{\partial y \partial x}\right)^2 = 0$?}
        \q{A altitude $z$ de uma montanha é dada por $z=2000-0,02x^2-0,04y^2$,
            onde $x$ e $y$ estão no plano $xy$ a nível do mar com o eixo $x$
            apontando para leste e o eixo $y$ para norte. Se um alpinista está no
            ponto $(-10, 5, 1997)$ e escala para oeste, ele irá subir ou descer a
            montanha?}
        \q{A energia consumida num resistor é dada por $\ds P = \frac{C^2}{R}$.
            Calcule a aproximação linear para a energia consumida quando $C$ decresce
            $0,002$V, $R$ aumenta $0,01\ohm$, $C=100$V e $R=10\ohm$.}
    \end{questionario}

    \vspace{1cm}

    {\bf Avalia\c{c}\~ao P2:}
	\begin{questionario}
        \q{Encontre os máximos locais, mínimos locais e pontos de sela de
            $f(x,y) = e^{x/2}(y^2+x)$.}
        \q{Calcule a integral invertendo a ordem de integração}
            \[\int_0^{\sqrt{\pi}} \int_y^{\sqrt{\pi}} \cos{(x^2)}\ dxdy\]
        \q{Calcule a integral utilizando coordenadas esféricas}
            \[\int_{-2}^2 \int_0^{\sqrt{4-y^2}}
              \int_{-\sqrt{4-x^2-y^2}}^{\sqrt{4-x^2-y^2}}\ \ y\sqrt{x^2+y^2+z^2}\ dz dx dy\]
        \q{Sejam $F(x,y) = (xy^2, -x^2y)$ e $C$ uma curva que consiste no arco
            de parábola $y=x^2$ de $(-1,1)$ a $(1,1)$ e do segmento de reta
            ligando $(1,1)$ a $(-1,1)$. Calcule o trabalho realizado pelo campo
            $F$ ao mover uma partícula ao longo do caminho $C$.}
	\end{questionario}
\end{document}
