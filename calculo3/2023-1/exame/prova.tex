\documentclass{prova}

\usepackage{amsmath}
\usepackage{amsfonts}

\setlength{\textheight}{25cm}

\renewcommand{\sin}{\,\mbox{sen}\,}
\newcommand{\ds}{\displaystyle}

\professor{Prof.\@ Adriano Barbosa}
\disciplina{C\'alculo de V\'arias Vari\'aveis}
\avaliacao{Final}
\curso{Matem\'atica}
\data{13/08/2023}

\begin{document}
	\cabecalho{5}  % o numero 5 indica a qnt de quadros na tabela de nota

    \textbf{Todas as respostas devem ser justificadas.}

    \begin{questionario}
        \q{Calcule as derivadas parciais das fun\c{c}\~oes abaixo:}
            \begin{questionario}
                \qq{$f(x,y,z) = \sqrt{x^2+y^2+z^2}$}
                \qq{$f(x,y,z) = \ds e^{y\cos{x}} \ln(yz)$}
            \end{questionario}
        \q{Seja $f(x,y) = xy-x^3-y^2$. Encontre seus pontos cr\'{\i}ticos e
           classifique-os em ponto de m\'aximo, de m\'{\i}nimo ou se sela.}
        \q{Um morro possui sua forma definida pelo gr\'afico de $f(x,y) = 1000 -
           0,005x^2 - 0,01y^2$.}
            \begin{questionario}
                \qq{Se um alpinista est\'a no ponto com coordenadas $x=60$ e
                    $y=40$, que dire\c{c}\~ao e sentido ele deve tomar para subir
                    pela parte mais \'{\i}ngrime do morro? Qual a taxa de varia\c{c}\~ao
                    da altura neste ponto e nesta dire\c{c}\~ao?}
                \qq{Se o alpinista se mover na dire\c{c}\~ao do vetor $v=(-1,-2)$,
                    ele estar\'a subindo ou descendo? Qual a taxa de varia\c{c}\~ao?}
            \end{questionario}
        \q{Utilizando uma integral dupla, calcule a \'area da regi\~ao determinada
           por}
           \[R=\{(x,y)\in\mathbb{R}^2\ |\ xy\le 2, x\le y\le x+1, x\ge 0\}\]
        \q{Calcule $\ds\iiint_E xyz\ dV$, onde $E$ fica entre as esferas $\rho=3$
           e $\rho=6$, com $z\le 0$.}
    \end{questionario}
\end{document}
