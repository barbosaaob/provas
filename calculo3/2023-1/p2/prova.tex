\documentclass{prova}

\usepackage{amsmath}
\usepackage{amsfonts}

\setlength{\textheight}{25cm}

\renewcommand{\sin}{\,\mbox{sen}\,}
\newcommand{\ds}{\displaystyle}

\professor{Prof.\@ Adriano Barbosa}
\disciplina{C\'alculo de V\'arias Vari\'aveis}
\avaliacao{P2}
\curso{Matem\'atica}
\data{30/08/2023}

\begin{document}
	\cabecalho{5}  % o numero 5 indica a qnt de quadros na tabela de nota

    \textbf{Todas as respostas devem ser justificadas.}

    \begin{questionario}
        \q{Determine o volume do s\'olido definido abaixo da superf\'{\i}cie $f(x,y) =
           y^3e^{2x}$ e acima do ret\^angulo $R=[0,2]\times [0,4]$.}
        \q{Calcule a integral $\ds\int_0^4 \int_{\sqrt{x}}^2 \frac{1}{y^3+1}\
           dy\ dx$ invertendo a ordem de integra\c{c}\~ao.}
        \q{Calcule a integral $\ds\int_0^1 \int_y^{\sqrt{2-y^2}} x-y\ dx\ dy$.}
        \q{Calcule a integral $\ds\iiint_E \sqrt{x^2+y^2}\ dV$, onde $E$ \'e a
           regi\~ao que est\'a dentro do cilindro $x^2+y^2=9$ e entre os planos $z=-1$
           e $z=3$.}
        \q{Descreva o s\'olido cujo volume \'e dado pela integral
           $\ds\int_0^{\pi/6} \int_0^{\pi/2} \int_0^3 \rho^2\sin\phi\ d\rho\
           d\theta\ d\phi$ e calcule seu volume.}
    \end{questionario}
\end{document}
