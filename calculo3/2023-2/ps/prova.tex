\documentclass{prova}

\usepackage{amsmath}
\usepackage{amsfonts}

\setlength{\textheight}{25cm}

\renewcommand{\sin}{\,\mbox{sen}\,}
\newcommand{\ds}{\displaystyle}

\professor{Prof.\@ Adriano Barbosa}
\disciplina{C\'alculo de V\'arias Vari\'aveis}
\avaliacao{PS}
\curso{Matem\'atica}
\data{28/02/2024}

\begin{document}
	\cabecalho{5}  % o numero 5 indica a qnt de quadros na tabela de nota

    \textbf{Todas as respostas devem ser justificadas.}

    \begin{questionario}
        \q{}
            \begin{questionario}
                \qq{(1 pt) Determine e esboce o maior dom\'{\i}nio de $f(x,y) = \sqrt{4-x^2-y^2} +
                    \sqrt{1-x^2}$.}
                \qq{(1 pt) Calcule o limite $\ds\lim_{(x,y)\rightarrow (0,0)}
                    \frac{xy}{2x^2+y^2}$.}
            \end{questionario}
        \q{}
            \begin{questionario}
                \qq{(1 pt) Calcule as derivadas parciais de $F(x,y) = \ds\int_x^y
                    \sin(\cos(t))\ dt$.}
                \qq{(1 pt) Seja $z = \sin(\theta)\cos(\phi)$, onde
                    $\theta=st^2$ e $\phi=s^2t$.  Use a regra da cadeia para
                    calcular $\ds\frac{\partial z}{\partial s}$ e
                    $\ds\frac{\partial z}{\partial t}$.}
            \end{questionario}
        \q{(2 pts) Encontre a taxa de varia\c{c}\~ao m\'axima de $f(x,y) = x^2y+\sqrt{y}$ no
           ponto $(2,1)$. Em que dire\c{c}\~ao ela ocorre?}
        \q{(2 pts) Encontre os pontos de m\'aximo e m\'{\i}nimo locais e pontos de sela de
           $f(x,y) = 3xy - x^2y -xy^2$.}
        \q{(2 pts) Encontre os pontos da superf\'{\i}cie $xy^2z^3 = 2$ que s\~ao mais pr\'oximos
           da origem.}
    \end{questionario}
\end{document}
