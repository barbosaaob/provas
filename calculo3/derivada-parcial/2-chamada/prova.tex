\documentclass[a4paper,5pt]{amsbook}
%%%%%%%%%%%%%%%%%%%%%%%%%%%%%%%%%%%%%%%%%%%%%%%%%%%%%%%%%%%%%%%%%%%%%

\usepackage{booktabs}
% \usepackage{graphics}
% \usepackage[]{float}
% \usepackage{amssymb}
% \usepackage{amsfonts}
% \usepackage[]{amsmath}
% \usepackage[]{epsfig}
% \usepackage[brazil]{babel}
% \usepackage[utf8]{inputenc}
% \usepackage{verbatim}
%\usepackage[]{pstricks}
%\usepackage[notcite,notref]{showkeys}

%%%%%%%%%%%%%%%%%%%%%%%%%%%%%%%%%%%%%%%%%%%%%%%%%%%%%%%%%%%%%%

\newcommand{\sen}{\text{\ sen}}
\newcommand{\ds}{\displaystyle}

%%%%%%%%%%%%%%%%%%%%%%%%%%%%%%%%%%%%%%%%%%%%%%%%%%%%%%%%%%%%%%%%%%%%%%%%

\setlength{\textwidth}{16cm} %\setlength{\topmargin}{-0.1cm}
\setlength{\leftmargin}{1.2cm} \setlength{\rightmargin}{1.2cm}
\setlength{\oddsidemargin}{0cm}\setlength{\evensidemargin}{0cm}

%%%%%%%%%%%%%%%%%%%%%%%%%%%%%%%%%%%%%%%%%%%%%%%%%%%%%%%%%%%%%%%%%%%%%%%%

% \renewcommand{\baselinestretch}{1.6}
% \renewcommand{\thefootnote}{\fnsymbol{footnote}}
% \renewcommand{\theequation}{\thesection.\arabic{equation}}
% \setlength{\voffset}{-50pt}
% \numberwithin{equation}{chapter}

%%%%%%%%%%%%%%%%%%%%%%%%%%%%%%%%%%%%%%%%%%%%%%%%%%%%%%%%%%%%%%%%%%%%%%%

\begin{document}
\thispagestyle{empty}
\begin{minipage}[b]{0.45\linewidth}
\begin{tabular}{c}
\toprule{}
{{\bf UNIVERSIDADE FEDERAL DA GRANDE DOURADOS}}\\
{{\bf Prof.\ Adriano Barbosa}}\\



{{\bf C\'alculo III}}\\

\midrule{}
\hspace{8cm}26 de Agosto de 2016  \\
\bottomrule{}
\end{tabular}
%
\end{minipage} \hfill
\begin{minipage}[b]{0.58\linewidth}
\begin{flushright}
\def\arraystretch{1.2}
\begin{tabular}{|c|c|}
\hline\hline
1 & \hspace{1.2cm} \\
\hline
2& \\
\hline
3& \\
\hline
4&  \\
\hline
5&  \\
\hline
6&  \\
\hline
{\small Total}&  \\
\hline\hline
\end{tabular}
\end{flushright}
\end{minipage} \hfill
%------------------------
\vspace{0.3cm}\\
{\bf Aluno(a):}\dotfill{} \\
%----------------------------


\vspace{0.2cm}
%%%%%%%%%%%%%%%%%%%%%%%%%%%%%%%%   formulario  in\'icio  %%%%%%%%%%%%%%%%%%%%%%%%%%%%%%%%
\begin{enumerate}

\item Dada a fun\c{c}\~ao $r(t) = (2\cos(t), \sen(2t), 2t)$, calcule $r'(t)$ e $\ds\int r(t)\ dt$.
\vspace{0.5cm}

\item Calcule, se existir, $\ds\lim_{(x,y)\rightarrow(0,0)} f(x,y)$, onde 
	$$f(x,y) = \left\{
		\begin{array}{cl}
			\ds\frac{x^2-y^2}{x^2+y^2}, & \mbox{se } (x,y) \neq (0,0) \\
			0, & \mbox{se } (x,y) = (0,0)
		\end{array}\right.$$
	A fun\c{c}\~ao $f(x,y)$ \'e cont\'inua em $(0,0)$? Justifique.
\vspace{0.5cm}

\item Se $z=y+f(x^2-y^2)$, utilize a regra da cadeia para mostrar que $y\ds\frac{\partial z}{\partial x} + x\ds\frac{\partial z}{\partial y} = x$.
\vspace{0.5cm}

\item Utilizando o m\'etodo dos multiplicadores de Lagrange, encontre o ponto da esfera $x^2+y^2+z^2=4$ que \'e mais pr\'oximo do ponto $(3,1,-1)$.
\vspace{0.5cm}

\item Encontre, se existir, os pontos de m\'aximo, m\'inimo e sela da fun\c{c}\~ao
	$f(x,y) = xy + \ds\frac{1}{x} +\ds\frac{1}{y}$.
\vspace{0.25cm}

\item Linearize a fun\c{c}\~ao $f(x,y) = 1+x\ln(xy-5)$ no ponto $(2,3)$.

\end{enumerate}

\begin{flushright}
	\textit{Boa Prova!}
\end{flushright}

\end{document}
