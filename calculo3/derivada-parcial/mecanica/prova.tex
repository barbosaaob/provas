\documentclass[a4paper,5pt]{amsbook}
%%%%%%%%%%%%%%%%%%%%%%%%%%%%%%%%%%%%%%%%%%%%%%%%%%%%%%%%%%%%%%%%%%%%%

\usepackage{booktabs}
% \usepackage{graphics}
% \usepackage[]{float}
% \usepackage{amssymb}
% \usepackage{amsfonts}
% \usepackage[]{amsmath}
% \usepackage[]{epsfig}
% \usepackage[brazil]{babel}
% \usepackage[utf8]{inputenc}
% \usepackage{verbatim}
%\usepackage[]{pstricks}
%\usepackage[notcite,notref]{showkeys}

%%%%%%%%%%%%%%%%%%%%%%%%%%%%%%%%%%%%%%%%%%%%%%%%%%%%%%%%%%%%%%

\newcommand{\sen}{\text{sen}}
\newcommand{\ds}{\displaystyle}

%%%%%%%%%%%%%%%%%%%%%%%%%%%%%%%%%%%%%%%%%%%%%%%%%%%%%%%%%%%%%%%%%%%%%%%%

\setlength{\textwidth}{16cm} %\setlength{\topmargin}{-0.1cm}
\setlength{\leftmargin}{1.2cm} \setlength{\rightmargin}{1.2cm}
\setlength{\oddsidemargin}{0cm}\setlength{\evensidemargin}{0cm}

%%%%%%%%%%%%%%%%%%%%%%%%%%%%%%%%%%%%%%%%%%%%%%%%%%%%%%%%%%%%%%%%%%%%%%%%

% \renewcommand{\baselinestretch}{1.6}
% \renewcommand{\thefootnote}{\fnsymbol{footnote}}
% \renewcommand{\theequation}{\thesection.\arabic{equation}}
% \setlength{\voffset}{-50pt}
% \numberwithin{equation}{chapter}

%%%%%%%%%%%%%%%%%%%%%%%%%%%%%%%%%%%%%%%%%%%%%%%%%%%%%%%%%%%%%%%%%%%%%%%

\begin{document}
\thispagestyle{empty}
\begin{minipage}[b]{0.45\linewidth}
\begin{tabular}{c}
\toprule{}
{{\bf UNIVERSIDADE FEDERAL DA GRANDE DOURADOS}}\\
{{\bf Prof.\ Adriano Barbosa}}\\



{{\bf C\'alculo III}}\\

\midrule{}
\hspace{8cm}09 de Agosto de 2016  \\
\bottomrule{}
\end{tabular}
%
\end{minipage} \hfill
\begin{minipage}[b]{0.58\linewidth}
\begin{flushright}
\def\arraystretch{1.2}
\begin{tabular}{|c|c|}
\hline\hline
1 & \hspace{1.2cm} \\
\hline
2& \\
\hline
3& \\
\hline
4&  \\
\hline
5&  \\
\hline
6&  \\
\hline
{\small Total}&  \\
\hline\hline
\end{tabular}
\end{flushright}
\end{minipage} \hfill
%------------------------
\vspace{0.3cm}\\
{\bf Aluno(a):}\dotfill{} \\
%----------------------------


\vspace{0.2cm}
%%%%%%%%%%%%%%%%%%%%%%%%%%%%%%%%   formulario  in\'icio  %%%%%%%%%%%%%%%%%%%%%%%%%%%%%%%%
\begin{enumerate}

\item Seja $r(t) = \left(t\ \sen(t), t^2\right)$. Calcule $r'(t)$ e $\ds\int r(t)\ dt$.
\vspace{0.5cm}

\item Calcule o limite: $\ds\lim_{(x,y)\rightarrow(0,0)} \frac{xy}{\sqrt{x^2+y^2}}$.
\vspace{0.5cm}

\item Linearize a fun\c{c}\~ao $f(x,y,z) = \sqrt{x^2+y^2+z^2}$ em $(3,2,6)$ e aproxime o valor de $f(3.02, 1.97, 5.99)$.
\vspace{0.5cm}

\item Dada uma fun\c{c}\~ao $f(x,y)$ com $x = s + t$ e $y = s-t$, utilize a regra da cadeia para mostrar que
	$$\ds{\left(\frac{\partial f}{\partial x}\right)}^2 - {\left(\frac{\partial f}{\partial y}\right)}^2 = \frac{\partial f}{\partial s}\ \frac{\partial f}{\partial t}$$
\vspace{0.5cm}

\item Encontre, se existirem, os pontos de m\'aximo, m\'inimo e sela da fun\c{c}\~ao $f(x,y) = x^2 + xy + y^2 + y$.
\vspace{0.5cm}

\item Utilize o m\'etodo dos multiplicadores de Lagrange para encontrar o volume
	m\'aximo de uma caixa retangular sem tampa utilizando $12m^2$ de papel\~ao.

\end{enumerate}

\begin{flushright}
	\textit{Boa Prova!}
\end{flushright}

\end{document}
