\documentclass[a4paper,5pt]{amsbook}
%%%%%%%%%%%%%%%%%%%%%%%%%%%%%%%%%%%%%%%%%%%%%%%%%%%%%%%%%%%%%%%%%%%%%

\usepackage{booktabs}
% \usepackage{graphics}
% \usepackage[]{float}
% \usepackage{amssymb}
% \usepackage{amsfonts}
% \usepackage[]{amsmath}
% \usepackage[]{epsfig}
% \usepackage[brazil]{babel}
% \usepackage[utf8]{inputenc}
% \usepackage{verbatim}
%\usepackage[]{pstricks}
%\usepackage[notcite,notref]{showkeys}

%%%%%%%%%%%%%%%%%%%%%%%%%%%%%%%%%%%%%%%%%%%%%%%%%%%%%%%%%%%%%%

\newcommand{\sen}{\text{sen}}
\newcommand{\ds}{\displaystyle}

%%%%%%%%%%%%%%%%%%%%%%%%%%%%%%%%%%%%%%%%%%%%%%%%%%%%%%%%%%%%%%%%%%%%%%%%

\setlength{\textwidth}{16cm} %\setlength{\topmargin}{-0.1cm}
\setlength{\leftmargin}{1.2cm} \setlength{\rightmargin}{1.2cm}
\setlength{\oddsidemargin}{0cm}\setlength{\evensidemargin}{0cm}

%%%%%%%%%%%%%%%%%%%%%%%%%%%%%%%%%%%%%%%%%%%%%%%%%%%%%%%%%%%%%%%%%%%%%%%%

% \renewcommand{\baselinestretch}{1.6}
% \renewcommand{\thefootnote}{\fnsymbol{footnote}}
% \renewcommand{\theequation}{\thesection.\arabic{equation}}
% \setlength{\voffset}{-50pt}
% \numberwithin{equation}{chapter}

%%%%%%%%%%%%%%%%%%%%%%%%%%%%%%%%%%%%%%%%%%%%%%%%%%%%%%%%%%%%%%%%%%%%%%%

\begin{document}
\thispagestyle{empty}
\begin{minipage}[b]{0.45\linewidth}
\begin{tabular}{c}
\toprule{}
{{\bf UNIVERSIDADE FEDERAL DA GRANDE DOURADOS}}\\
{{\bf Prof.\ Adriano Barbosa}}\\



{{\bf C\'alculo III}}\\

\midrule{}
\hspace{8cm}30 de Setembro de 2016 \\
\bottomrule{}
\end{tabular}
%
\end{minipage} \hfill
\begin{minipage}[b]{0.58\linewidth}
\begin{flushright}
\def\arraystretch{1.2}
\begin{tabular}{|c|c|}  % chktex 44
\hline\hline  % chktex 44
1 & \hspace{1.2cm} \\
\hline  % chktex 44
2& \\
\hline  % chktex 44
3& \\
\hline  % chktex 44
4&  \\
\hline  % chktex 44
5&  \\
\hline  % chktex 44
{\small Total}&  \\
\hline\hline  % chktex 44
\end{tabular}
\end{flushright}
\end{minipage} \hfill
%------------------------
\vspace{0.3cm}\\
{\bf Aluno(a):}\dotfill{} \\  % chktex 36
%----------------------------


\vspace{0.2cm}
%%%%%%%%%%%%%%%%%%%%%%%%%%%%%%%%   formulario  in\'icio  %%%%%%%%%%%%%%%%%%%%%%%%%%%%%%%%
\begin{enumerate}

\item Calcule a integral dupla $\ds\int_1^2\int_{\pi/2}^{\pi} x\cos(xy)\ dxdy$.
\vspace{1cm}

\item Calcule a integra tripla $\ds\iiint_B xy\ dV$, onde $B$ \'e
	o cilindro parab\'olico delimitado pelas equa\c{c}\~oes $x=y^2$, $y=x^2$, $z=0$ e
	$z=x+y$.
\vspace{1cm}

\item Utilizando coordenadas polares, calcule a integral $\ds \iint_R
	{(x^2+y^2)}^{3/2}\ dA$, onde $R$ \'e a metade superior do c\'{\i}rculo unit\'ario de
	centro na origem.
\vspace{1cm}

\item Calcule o trabalho realizado pelo campo $F(x,y) = (3+2xy, x^2-3y^2)$ ao
	mover uma part\'{\i}cula ao longo do caminho parametrizado por $r(t) = (e^t\sen
	t, e^t\cos t)$, com $0\le t \le \pi$.
\vspace{1cm}

\item
	\begin{enumerate}
		\item Enuncie as hip\'oteses do Teorema de Green.
		\item Dada a integral de linha $\ds\int_C y\ dx - x\ dy$, onde $C$ \'e a
			curva que percorre o tri\^angulo de v\'ertices $(0,0)$, $(2,0)$, $(0,4)$,
			$(0,0)$, nessa ordem. \'{E} poss\'{\i}vel aplicar o Teorema de Green
			para resolver essa integral? Resolva a integral.
	\end{enumerate}

\end{enumerate}

\begin{flushright}
	\textit{Boa Prova!}
\end{flushright}

\end{document}
