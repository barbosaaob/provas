\documentclass[a4paper,5pt]{amsbook}
%%%%%%%%%%%%%%%%%%%%%%%%%%%%%%%%%%%%%%%%%%%%%%%%%%%%%%%%%%%%%%%%%%%%%

\usepackage{booktabs}
% \usepackage{graphics}
% \usepackage[]{float}
% \usepackage{amssymb}
% \usepackage{amsfonts}
% \usepackage[]{amsmath}
% \usepackage[]{epsfig}
% \usepackage[brazil]{babel}
% \usepackage[utf8]{inputenc}
% \usepackage{verbatim}
%\usepackage[]{pstricks}
%\usepackage[notcite,notref]{showkeys}

%%%%%%%%%%%%%%%%%%%%%%%%%%%%%%%%%%%%%%%%%%%%%%%%%%%%%%%%%%%%%%

\newcommand{\sen}{\text{sen}}
\newcommand{\ds}{\displaystyle}

%%%%%%%%%%%%%%%%%%%%%%%%%%%%%%%%%%%%%%%%%%%%%%%%%%%%%%%%%%%%%%%%%%%%%%%%

\setlength{\textwidth}{16cm} %\setlength{\topmargin}{-0.1cm}
\setlength{\leftmargin}{1.2cm} \setlength{\rightmargin}{1.2cm}
\setlength{\oddsidemargin}{0cm}\setlength{\evensidemargin}{0cm}

%%%%%%%%%%%%%%%%%%%%%%%%%%%%%%%%%%%%%%%%%%%%%%%%%%%%%%%%%%%%%%%%%%%%%%%%

% \renewcommand{\baselinestretch}{1.6}
% \renewcommand{\thefootnote}{\fnsymbol{footnote}}
% \renewcommand{\theequation}{\thesection.\arabic{equation}}
% \setlength{\voffset}{-50pt}
% \numberwithin{equation}{chapter}

%%%%%%%%%%%%%%%%%%%%%%%%%%%%%%%%%%%%%%%%%%%%%%%%%%%%%%%%%%%%%%%%%%%%%%%

\begin{document}
\thispagestyle{empty}
\begin{minipage}[b]{0.45\linewidth}
\begin{tabular}{c}
\toprule{}
{{\bf UNIVERSIDADE FEDERAL DA GRANDE DOURADOS}}\\
{{\bf Prof.\ Adriano Barbosa}}\\



{{\bf C\'alculo III}}\\

\midrule{}
\hspace{8cm}29 de Setembro de 2016 \\
\bottomrule{}
\end{tabular}
%
\end{minipage} \hfill
\begin{minipage}[b]{0.58\linewidth}
\begin{flushright}
\def\arraystretch{1.2}
\begin{tabular}{|c|c|}  % chktex 44
\hline\hline  % chktex 44
1 & \hspace{1.2cm} \\
\hline  % chktex 44
2& \\
\hline  % chktex 44
3& \\
\hline  % chktex 44
4&  \\
\hline  % chktex 44
5&  \\
\hline  % chktex 44
{\small Total}&  \\
\hline\hline  % chktex 44
\end{tabular}
\end{flushright}
\end{minipage} \hfill
%------------------------
\vspace{0.3cm}\\
{\bf Aluno(a):}\dotfill{} \\  % chktex 36
%----------------------------


\vspace{0.2cm}
%%%%%%%%%%%%%%%%%%%%%%%%%%%%%%%%   formulario  in\'icio  %%%%%%%%%%%%%%%%%%%%%%%%%%%%%%%%
\begin{enumerate}

\item Calcule a integral dupla $\ds\int_0^{\ln(2)} \int_0^1 xye^{y^2x}\ dy dx$.
\vspace{1cm}

\item Calcule a integra tripla $\ds\iiint_B 2y\sen(xy)\ dV$, onde $B$ \'e
	a regi\~ao delimitada pelos planos $\ds x=\pi$, $\ds y=\frac{\pi}{2}$, $\ds
	z=\frac{\pi}{3}$ e pelos planos coordenados.
\vspace{1cm}

\item Utilizando coordenadas polares, calcule o volume do s\'olido delimitado por
	$z=0$ e $z=1-x^2-y^2$.
\vspace{1cm}

\item Calcule o trabalho realizado pelo campo $F(x,y) = (ye^{xy}, xe^{xy})$ ao
	mover uma part\'{\i}cula do ponto $(-1,1)$ at\'e o ponto $(2,0)$ ao longo do
	segmento de reta que liga esses dois pontos.
\vspace{1cm}

\item Sejam $\ds F(x,y)=\left(\frac{-y}{x^2+y^2}, \frac{x}{x^2+y^2}\right)$ um
	campo vetorial definido para $(x,y)\neq(0,0)$, $C$ a circunfer\^encia
	unit\'aria de centro na origem e $D$ a regi\~ao delimitada pela curva $C$.
	\begin{enumerate}
		\item Mostre que $\ds \int_C P\ dx + Q\ dy = 2\pi$.
		\item Mostre que $\ds \iint_D \left(\frac{\partial Q}{\partial x} -
				\frac{\partial P}{\partial x}\right)\ dA = 0$.
		\item Por que os itens acima n\~ao contradizem o Teorema de Green?
	\end{enumerate}

\end{enumerate}

\begin{flushright}
	\textit{Boa Prova!}
\end{flushright}

\end{document}
