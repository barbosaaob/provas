\documentclass[a4paper,5pt]{amsbook}
%%%%%%%%%%%%%%%%%%%%%%%%%%%%%%%%%%%%%%%%%%%%%%%%%%%%%%%%%%%%%%%%%%%%%

\usepackage{booktabs}
% \usepackage{graphics}
% \usepackage[]{float}
% \usepackage{amssymb}
% \usepackage{amsfonts}
% \usepackage[]{amsmath}
% \usepackage[]{epsfig}
% \usepackage[brazil]{babel}
% \usepackage[utf8]{inputenc}
% \usepackage{verbatim}
%\usepackage[]{pstricks}
%\usepackage[notcite,notref]{showkeys}

%%%%%%%%%%%%%%%%%%%%%%%%%%%%%%%%%%%%%%%%%%%%%%%%%%%%%%%%%%%%%%

\newcommand{\sen}{\text{sen}}
\newcommand{\ds}{\displaystyle}

%%%%%%%%%%%%%%%%%%%%%%%%%%%%%%%%%%%%%%%%%%%%%%%%%%%%%%%%%%%%%%%%%%%%%%%%

\setlength{\textwidth}{18cm} \setlength{\textheight}{25cm}
\setlength{\topmargin}{-1.5cm}
% \setlength{\leftmargin}{-5cm} \setlength{\rightmargin}{0cm}
\setlength{\oddsidemargin}{-1.3cm}\setlength{\evensidemargin}{0cm}

%%%%%%%%%%%%%%%%%%%%%%%%%%%%%%%%%%%%%%%%%%%%%%%%%%%%%%%%%%%%%%%%%%%%%%%%

% \renewcommand{\baselinestretch}{1.6}
% \renewcommand{\thefootnote}{\fnsymbol{footnote}}
% \renewcommand{\theequation}{\thesection.\arabic{equation}}
% \setlength{\voffset}{-50pt}
% \numberwithin{equation}{chapter}

%%%%%%%%%%%%%%%%%%%%%%%%%%%%%%%%%%%%%%%%%%%%%%%%%%%%%%%%%%%%%%%%%%%%%%%

\begin{document}
\thispagestyle{empty}
\begin{minipage}[b]{0.45\linewidth}
\begin{tabular}{c}
\toprule{}
{{\bf UNIVERSIDADE FEDERAL DA GRANDE DOURADOS}}\\
{{\bf Prof.\ Adriano Barbosa}}\\

{{\bf PS --- C\'alculo III}}\\

\midrule{}
Eng.\ de Energia\hspace{5cm}7 de Outubro de 2016 \\
\bottomrule{}
\end{tabular}
%
\end{minipage} \hfill
\begin{minipage}[b]{0.58\linewidth}
\begin{flushright}
\def\arraystretch{1.2}
\begin{tabular}{|c|c|}  % chktex 44
\hline\hline  % chktex 44
1 & \hspace{1.2cm} \\
\hline  % chktex 44
2& \\
\hline  % chktex 44
3& \\
\hline  % chktex 44
4&  \\
\hline  % chktex 44
5&  \\
\hline  % chktex 44
{\small Total}&  \\
\hline\hline  % chktex 44
\end{tabular}
\end{flushright}
\end{minipage} \hfill

%------------------------
\vspace{0.3cm}
{\bf Aluno(a):}\dotfill{}  % chktex 36

\vspace{0.3cm}
{\bf Avalia\c{c}\~ao respondida:}\dotfill{}
%----------------------------

\noindent{}\rule{\textwidth}{0.4pt}
\begin{center}
	\textbf{Voc\^e deve responder apenas as quest\~oes referentes a sua menor nota\@!}
\end{center}
\noindent{}\rule{\textwidth}{0.4pt}

\vspace{0.2cm}
%%%%%%%%%%%%%%%%%%%%%%%%%%%%%%%%   formulario  inicio  %%%%%%%%%%%%%%%%%%%%%%%%%%%%%%%%
\textbf{Avalia\c{c}\~ao P1:}
\begin{enumerate}
\item Calcule o limite: $\ds\lim_{(x,y)\rightarrow(0,0)}
	\frac{xy}{\sqrt{x^2+y^2}}$.
\vspace{0.5cm}

\item Linearize a fun\c{c}\~ao $f(x,y,z) = \sqrt{x^2+y^2+z^2}$ em $(3,2,6)$ e
	aproxime o valor de $f(3.02, 1.97, 5.99)$.
\vspace{0.5cm}

\item Dada uma fun\c{c}\~ao $f(x,y)$ com $x = s + t$ e $y = s-t$, utilize a regra da
	cadeia para mostrar que $$\ds{\left(\frac{\partial f}{\partial x}\right)}^2
	- {\left(\frac{\partial f}{\partial y}\right)}^2 = \frac{\partial
		f}{\partial s}\ \frac{\partial f}{\partial t}$$
\vspace{0.5cm}

\item Encontre os pontos de m\'aximo, m\'{\i}nimo e sela, se existirem, da fun\c{c}\~ao
	$$f(x,y) = xy-2x-2y-x^2-y^2$$
\vspace{0.25cm}

\item Utilize o m\'etodo dos multiplicadores de Lagrange para encontrar o volume
	m\'aximo de uma caixa retangular sem tampa utilizando $12m^2$ de papel\~ao.
\vspace{0.5cm}

\end{enumerate}

\textbf{Avalia\c{c}\~ao P2:}
\begin{enumerate}
\item Calcule a integral dupla $\ds\int_0^{\ln(2)} \int_0^1 xye^{y^2x}\ dy dx$.
\vspace{0.5cm}

\item Calcule a integra tripla $\ds\iiint_B xy\ dV$, onde $B$ \'e
	o cilindro parab\'olico delimitado pelas equa\c{c}\~oes $x=y^2$, $y=x^2$, $z=0$ e
	$z=x+y$.
\vspace{0.5cm}

\item Utilizando coordenadas polares, calcule o volume do s\'olido delimitado por
	$z=0$ e $z=1-x^2-y^2$.
\vspace{0.5cm}

\item Calcule o trabalho realizado pelo campo $F(x,y) = (3+2xy, x^2-3y^2)$ ao
	mover uma part\'{\i}cula ao longo do caminho parametrizado por $r(t) = (e^t\sen
	t, e^t\cos t)$, com $0\le t \le \pi$.
\vspace{0.5cm}

\item Sejam $\ds F(x,y)=\left(\frac{-y}{x^2+y^2}, \frac{x}{x^2+y^2}\right)$ um
	campo vetorial definido para $(x,y)\neq(0,0)$, $C$ a circunfer\^encia
	unit\'aria de centro na origem e $D$ a regi\~ao delimitada pela curva $C$.
	\begin{enumerate}
		\item Mostre que $\ds \int_C P\ dx + Q\ dy = 2\pi$.
		\item Mostre que $\ds \iint_D \left(\frac{\partial Q}{\partial x} -
				\frac{\partial P}{\partial y}\right)\ dA = 0$.
		\item Por que os itens acima n\~ao contradizem o Teorema de Green?
	\end{enumerate}
\end{enumerate}

\begin{flushright}
	\textit{Boa Prova!}
\end{flushright}

\end{document}
