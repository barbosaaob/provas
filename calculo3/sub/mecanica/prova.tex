\documentclass[a4paper,5pt]{amsbook}
%%%%%%%%%%%%%%%%%%%%%%%%%%%%%%%%%%%%%%%%%%%%%%%%%%%%%%%%%%%%%%%%%%%%%

\usepackage{booktabs}
% \usepackage{graphics}
% \usepackage[]{float}
% \usepackage{amssymb}
% \usepackage{amsfonts}
% \usepackage[]{amsmath}
% \usepackage[]{epsfig}
% \usepackage[brazil]{babel}
% \usepackage[utf8]{inputenc}
% \usepackage{verbatim}
%\usepackage[]{pstricks}
%\usepackage[notcite,notref]{showkeys}

%%%%%%%%%%%%%%%%%%%%%%%%%%%%%%%%%%%%%%%%%%%%%%%%%%%%%%%%%%%%%%

\newcommand{\sen}{\text{sen}}
\newcommand{\ds}{\displaystyle}

%%%%%%%%%%%%%%%%%%%%%%%%%%%%%%%%%%%%%%%%%%%%%%%%%%%%%%%%%%%%%%%%%%%%%%%%

\setlength{\textwidth}{18cm} \setlength{\textheight}{25cm}
\setlength{\topmargin}{-1.5cm}
% \setlength{\leftmargin}{-5cm} \setlength{\rightmargin}{0cm}
\setlength{\oddsidemargin}{-1.3cm}\setlength{\evensidemargin}{0cm}

%%%%%%%%%%%%%%%%%%%%%%%%%%%%%%%%%%%%%%%%%%%%%%%%%%%%%%%%%%%%%%%%%%%%%%%%

% \renewcommand{\baselinestretch}{1.6}
% \renewcommand{\thefootnote}{\fnsymbol{footnote}}
% \renewcommand{\theequation}{\thesection.\arabic{equation}}
% \setlength{\voffset}{-50pt}
% \numberwithin{equation}{chapter}

%%%%%%%%%%%%%%%%%%%%%%%%%%%%%%%%%%%%%%%%%%%%%%%%%%%%%%%%%%%%%%%%%%%%%%%

\begin{document}
\thispagestyle{empty}
\begin{minipage}[b]{0.45\linewidth}
\begin{tabular}{c}
\toprule{}
{{\bf UNIVERSIDADE FEDERAL DA GRANDE DOURADOS}}\\
{{\bf Prof.\ Adriano Barbosa}}\\

{{\bf PS --- C\'alculo III}}\\

\midrule{}
Eng.\ Mec\^anica\hspace{6cm}6 de Outubro de 2016 \\
\bottomrule{}
\end{tabular}
%
\end{minipage} \hfill
\begin{minipage}[b]{0.58\linewidth}
\begin{flushright}
\def\arraystretch{1.2}
\begin{tabular}{|c|c|}  % chktex 44
\hline\hline  % chktex 44
1 & \hspace{1.2cm} \\
\hline  % chktex 44
2& \\
\hline  % chktex 44
3& \\
\hline  % chktex 44
4&  \\
\hline  % chktex 44
5&  \\
\hline  % chktex 44
{\small Total}&  \\
\hline\hline  % chktex 44
\end{tabular}
\end{flushright}
\end{minipage} \hfill

%------------------------
\vspace{0.3cm}
{\bf Aluno(a):}\dotfill{}  % chktex 36

\vspace{0.3cm}
{\bf Avalia\c{c}\~ao respondida:}\dotfill{}
%----------------------------

\noindent{}\rule{\textwidth}{0.4pt}
\begin{center}
	\textbf{Voc\^e deve responder apenas as quest\~oes referentes a sua menor nota\@!}
\end{center}
\noindent{}\rule{\textwidth}{0.4pt}

\vspace{0.2cm}
%%%%%%%%%%%%%%%%%%%%%%%%%%%%%%%%   formulario  inicio  %%%%%%%%%%%%%%%%%%%%%%%%%%%%%%%%
\textbf{Avalia\c{c}\~ao P1:}
\begin{enumerate}
\item Calcule o limite $\ds\lim_{(x,y)\rightarrow(0,0)} f(x,y)$, onde 
	$$f(x,y) = \left\{
		\begin{array}{cl}
			\ds\frac{xy}{x^2+xy+y^2}, & \mbox{se } (x,y) \neq (0,0) \\
			0, & \mbox{se } (x,y) = (0,0)
		\end{array}\right.$$
\vspace{0.5cm}

\item Linearize a fun\c{c}\~ao $f(x,y,z) = \sqrt{x^2+y^2+z^2}$ em $(3, 2, 6)$ e  % chktex 36
	aproxime o valor de $f(3.02, 1.97, 5.99)$.  % chktex 36
\vspace{0.5cm}

\item Dada $f(x,y)$, com $x=r\cos(\theta)$ e $y=r\sen(\theta)$, mostre que  % chktex 36
	\begin{equation*}
	\ds{\left(\frac{\partial f}{\partial x}\right)}^2 + {\left(\frac{\partial f}{\partial y}\right)}^2 =
	{\left(\frac{\partial f}{\partial r}\right)}^2 + \frac{1}{r^2}{\left(\frac{\partial f}{\partial \theta}\right)}^2
	\end{equation*}

[Lembre que: $\sen^2(x) + \cos^2(x) = 1$.]
\vspace{0.5cm}

\item Encontre, se existirem, os pontos de m\'aximo, m\'{\i}nimo e sela da fun\c{c}\~ao
	$f(x,y) = x^2 + xy + y^2 + y$.  % chktex 36
\vspace{0.5cm}

\item Utilizando o m\'etodo dos multiplicadores de Lagrange, encontre as
	dimens\~oes da caixa retangular com volume m\'aximo cuja \'area total da
	superf\'{\i}cie \'e $64cm^2$.
\vspace{0.5cm}
\end{enumerate}

\textbf{Avalia\c{c}\~ao P2:}
\begin{enumerate}
\item Calcule a integral dupla $\ds\int_1^2\int_{\pi/2}^{\pi} x\cos(xy)\ dxdy$.
\vspace{0.5cm}

\item Calcule a integra tripla $\ds\iiint_B 2y\sen(xy)\ dV$, onde $B$ \'e
	a regi\~ao delimitada pelos planos $\ds x=\pi$, $\ds y=\frac{\pi}{2}$, $\ds
	z=\frac{\pi}{3}$ e pelos planos coordenados.
\vspace{0.5cm}

\item Utilizando coordenadas polares, calcule a integral $\ds \iint_R
	{(x^2+y^2)}^{3/2}\ dA$, onde $R$ \'e a metade superior do c\'{\i}rculo unit\'ario de
	centro na origem.
\vspace{0.5cm}

\item Calcule o trabalho realizado pelo campo $F(x,y) = (ye^{xy}, xe^{xy})$ ao
	mover uma part\'{\i}cula do ponto $(-1,1)$ at\'e o ponto $(2,0)$ ao longo do
	segmento de reta que liga esses dois pontos.
\vspace{0.5cm}

\item
	\begin{enumerate}
		\item Enuncie as hip\'oteses do Teorema de Green.
		\item Dada a integral de linha $\ds\int_C y\ dx - x\ dy$, onde $C$ \'e a
			curva que percorre o tri\^angulo de v\'ertices $(0,0)$, $(2,0)$, $(0,4)$,
			$(0,0)$, nessa ordem. \'{E} poss\'{\i}vel aplicar o Teorema de Green
			para resolver essa integral? Resolva a integral.
	\end{enumerate}
\end{enumerate}

\begin{flushright}
	\textit{Boa Prova!}
\end{flushright}

\end{document}
