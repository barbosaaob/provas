\documentclass{prova}

\usepackage{amssymb}

\renewcommand{\sin}{\mbox{sen}}

\professor{Prof.\@ Adriano Barbosa}
\disciplina{Fundamentos de Matem\'atica III}
\avaliacao{Final}
\curso{Matem\'atica}
\data{06/03/2018}

\begin{document}
	\cabecalho{5}  % o numero 5 indica a qnt de quadros na tabela de nota

	\textbf{Todas as respostas devem ser justificadas.}
    \vspace{0.5cm}

    \begin{questionario}
        \q{Calcule $\displaystyle{\left(\frac{-\sqrt{3}}{2}-\frac{i}{2}\right)}^{100}$.}
        \q{Resolva a equa\c{c}\~ao $x^2(x^2-2)=-2$.}
        \q{Mostre que $f=x^4+2x^3-x-2$ \'e divis\'{\i}vel por $x+2$ e por $x-1$.
            Conclua que $f$ tamb\'em \'e divis\'{\i}vel por $x^2+x-2$. }
        \q{Encontre as ra\'{\i}zes da equa\c{c}\~ao $x^3-4x^2+x+6=0$ sabendo que uma das
            ra\'{\i}zes \'e igual a soma das outras duas.}
        \q{Dado $x^2+ax+b$ um trin\^omio m\^onico (coeficiente l\'{\i}der igual a $1$)
            do segundo grau com $a$ e $b$ n\'umeros racionais e que possui uma
            raiz irracional $\alpha$. Mostre que este \'e o \'unico trin\^omio m\^onico
            do segundo grau tal que $\alpha$ \'e raiz.}
    \end{questionario}

    \vspace{2cm}
    Rela\c{c}\~oes de Girard:

    \vspace{0.3cm}
    Para $ax^2+bx+c=0$: $r_1+r_2=-\displaystyle \frac{b}{a}$ e
    $r_1r_2=\displaystyle \frac{c}{a}$
    \vspace{0.3cm}

    Para $ax^3+bx^2+cx+d=0$: $r_1+r_2+r_3=-\displaystyle \frac{b}{a}$,
    $r_1r_2+r_2r_3+r_3r_1=\displaystyle \frac{c}{a}$ e $r_1r_2r_3=-\displaystyle \frac{d}{a}$

\end{document}
