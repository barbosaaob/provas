\documentclass{prova}

\usepackage{amssymb}

\renewcommand{\sin}{\mbox{sen}}

\professor{Prof.\@ Adriano Barbosa}
\disciplina{Fundamentos de Matem\'atica III}
\avaliacao{P2}
\curso{Matem\'atica}
\data{20/02/2018}

\begin{document}
	\cabecalho{5}  % o numero 5 indica a qnt de quadros na tabela de nota

	\textbf{Todas as respostas devem ser justificadas.}
	\begin{questionario}
        \q{Determine o polin\^omio $p(x)$ de grau 3 cujas ra\'{\i}zes s\~ao $1$, $2$ e
            $3$ sabendo que $p(0)=1$.}
        \q{Resolva a equa\c{c}\~ao $x^3-4x^2+x+6=0$ sabendo que uma raiz \'e igual a
            soma das outras duas.}
        \q{Resolva a equa\c{c}\~ao $x^7-x^6+3x^5-3x^4+3x^3-3x^2+x-1=0$ sabendo que
            $i$ \'e uma raiz com multiplicidade $3$.}
        \q{Escreva as fun\c{c}\~oes quadr\'aricas abaixo na forma can\^onica e esboce
            seus gr\'aficos indicando o v\'ertice da par\'abola:}
            \begin{questionario}
                \qq{$f(x)=-x^2-x+1$}
                \qq{$f(x)=\displaystyle\frac{1}{2}x^2+x+\frac{1}{3}$}
            \end{questionario}
        \q{Verifique se a equa\c{c}\~ao $x^3+x^2-4x+6=0$ possui ra\'{\i}zes racionais.}
	\end{questionario}

    \vspace{2cm}
    Rela\c{c}\~oes de Girard:

    \vspace{0.3cm}
    Para $ax^2+bx+c=0$: $r_1+r_2=-\displaystyle \frac{b}{a}$ e
    $r_1r_2=\displaystyle \frac{c}{a}$
    \vspace{0.3cm}

    Para $ax^3+bx^2+cx+d=0$: $r_1+r_2+r_3=-\displaystyle \frac{b}{a}$,
    $r_1r_2+r_2r_3+r_3r_1=\displaystyle \frac{c}{a}$ e $r_1r_2r_3=-\displaystyle \frac{d}{a}$

\end{document}
