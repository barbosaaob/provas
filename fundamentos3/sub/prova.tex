\documentclass{prova}

\usepackage{amssymb}

\renewcommand{\sin}{\mbox{sen}}

\professor{Prof.\@ Adriano Barbosa}
\disciplina{Fundamentos de Matem\'atica III}
\avaliacao{PS}
\curso{Matem\'atica}
\data{27/02/2018}

\begin{document}
	\cabecalho{5}  % o numero 5 indica a qnt de quadros na tabela de nota

	\textbf{Todas as respostas devem ser justificadas.}
    \vspace{1cm}

    \textbf{Avalia\c{c}\~ao P1:}
	\begin{questionario}
        \q{Determine $x,y\in\mathbb{R}$ tais que:}
            \begin{questionario}
                \qq{$(x+yi)(2+3i)=1+8i$}
                \qq{${(x+yi)}^2=i^5$}
            \end{questionario}
        \q{Represente no plano cartesiano os conjuntos abaixo:}
            \begin{questionario}
                \qq{$\{z\in\mathbb{C}\ |\ |z|=1\}$}
                \qq{$\{z\in\mathbb{C}\ |\ Re(z)\ge 1\ \mbox{e}\ Im(z)\ge 2\}$}
            \end{questionario}
        \q{Calcule ${(-1+i)}^6$.}
        \q{Resolva a equa\c{c}\~ao $x^4=-1$.}
        \q{Calcule os valores de $\alpha\in\mathbb{R}$ tais que $f=g^2$, onde
            $f=x^4+2\alpha x^3-4\alpha x+4$ e $g=x^2+2x+2$.}
	\end{questionario}

    \vspace{1cm}
    \textbf{Avalia\c{c}\~ao P2}
    \begin{questionario}
        \q{Efetue a divis\~ao de $f=3x^5-x^4+2x^3+4x-3$ por $g=x^3-2x+1$.}
        \q{Resolva a equa\c{c}\~ao polinomial $(x+1)(x^2-x+1)={(x-1)}^3$.}
        \q{Determine $a$ e $b$ reais de modo que a equa\c{c}\~ao $2x^3-5x^2+ax+b=0$
            admita $2+i$ como raiz simples.}
        \q{Escreva as fun\c{c}\~oes quadr\'aticas abaixo na forma can\^onica e esboce
            seus gr\'aficos indicando o v\'ertice da par\'abola:}
            \begin{questionario}
                \qq{$f(x)=2x^2-x+2$}
                \qq{$f(x)=-x^2+2x-3$}
            \end{questionario}
        \q{As equa\c{c}\~oes $x^4+2x^3+3x^2+4x+2=0$ e $(x-a)(x-b)(x-c)(x-d)=0$,
            onde $a$, $b$, $c$ e $d$ s\~ao n\'umeros racionais, podem ter ra\'{\i}zes
            em comum?}
    \end{questionario}

    \vspace{2cm}
    Rela\c{c}\~oes de Girard:

    \vspace{0.3cm}
    Para $ax^2+bx+c=0$: $r_1+r_2=-\displaystyle \frac{b}{a}$ e
    $r_1r_2=\displaystyle \frac{c}{a}$
    \vspace{0.3cm}

    Para $ax^3+bx^2+cx+d=0$: $r_1+r_2+r_3=-\displaystyle \frac{b}{a}$,
    $r_1r_2+r_2r_3+r_3r_1=\displaystyle \frac{c}{a}$ e $r_1r_2r_3=-\displaystyle \frac{d}{a}$

\end{document}
