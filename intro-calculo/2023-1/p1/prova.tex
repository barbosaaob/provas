\documentclass{prova}

\usepackage{amsmath}
\usepackage{amsfonts}

\setlength{\textheight}{25cm}

\renewcommand{\sin}{\,\mbox{sen}\,}
\newcommand{\ds}{\displaystyle}

\professor{Prof.\@ Adriano Barbosa}
\disciplina{Introdu\c{c}\~ao ao C\'alculo}
\avaliacao{P1}
\curso{Qu\'{\i}mica}
\data{11/07/2023}

\begin{document}
	\cabecalho{5}  % o numero 5 indica a qnt de quadros na tabela de nota

    \textbf{Todas as respostas devem ser justificadas.}

    \begin{questionario}
        \q{Represente os intervalos abaixo geometricamente:}
            \begin{questionario}
                \begin{minipage}{0.4\textwidth}
                    \qq{$A = [0, 3]$}
                    \qq{$B = (-2, 2]$}
                    \qq{$C = (-\infty, 1]$}
                    \qq{$D = (0, \infty)$}
                \end{minipage}
                \begin{minipage}{0.4\textwidth}
                    \qq{$A \cap B$}
                    \qq{$A \cup B$}
                    \qq{$C \cap D$}
                    \qq{$C \cup D$}
                \end{minipage}
            \end{questionario}
        \q{Determine os algarismos que faltam em cada fra\c{c}\~ao:}
            \begin{questionario}
                \qq{$\ds\frac{3}{5}+\frac{3}{\_\!\_}=\frac{3\,\_\!\_}{35}$}
                \qq{$\ds\frac{1}{2}+\frac{3}{4}+\frac{5}{6}=\frac{2\,\_\!\_}{\_\!\_\,2}$}
                \qq{$\ds\frac{126}{8\,\_\!\_}=\frac{21}{\_\!\_\,\_\!\_}$}
            \end{questionario}
        \q{Duas pessoas combinaram de se encontrar entre 13h e 14h, no exato
           instante em que a posi\c{c}\~ao do ponteiro dos minutos do rel\'ogio
           coincidisse com a posi\c{c}\~ao do ponteiro das horas. Dessa forma, qual o
           hor\'ario que o encontro foi marcado?}
        \q{Que condi\c{c}\~oes a medida do lado de um quadrado deve satisfazer para
           que sua \'area seja numericamente maior que seu per\'{\i}metro?}
	    \q{Uma sorveteria vende $130$ picol\'es por dia por R\$ $5,00$ cada.
	       Observou-se que, durante uma promo\c{c}\~ao de ver\~ao, cada vez que
	       diminuia R\$ $0,50$ no pre\c{c}o do picol\'e, vendia $20$ unidades a mais
	       por dia. Qual deve ser o pre\c{c}o do picol\'e para que a receita da
	       sorveteria seja m\'axima?}
    \end{questionario}
\end{document}
