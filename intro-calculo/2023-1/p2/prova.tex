\documentclass{prova}

\usepackage{amsmath}
\usepackage{amsfonts}

\setlength{\textheight}{25cm}

\renewcommand{\sin}{\,\mbox{sen}\,}
\newcommand{\ds}{\displaystyle}

\professor{Prof.\@ Adriano Barbosa}
\disciplina{Introdu\c{c}\~ao ao C\'alculo}
\avaliacao{P2}
\curso{Qu\'{\i}mica}
\data{29/08/2023}

\begin{document}
	\cabecalho{5}  % o numero 5 indica a qnt de quadros na tabela de nota

    \textbf{Todas as respostas devem ser justificadas.}

    \begin{questionario}
        \q{Luiz ingeriu 500g de amoxicilina \`as 8h. Suponha que a meia-vida dessa
           subst\^ancia \'e de 1h.}
            \begin{questionario}
                \qq{Determine a massa dessa subst\^ancia no organismo de Luiz \`as
                    9h, 10h e 11h.}
                \qq{Qual \'e a massa restante no organismo de Luiz ap\'os $t$ horas
                    da ingest\~ao do rem\'edio?}
            \end{questionario}
        \q{}
            \begin{questionario}
                \qq{Se $\log{a}=2$, $\log{b}=3$ e $\log{c}=4$, determine o valor de
                    $\displaystyle\log{\left(\frac{bc^2}{a^4}\right)}$.}
                \qq{Determine os valores de $x$ na equa\c{c}\~ao $3^{2x} - 7\cdot 3^x + 12 =
                    0$.}
            \end{questionario}
        \q{Jo\~ao tomou emprestado R\$ 2.200,00 em um banco que cobra uma taxa de
           juros compostos de 10\% ao m\^es.}
            \begin{questionario}
                \qq{Se Jo\~ao pagar sua d\'{\i}vida ap\'os 3 meses, qual ser\'a o valor
                    total pago?}
                \qq{Escreva uma fun\c{c}\~ao $f$ que expresse a quantia paga em
                    fun\c{c}\~ao do tempo $t$, dado em meses.}
            \end{questionario}
        \q{Seja $f:\mathbb{R}\rightarrow\mathbb{R}$ \'e a fun\c{c}\~ao definida por
           $f(x) = 3x + \displaystyle\sin\left(\frac{\pi}{2}x\right)$. Qual o valor da soma
           $f(3)+\ldots+f(13)$?}
        \q{Se $\cos{x}+\sin{x} = 1$, qual o valor de $\sin(2x)$?}
    \end{questionario}
\end{document}
