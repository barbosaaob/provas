\documentclass{prova}

\usepackage{gensymb}
\usepackage{amsmath}
\usepackage{amsfonts}

\setlength{\textwidth}{18.5cm}
\setlength{\textheight}{27cm}
\setlength{\topmargin}{-2.5cm}
\setlength{\oddsidemargin}{-1.5cm}

\renewcommand{\sin}{\,\mbox{sen}\,}
\DeclareMathOperator{\sen}{sen}
\newcommand{\cotg}{\,\mbox{cotg}\,}
\newcommand{\ds}{\displaystyle}

\professor{Prof.\@ Adriano Barbosa}
\disciplina{Introdu\c{c}\~ao ao C\'alculo}
\avaliacao{PS}
\curso{Qu\'{\i}mica}
\data{05/09/2023}

\begin{document}
	\cabecalho{5}  % o numero 5 indica a qnt de quadros na tabela de nota

    \textbf{Todas as respostas devem ser justificadas.}

    \vspace{0.5cm}
    \textbf{Avalia\c{c}\~ao P1:}
    \begin{questionario}
        \q{M\'ario e Marcos decidiram comer pizza juntos. M\'ario decidiu repartir
           a pizza e retirou $\ds\frac{1}{4}$ da pizza para ele e deu
           $\ds\frac{1}{6}$ do que restou para Marcos. Para evitar discuss\~oes
           sobre quem comeu mais, da segunda vez que M\'ario foi repartir a pizza,
           ele ficou com $\ds\frac{1}{6}$ do que havia restado e deu
           $\ds\frac{1}{4}$ do que ficou para Marcos, dizendo que agora eles
           haviam comido a mesma quantidade de pizza. M\'ario estava certo?}
        \q{Uma f\'abrica de canetas tem um custo fixo di\'ario de produ\c{c}\~ao de R\$
           120,00, mais R\$ 0,40 por caneta. Cada caneta \'e vendida por R\$ 1,20.
           Determine:}
            \begin{questionario}
                \qq{O custo di\'ario de produ\c{c}\~ao de 80 canetas.}
                \qq{O custo di\'ario de produ\c{c}\~ao de $x$ canetas.}
                \qq{O lucro da empresa com a venda de 200 canetas.}
            \end{questionario}
        \q{Um experimento de agronomia mostra que a temperatura m\'edia da
           superf\'{\i}cie do solo $t(x)$, em graus Celsius, \'e determinada em fun\c{c}\~ao do
           res\'{\i}duo $x$ de planta e biomassa na superf\'{\i}cie, em $g/m^2$, conforme
           registrado na tabela abaixo:}
           \[\begin{array}{c|c|c|c|c|c}
             x(g/m^2) & 10 & 20 & 30 & 40 & 50 \\
             \hline
             t(x) (^{\circ} C) & 7,24 & 7,28 & 7,32 & 7,36 & 7,40
           \end{array}\]
           Qual a lei de forma\c{c}\~ao da fun\c{c}\~ao $t(x)$?
        \q{Determine todos os valores reais de $x$ para os quais $(x-2)(x-1)>0$.}
        \q{Uma ind\'ustria produz mensalmente $x$ lotes de um produto. O valor
           mensal resultante da venda deste produto \'e $V(x)=3x^2-12x$ e o custo
           mensal de produ\c{c}\~ao \'e dado por $C(x)=5x^2-40x-40$. Qual \'e o n\'umero de
           lotes mensais que essa ind\'ustria deve vender para obter lucro m\'aximo?}
    \end{questionario}

    \textbf{Avalia\c{c}\~ao P2:}
    \begin{questionario}
        \q{H\'a uma lenda que credita a inven\c{c}\~ao do xadrez a um br\^amane de uma
           c\^orte indiana que, atendendo a um pedido do rei, inventou o jogo para
           demonstrar o valor da intelig\^encia. O rei, encantado com o invento,
           ofereceu ao br\^amane a escolha de uma recompensa. De acordo com essa
           lenda, o inventor do jogo de xadrez pediu ao rei que a recompensa fosse
           pega em gr\~aos de arroz da seguinte maneira: 1 gr\~ao para a casa 1 do
           tabuleiro, 2 gr\~aos para a casa 2, 4 para a casa 3, 8 para a casa 4 e
           assim sucessivamente. Ou seja, a quantidade de gr\~aos para cada casa do
           tabuleiro correspondia ao dobro da quantidade da casa imediatamente
           anterior.}
           \begin{questionario}
               \qq{De acordo com a lenda, qual \'e a quantidade de gr\~aos de arroz
                   correspondente \`a casa 8?}
               \qq{Escreve uma fun\c{c}\~ao $f$ que expresse a quantidade de gr\~aos de
                   arroz em fun\c{c}\~ao do n\'umero $x$ da casa do tabuleiro.}
               \qq{Escreva, na forma de pot\^encia, quantos gr\~aos de arroz devem
                   ser colocados na \'ultima casa do tabuleiro de xadrez.}
           \end{questionario}
        \q{Suponha que a desvaloriza\c{c}\~ao de um autom\'ovel seja de 20\% ao ano a
           partir de sua compra. Carlos comprou um autom\'ovel pagando R\$
           50.000,00. Depois de quanto tempo seu valor ser\'a de R\$ 25.000,00?
           (Utilize $\log 2=0,3$)}
        \q{}
          \begin{questionario}
            \qq{Seja $\ds\frac{\pi}{2} < x < \pi$. O valor de $x$ tal que
                $\sin{x} = \frac{1}{2}$.}
            \qq{Seja $x$ um arco do terceiro quadrante. Se $\sec{x}=-4$,
                determine o valor de $\cotg{x}$. $\left(
                \ds\sec{x}=\frac{1}{\cos{x}},
                \cotg{x}=\frac{\cos{x}}{\sen{x}}\right)$}
          \end{questionario}
        \q{A popula\c{c}\~ao de peixes em uma lagoa varia conforme o regime de chuvas
           da regi\~ao. Ela cresce no per\'{\i}odo chuvoso e decresce no per\'{\i}odo de
           estiagem. Esta popula\c{c}\~ao \'e descrita pela express\~ao $P(t) =
           10^3\ds\left[\cos\left(\frac{t-2}{6}\ \pi\right) + 5\right]$ em que o
           tempo $t$ \'e medido em meses. Determine:}
           \begin{questionario}
               \qq{O valor m\'aximo e m\'{\i}nimo da popula\c{c}\~ao.}
               \qq{Em quais meses do ano a popula\c{c}\~ao atinge seu m\'aximo e seu
                   m\'{\i}nimo.}
           \end{questionario}
        \q{Suponha que uma revista publicou um artigo no qual era estimado que
           no ano $2016+x$, com $x\in\{0,1,2,\cdots,10\}$, o valor arrecadado dos
           impostos incidentes sobre as exporta\c{c}\~oes em certo pa\'{\i}s, em milh\~oes de
           d\'olares, poderia ser obtido pela fun\c{c}\~ao $f(x) =
           200+12\cos\left(\frac{\pi}{3}x\right)$. Caso essa previs\~ao se
           confirme, relativamente ao total arrecadado a cada ano,
           determine se as afirma\c{c}\~oes abaixo s\~ao verdadeiras ou falsas.}
           \begin{questionario}
               \qq{O valor m\'aximo ocorrer\'a apenas em 2021.}
               \qq{Atingir\'a o valor m\'{\i}nimo apenas em duas ocasi\~oes.}
               \qq{Poder\'a superar 300 milh\~oes de d\'olares.}
               \qq{nunca ser\'a inferior a 200 milh\~oes de d\'olares.}
           \end{questionario}
    \end{questionario}

\end{document}
