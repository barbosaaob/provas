\documentclass{prova}

\usepackage{amssymb}
\usepackage{gensymb}

\renewcommand{\sin}{\mbox{sen}}
\newcommand{\ra}{\rightarrow}
\newcommand{\lra}{\leftrightarrow}
\newcommand{\Ra}{\Rightarrow}
\newcommand{\LRa}{\Leftrightarrow}
\renewcommand{\lnot}{\sim}
\newcommand{\larg}{\vdash}
\newcommand{\ds}{\displaystyle}
\newcommand{\sen}{\mathop\mathrm{sen}\nolimits}
\newcommand{\tg}{\mathop\mathrm{tg}\nolimits}
\newcommand{\cotg}{\mathop\mathrm{cotg}\nolimits}
\newcommand{\cossec}{\mathop\mathrm{cossec}\nolimits}

\professor{Prof. Adriano Barbosa}
\disciplina{Introdução ao Cálculo}
\avaliacao{P2}
\curso{Matem\'atica}
\data{27/11/2020}

\begin{document}
    \cabecalho{5}  % o numero 5 indica a qnt de quadros na tabela de nota
    
    \textbf{Todas as respostas devem ser justificadas.}
    \begin{questionario}
        \q{(2,0 pts) Determine o valor de $k$ de modo que as retas $r$ e $s$
           sejam paralelas.}
           \[r: (k+1)x+ky+1=0\]
           \[s: kx+(k+1)y+1=0\]
        \q{A empresa $A$ oferece salário mensal de R\$2.000,00 mais comissão
           de R\$400,00 para cada produto vendido. A empresa $B$ oferece salário
           mensal de R\$2.600,00 mais comissão de R\$300,00 para cada produto vendido.}
            \begin{questionario}
                \qq{(0,5 pts) Vendendo 3 produtos por mês, qual empresa oferece maior pagamento?}
                \qq{(0,5 pts) Vendendo 8 produtos por mês, qual empresa oferece maior pagamento?}
                \qq{(1,0 pts) Quantos produtos precisam ser vendidos para que o pagamento
                    das duas empresas sejam iguais?}
            \end{questionario}
        \q{(2,0 pts) Seja $f:\mathbb{R}\rightarrow\mathbb{R}$, $f(x)=ax^2+bx+c$. Determine
           os valores de $a$, $b$ e $c$ de modo que o gráfico de $f$ cruze o eixo
           $x$ em $2$ e $8$ e que o vértice da parábola esteja no ponto $(5,9)$.}
        \q{(2,0 pts) Uma empresa alugou um ônibus com 40 lugares para uma
           excursão e cobrou de cada pessoa R\$ 50,00 mais R\$2,00 por cada lugar vago.
           Para qual número de pessoas o faturamento da empresa é máximo? Qual
           é o valor do faturamento máximo?}

           %Por exemplo: para uma excursão com 10 pessoas, teremos 30 lugares
           %vagos e, consequentemente, o custo por pessoa será de $50 + 2\times 30 = 110$
           %reais.
        \q{(2,0 pts) Uma loja fez a seguinte promoção
           ``na compra de $x$ bolas de gude, receba x\% de desconto''. A promoção é válida para
           compras de até 70 bolas de gude, onde o desconto máximo de 70\% é concedido. Maria e
           João compraram respectivamente 20 e 35 bolas de gude cada um. Algum deles poderia ter
           comprado mais bolas de gude pagando o mesmo valor?}
    \end{questionario}
\end{document}
