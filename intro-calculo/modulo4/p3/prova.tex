\documentclass{prova}

\usepackage{amssymb}
\usepackage{gensymb}

\renewcommand{\sin}{\mbox{sen}}
\newcommand{\ra}{\rightarrow}
\newcommand{\lra}{\leftrightarrow}
\newcommand{\Ra}{\Rightarrow}
\newcommand{\LRa}{\Leftrightarrow}
\renewcommand{\lnot}{\sim}
\newcommand{\larg}{\vdash}
\newcommand{\ds}{\displaystyle}
\newcommand{\sen}{\mathop\mathrm{sen}\nolimits}
\newcommand{\tg}{\mathop\mathrm{tg}\nolimits}
\newcommand{\cotg}{\mathop\mathrm{cotg}\nolimits}
\newcommand{\cossec}{\mathop\mathrm{cossec}\nolimits}

\professor{Prof. Adriano Barbosa}
\disciplina{Introdução ao Cálculo}
\avaliacao{P3}
\curso{Matem\'atica}
\data{04/12/2020}

\begin{document}
    \cabecalho{5}  % o numero 5 indica a qnt de quadros na tabela de nota
    
    \textbf{Todas as respostas devem ser justificadas.}
    \begin{questionario}
        \q{Dada $f:\mathbb{R}^{*}\rightarrow\mathbb{R}$, $f(x)=\left(\frac{1}{2}\right)^{1-\frac{1}{x}}$.}
            \begin{questionario}
                \qq{Encontre as soluções reais das equações $f(x)=\frac{1}{2}$ e $f(x)=\frac{1}{4}$.}
                \qq{Encontre as soluções da inequação $f(x)<\frac{1}{4}$, para $x\in\mathbb{R}$.}
            \end{questionario}
        \q{Alguns elementos radioativos decaem a uma taxa
           proporcional à própria massa. Considere um elemento $E$ cuja meia-vida é de 1000
           anos.}
            \begin{questionario}
                \qq{Qual o tempo necessário para que a massa do elemento $E$
                    reduza a $\frac{1}{4}$ da inicial?}
                \qq{Determine a função que dá a massa do elemento $E$ em função
                    do tempo medido em anos.}
            \end{questionario}
        \q{O crescimento de uma cultura de bactérias em função do tempo é dado
           pela função $P(t)=P_0a^{t}$, onde $P_0$ é a população inicial e $a>1$ é uma constante real. Se uma
           certa cultura de bactérias ficou 32 vezes maior após 1 ano, quantas
           vezes maior ela estava após 3 meses?}

           Não use calculadoras até que seja necessário apresentar a resposta final.
        \q{Um indivíduo tem um saldo devedor de R\$50,00 e a empresa de cartão
           de crédito cobra juros de 13\% ao mês.}
            \begin{questionario}
                \qq{Em quanto tempo a dívida dobra se não for paga?}
                \qq{Em quanto tempo a dívida triplica se não for paga?}
                \qq{Observando as respostas dos itens anteriores, quanto tempo
                    levou para a dívida ir R\$50,00 para R\$100,00 (aumentar
                    R\$50,00)? E para ir de R\$100,00 para R\$150,00 (novamente
                    aumentar R\$50,00)? Qual a justificativa para esse
                    comportamento?}
            \end{questionario}
           Não use calculadoras até que seja necessário apresentar a resposta final.
%        \q{Sabendo que o decaimento da quantidade de uma droga no sangue é
%           exponencial e que uma determinada droga reduz-se a 15\% da quantidade inicial
%           após 24h
%           da sua administração. Qual porcentagem resta 10h após a
%           administração? Em quanto tempo a quantidade de droga no organismo
%           reduz-se a 50\% da dose inicial?}
%
%           Não use calculadoras até que seja necessário apresentar a resposta final.
        \pagebreak
        \q{Sabe-se que, após períodos de mesma duração, a população de uma
           cidade fica multiplicada pelo mesmo fator. Sabendo que a população de uma
           cidade era de 50 mil habitantes em 2015 e 130 mil em 2020:}
            \begin{questionario}
                \qq{Estime a população da cidade em 2025.}
                \qq{Em que ano a cidade terá 200 mil habitantes?}
            \end{questionario}
           
           Use os valores $\log_{10}4=0,6$, $\log_{10}5=0,69$ e $\log_{10}13=1,11$.

           Não use calculadoras até que seja necessário apresentar a resposta final.
    \end{questionario}
\end{document}
