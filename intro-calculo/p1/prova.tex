\documentclass{prova}

\usepackage{amssymb}

\renewcommand{\sin}{\mbox{sen}}
\newcommand{\ra}{\rightarrow}
\newcommand{\lra}{\leftrightarrow}
\newcommand{\Ra}{\Rightarrow}
\newcommand{\LRa}{\Leftrightarrow}
\renewcommand{\lnot}{\sim}
\newcommand{\larg}{\vdash}

\professor{Prof. Adriano Barbosa}
\disciplina{Introdução ao Cálculo}
\avaliacao{P1}
\curso{Matem\'atica}
\data{11/09/2020}

\begin{document}
    \cabecalho{5}  % o numero 5 indica a qnt de quadros na tabela de nota
    
    \textbf{Todas as respostas devem ser justificadas.}
    \begin{questionario}
        \q{}
            \begin{questionario}
                \qq{Encontre a fração irredutível equivalente a $0,2666...$.}
                \qq{Liste, em ordem crescente, as frações da forma
                    $\displaystyle\frac{n}{n+1},\ n\in\mathbb{N}$, que são menores do que
                    $\displaystyle\frac{9}{11}$.}
            \end{questionario}
        \q{Determine os algarismos que faltam em cada fração:}
	    \begin{questionario}
                \qq{$\displaystyle\frac{126}{8\,\_\!\_}=\frac{21}{\_\!\_\,\_\!\_}$}
                \qq{$\displaystyle\frac{\_\!\_\,\_\!\_\,8}{33\,\_\!\_}=\frac{4}{5}$}
            \end{questionario}
	\q{Dadas as funções $f:X\rightarrow Y$, $g:Y\rightarrow X$ tais que
	   $f(g(y))=y, \forall y\in Y$, $X\subset\mathbb{R}$ e
	   $Y\subset\mathbb{R}$. Determine se as afirmações abaixo são
	   verdadeiras ou falsas e justifique ou dê um contraexemplo.}
	    \begin{questionario}
                \qq{$f$ é injetiva.}
		\qq{$f$ é sobrejetiva.}
		\qq{$f$ é bijetiva.}
		\qq{$g$ é injetiva.}
		\qq{$g$ é sobrejetiva.}
	    \end{questionario}
	\q{Seja $f:\mathbb{R}\rightarrow\mathbb{R}, f(x)=2x-3$ com
	   $f(x)=3g^{-1}(x)$ e $g^{-1}(x)$ é a inversa de $g(x)=ax+b$.
           Determine o valor de $b-a$.}
        \q{Seja $f:[1,+\infty]\rightarrow[3,+\infty], f(x)=2(x-1)^2+3$.}
	    \begin{questionario}
                \qq{Mostre que $f$ é bijetiva.}
		\qq{Encontre a inversa de $f$.}
		\qq{Sem usar a regra da função inversa, determine os valores de
		    $y$ para os quais $f^{-1}(y)=1$.}
            \end{questionario}
    \end{questionario}
\end{document}
