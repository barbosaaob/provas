\documentclass{prova}

\usepackage{amssymb}
\usepackage{gensymb}

\renewcommand{\sin}{\mbox{sen}}
\newcommand{\ra}{\rightarrow}
\newcommand{\lra}{\leftrightarrow}
\newcommand{\Ra}{\Rightarrow}
\newcommand{\LRa}{\Leftrightarrow}
\renewcommand{\lnot}{\sim}
\newcommand{\larg}{\vdash}
\newcommand{\ds}{\displaystyle}

\professor{Prof. Adriano Barbosa}
\disciplina{Introdução ao Cálculo}
\avaliacao{P3}
\curso{Matem\'atica}
\data{25/09/2020}

\begin{document}
    \cabecalho{5}  % o numero 5 indica a qnt de quadros na tabela de nota
    
    \textbf{Todas as respostas devem ser justificadas.}
    \begin{questionario}
	\q{Sabe-se que, após períodos de mesma duração, a população de uma
	   cidade fica multiplicada pelo mesmo fator. Sabendo que a população
	   de uma cidade era de 150 mil habitantes em 2010 e 200 mil em 2020:}
	    \begin{questionario}
	        \qq{Estime a população da cidade em 2030.}
	        \qq{Em que ano a cidade terá 400 mil habitantes?}
	    \end{questionario}
	   Não use calculadoras até que seja necessário apresentar a resposta final.
	\q{A bula de um medicamento informava que sua meia-vida (tempo
	   necessário para que a substância atinja metade de sua quantidade)
	   é de 3 horas. Sabendo que um indivíduo ingeriu 30 mg desse
	   medicamento:}
	    \begin{questionario}
		\qq{Qual a quantidade de remédio após 12 horas da ingestão?}
		\qq{Quanto tempo após a ingestão a quantidade do remédio no
		    organismo é igual a 10 mg?}
	    \end{questionario}
	   Não use calculadoras até que seja necessário apresentar a resposta final.
        \q{Determine o valor das expressões abaixo:}
	    \begin{questionario}
                \qq{$\ds\log_b\left[\log_a\sqrt[b]{\sqrt[b]{a}}\right]$.}
	        \qq{$\ds x^\frac{\log_b a}{\log_b x}$, onde $a>0$ e $x>0$.}
            \end{questionario}
	\q{A função $\ds f(t) = 100e^{-\frac{t\ln3}{20}}$ dá a massa de um material
	   radioativo $R$ após $t$ anos de decaimento radioativo.}
	    \begin{questionario}
	        \qq{Qual a massa inicial do elemento $R$?}
		\qq{Quanto tempo é necessário para que a massa do elemento $R$
		    seja a metade da quantidade inicial?}
	    \end{questionario}
	   [Observe que $\ln x = \log_e x$.] \\
	   Não use calculadoras até que seja necessário apresentar a resposta final.
	\newpage{}
	\q{Um indivíduo tem um saldo devedor de R\$100,00 e a empresa de cartão
	   cobra juros 11\% ao mês.}
	    \begin{questionario}
	        \qq{Em quanto tempo a dívida dobra se não for paga?}
		\qq{Em quanto tempo a dívida triplica se não for paga?}
		\qq{Observando as respostas dos itens anteriores, quando
		    tempo levou para que a dívida ir de R\$100,00 para
		    R\$200,00? E para ir de R\$200,00 para R\$ 300,00? Qual a
		    justificativa para esse comportamento?}
            \end{questionario}
	   Não use calculadoras até que seja necessário apresentar a resposta final.
    \end{questionario}
\end{document}
