\documentclass{prova}

\usepackage{amssymb}
\usepackage{gensymb}

\renewcommand{\sin}{\mbox{sen}}
\newcommand{\ra}{\rightarrow}
\newcommand{\lra}{\leftrightarrow}
\newcommand{\Ra}{\Rightarrow}
\newcommand{\LRa}{\Leftrightarrow}
\renewcommand{\lnot}{\sim}
\newcommand{\larg}{\vdash}
\newcommand{\ds}{\displaystyle}
\newcommand{\sen}{\mathop\mathrm{sen}\nolimits}
\newcommand{\tg}{\mathop\mathrm{tg}\nolimits}
\newcommand{\cotg}{\mathop\mathrm{cotg}\nolimits}
\newcommand{\cossec}{\mathop\mathrm{cossec}\nolimits}

\professor{Prof. Adriano Barbosa}
\disciplina{Introdução ao Cálculo}
\avaliacao{PS}
\curso{Matem\'atica}
\data{05/10/2020}

\begin{document}
    \cabecalho{5}  % o numero 5 indica a qnt de quadros na tabela de nota
    
    \textbf{Todas as respostas devem ser justificadas.}
    \begin{questionario}
        \q{Dadas as funções $f:X\rightarrow Y$, $g:Y\rightarrow X$ tais que
        $g(f(x))=x,\forall x\in X$, $X\in\mathbb{R}$ e $Y\in\mathbb{R}$.
        Determine se as afirmações abaixo são verdadeiras ou falsas e
        justifique ou dê um contraexemplo.}
        \begin{questionario}
	    \qq{$f$ é injetiva.}
	    \qq{$f$ é sobrejetiva.}
	    \qq{$g$ é injetiva.}
	    \qq{$g$ é sobrejetiva.}
	\end{questionario}
	\q{Supondo que 3 pessoas, trabalhando 8 horas por dia, levantem um muro
	de 40 m em 5 dias. Quantos dias são necessários para que um grupo de 5
	pessoas, trabalhando 4 horas por dia, construa um muro de 20 metros?}
	\q{Dada a função quadrática $f:[0,1]\rightarrow\mathbb{R}$,
	$f(x)=x^2-4x+1$, determine seu valor máximo e mínimo e para quais
	valores de $x$ eles são atingidos.}
        \q{Encontre as soluções reais da inequação $2^{3x+1} < 3^{2x-1}$.}
	\q{Sabendo que $\sen x \cos x<0$ e $\cotg x \cossec x >0$, determine o
	maior intervalo $I\subset [0, 2\pi]$ tal que $x\in I$.}
	\q{Calcule o limite $\ds\lim_{x\rightarrow 2}
        \frac{\sqrt{6-x}-2}{\sqrt{3-x}-1}$ sem usar tabelas.}
    \end{questionario}
\end{document}
