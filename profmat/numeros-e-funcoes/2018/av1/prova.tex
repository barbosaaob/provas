\documentclass{prova}

\usepackage{amssymb}

\setlength{\topmargin}{-2.5cm}
\setlength{\textheight}{26cm}

\professor{Prof.\@ Adriano Barbosa}
\disciplina{N\'umeros e Fun\c{c}\~oes Reais}
\avaliacao{AV1}
\curso{PROFMAT}
\data{12/05/2018}

\begin{document}
	\cabecalho{6}  % o numero 5 indica a qnt de quadros na tabela de nota
	\begin{questionario}
        \q{Dados conjuntos $A$, $B$ e $C$, mostre que:}
            \begin{questionario}
                \qq{$A-(B\cup C)=(A-B)\cap(A-C)$}
                \qq{$A-(B\cap C)=(A-B)\cup(A-C)$}
            \end{questionario}
        \q{Uma sequ\^encia $(a_n)$ \'e tal que $a_1=1$ e}
        \[a_{n+1} = \displaystyle\frac{a_1+a_2+\cdots+a_n}{n+1}\]
        para todo $n\ge1$. Mostre que os valores de $a_n$, para $n\ge2$ s\~ao
        todos iguais.
        \q{Sejam $f:X\rightarrow Y$ e $g:Y\rightarrow X$ duas fun\c{c}\~oes. Prove
            que:}
            \begin{questionario}
                \qq{se $g\circ f$ \'e injetiva, ent\~ao $f$ \'e injetiva.}
                \qq{se $f\circ g$ sobrejetiva, ent\~ao $f$ \'e sobrejetiva.}
            \end{questionario}
        \q{}
            \begin{questionario}
                \qq{Se $r\ne0$ \'e um n\'umero racional, prove que $r\sqrt{2}$ \'e
                    irracional.}
                \qq{Dado qualquer n\'umero real $\varepsilon>0$, prove que existe
                    um n\'umero irracional $\alpha$ tal que
                    $0<\alpha<\varepsilon$.}
                \qq{Mostre que todo intervalo $[a,b]$, com $a<b$, cont\'em algum
                    n\'umero irracional.}
            \end{questionario}
        \q{Sejam $x$ e $y$ n\'umeros reais quaisquer.}
            \begin{questionario}
                \qq{Mostre que $|x+y|\le|x|+|y|$.}
                \qq{Mostre que $\left| |x|-|y|\right| \le |x-y|$.}
            \end{questionario}
        \q{Um pequeno barco a vela, com 5 tripulantes, deve
            atravessar o oceano em 30 dias. Seu suprimento de \'agua pot\'avel
            permite a cada pessoa dispor de 2 litros de \'agua por dia (e \'e o
            que os tripulantes fazem). Ap\'os 13 dias de viagem, o barco encontra
            2 n\'aufragos numa jangada e os acolhe. Pergunta-se:}
            \begin{questionario}
                \qq{Quantos litros de \'agua por dia caber\~ao agora a cada pessoa
                    se a viagem prossegur como antes?}
                \qq{Se os 7 ocupantes de agora continuarem consumindo 2
                    litros de \'agua cada um, em quantos dias, no m\'aximo, ser\'a
                    necess\'ario encontrar uma ilha onde haja \'agua?}
            \end{questionario}
	\end{questionario}
\end{document}
