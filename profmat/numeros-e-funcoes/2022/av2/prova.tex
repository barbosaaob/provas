\documentclass{prova}

\usepackage{amssymb}
\usepackage{amsmath}

\DeclareMathOperator{\sen}{sen}
\DeclareMathOperator{\tg}{tg}

\setlength{\topmargin}{-2.5cm}
\setlength{\textheight}{26cm}

\professor{Prof.\@ Adriano Barbosa}
\disciplina{N\'umeros e Fun\c{c}\~oes Reais}
\avaliacao{2}
\curso{PROFMAT}
\data{08/07/2022}

\begin{document}
	\cabecalho{5}  % o numero 5 indica a qnt de quadros na tabela de nota
	\begin{questionario}
        %\q{Resolva a inequa\c{c}\~ao $3x^4+7x^2-10<0$.}
        \q{Um time de futebol joga num est\'adio com capacidade para 15.000
           espectadores. Com o ingresso custando R\$15,00, a m\'edia de p\'ublico nos
           jogos \'e de 10.000 pessoas. Uma pesquisa de mercado indicou que o
           p\'ublico aumentaria em 1.000 pessoas em cada jogo para cada R\$ 1,00
           diminuido no valor do ingresso. Qual deve ser o pre\c{c}o do ingresso para
           que o faturamenteo com a venda de ingressos seja o maior poss\'{\i}vel?}
        \q{Dada a fun\c{c}\~ao quadr\'atica $f(x)=ax^2+bx+c$, consideremos as fun\c{c}\~oes
           afins $g(x)=mx+t$, onde $m$ \'e fixo e $t$ ser\'a escolhido
           convenientemente. Prove que existe uma (\'unica) escolha de $t$ para a
           qual a equa\c{c}\~ao $f(x)=g(x)$ tem uma, e somente uma, raiz $x$.}
        \q{Seja $p(x)$ um polin\^omio cujo grau $n$ \'e um n\'umero \'{\i}mpar. Mostre que
           existem n\'umeros reais $x_1,x_2$ tais que $p(x_1)>0$ e $p(x_2)<0$.
           Conclua da\'{\i} que todo polin\^omio de grau \'{\i}mpar admite pelo menos uma raiz
           real.}
        \q{}
            \begin{questionario}
                \qq{Encontre uma express\~ao para $\sen(3x)$ como um polin\^omio de
                    coeficientes inteiros em termos de $\sen x$.}
                \qq{Mostre que $\sen 10^{\circ}$ \'e raiz de um polin\^omio com
                    coeficientes inteiros.}
            \end{questionario}
        \q{Mostre que, para todo $m>0$, a equa\c{c}\~ao $\sqrt{x}+m=x$ tem exatamente
           uma raiz.}
        \q{A grandeza $y$ se exprime como $y=ba^t$ em fun\c{c}\~ao de $t$. Sejam $d$
           o acrescimo que se deve dar a $t$ para que $y$ dobre e $m$ (meia-vida
           de $y$) o acr\'escimo de $t$ necess\'ario para que $y$ se reduza \`a metade.
           Mostre que $m=-d$ e $y=b2^{t/d}$, logo $d=\log_a
           2=\frac{1}{\log_2 a}$.}
        %\q{Calcule o valor das express\~oes:}
        %    \begin{questionario}
        %        \qq{$\log_n \left[\log_n \sqrt[n]{\sqrt[n]{\sqrt[n]{n}}}\right]$}
        %        \qq{$x^{\frac{\log a}{\log b}}$, onde $a,x>0$ e a base dos
        %            logaritmos \'e fixada arbitrariamente.}
        %    \end{questionario}
        \q{A express\~ao $M(t)=200e^{-(t\ln 2)/30}$ d\'a a massa em gramas do
           c\'esio 137 que restar\'a de uma quantidade inicial ap\'os $t$ anos de
           decaimento radionativo.}
            \begin{questionario}
                \qq{Quantos gramas havia inicialmente?}
                \qq{Quantos gramas permanecem depois de 10 anos? Use, caso
                    necess\'ario, $\frac{1}{\sqrt[3]{2}}\approx 0,794$.}
                \qq{Quantos anos levar\'a para reduzir pela metade a quantidade
                    inicial de c\'esio 137?}
            \end{questionario}
        %\q{Determine os valores m\'aximo e m\'{\i}nimo de $f(x)=\sen x+2\cos x$.}
        \q{Se $\tg x+\sec x=\frac{3}{2}$, calcule $\sen x$ e $\cos x$.}
    \end{questionario}
\end{document}
