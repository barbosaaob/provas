\documentclass{prova}

\usepackage{amssymb}

\professor{Adriano Barbosa}
\disciplina{Ver\~ao 2018}
\avaliacao{de N\'umeros e Fun\c{c}\~oes}
\curso{PROFMAT}
\data{22/01/2018}

\begin{document}
	\cabecalho{5}  % o numero 5 indica a qnt de quadros na tabela de nota
	\begin{questionario}
		% \q{}
            % \begin{questionario}
                % \qq{Seja $f:\mathbb{R}\rightarrow\mathbb{R}$, $f(x)=ax+b$. Se
                    % $f(k)\in\mathbb{Z}$ para todo $k\in\mathbb{Z}$, mostre que
                    % $a$ e $b$ s\~ao inteiros.}
                % \qq{Seja $f:\mathbb{R}\rightarrow\mathbb{R}$, $f(x)=ax^2+bx+c$.
                    % Se $f(k)\in\mathbb{Z}$ para todo $k\in\mathbb{Z}$, podemos
                    % afirmar que $a$,$b$ e $c$ s\~ao todos inteiros?}
            % \end{questionario}
        \q{Seja $f:\mathbb{R}\rightarrow\mathbb{R}$ uma fun\c{c}\~ao crescente tal
            que, para todo $x$ racional, vale $f(x)=ax+b$ (com $a$,
            $b\in\mathbb{R}$ constantes). Prove que se tem $f(x)=ax+b$ tamb\'em
            se $x$ for irracional.}
        \q{Um corpo est\'a impregnado de uma subst\^ancia radioativa cuja meia-vida
            \'e um ano. Quanto tempo levar\'a para que sua radioatividade se reduza
            a $20\%$ do que \'e?}
        \q{Se $a$ \'e irracional, prove que a fun\c{c}\~ao
            $f:\mathbb{R}\rightarrow\mathbb{R}$, $f(x) = \cos(ax) + \cos(x)$
            n\~ao \'e peri\'odica.}
        \q{Seja $f(x)=ax^2+bx+c$ uma fun\c{c}\~ao quadr\'atica com $a>0$ e
            $\Delta=b^2-4ac>0$. Considere o tri\^angulo $ABV$, onde $A$ e $B$
            s\~ao os pontos de interse\c{c}\~ao da par\'abola correspondente ao gr\'afico
            de $f$ com o eixo das abcissas e $V$ \'e o v\'ertice da par\'abola.}
            \begin{questionario}
                \qq{Mostre que
                    $\overline{BV}=\displaystyle\frac{\sqrt{\Delta(\Delta+4)}}{4a}$.}
                \qq{Mostre que o tri\^angulo $ABV$ \'e equil\'atero se, e somente se,
                    $\Delta=12$.}
            \end{questionario}
        \q{Sejam $x$ e $y$ n\'umeros reais positivos tais que $x+y=1$. Prove que
            $\displaystyle\left(1+\frac{1}{x}\right)\left(1+\frac{1}{y}\right)\ge
            9$.}
	\end{questionario}
\end{document}
